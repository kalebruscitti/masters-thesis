

\begin{abstract}
	We review the geometric quantization of the moduli space $\MM$ of flat $SU(2)$ connections on a compact Riemann surface in the real polarisation of Weitsman \cite{weitsman_real_1992}\cite{jeffrey_bohr-sommerfeld_1992}. We also use the methods of \cite{ramadas_comments_1989} to construct a line bundle over the toric variety $\cP$ associated to the moment polytope of Jeffrey and Weitsman's integrable system on the moduli space, which is compatible with the prequantum line bundle on $\MM$. There is a degeneration of the moduli space to this toric variety due to Biswas and Hurtubise \cite{biswas_degenerations_2021}, and we discuss how this degeneration might be used to prove results about the real and Kaehler polarisations of the moduli space.
	
	\begin{center}\textbf{Résumé}\end{center}
	
	\noindent Nous donnons un survol de la quantification géométrique de l'espace de modules $\MM$ des connexions $SU(2)$ plate sur une surface de Riemann compacte, dans une polarisation réelle de Weitsman \cite{weitsman_real_1992}\cite{jeffrey_bohr-sommerfeld_1992}. Aussi, nous utilisons les méthodes de \cite{ramadas_comments_1989} pour construire un fibré en droites sur la variété torique $\cP$ associée à l'application moment de le système intégrable de Jeffrey et Weitsman sur l'espace de modules, qui est compatible avec le fibré en droites préquantique sur $\MM$. Il y a un déformation de l'espace de modules à la variété torique, donnée par Biswas et Hurtubise \cite{biswas_degenerations_2021}, et nous discutons de la façon dont cette déformation peut être utilisée pour prouver des résultats concernant les polarisations réelle et complexe de l'espace de modules.
\end{abstract}

	The author would like to acknowledge the assistance of my fellow graduate students at McGill with understanding many of the concepts in this work. In particular, I'd like to thank Bartosz Syroka and Carlos Valero for many enlightening conversations, and Yuzhen Cao for assistance understanding the concepts in Chapters 2 and 3. In addition, the academic and mathematical supervision provided by Jacques Hurtubise was a key element in the production of this document. 
	
	Chapters 2 through 4 are entirely based on previous mathematical work. The author's contribution to these chapters is merely exposition and elaboration on previous work, as well as computations in some specific cases. Chapter 5 is also primarily based on the previous work of Biswas and Hurtubise, but with some new ideas contributed by the author, specifically in Sections \ref{s:l0-def} and \ref{s:holosecs}. The conclusions in Chapter 6 are the authors own mathematical thought.
	
	
	
	
	