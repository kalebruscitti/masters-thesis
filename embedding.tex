\documentclass[]{article}

%opening
\title{no title}
\date{\today}
\usepackage{amssymb}
\usepackage{amsmath}
\usepackage{amsthm}
\usepackage{tikz-cd}
\usepackage{quiver}
\usepackage{mathtools}

\newtheorem{lemma}{Lemma}
\newtheorem{theorem}{Theorem}
\newtheorem{definition}{Definition}

\newcommand{\C}{\mathbb{C}}
\newcommand{\Hom}{\text{Hom}}
\newcommand{\Ann}{\text{Ann}}
\newcommand{\OO}{\mathcal{O}}
\newcommand{\LL}{\mathcal{L}}
\newcommand{\MM}{\mathcal{M}}
\newcommand{\cP}{\mathcal{P}}
\newcommand{\End}{\text{End}~}
\newcommand{\coker}{\text{coker}~}
\newcommand{\dbar}{\overline{\partial}}
\newcommand{\PP}{\mathbb{P}}
\newcommand{\Tr}{\text{Tr }}
\newcommand{\HH}{\mathbb{H}}
\newcommand{\sslash}{\mathbin{/\mkern-4mu/}}

\begin{document}
	\subsection{Projective Embedding of $\MM$}
	The moduli $\MM$ of flat $SL(n,\C)$ bundles can be embedded into projective space, as we will now describe following the work of Thaddeus and Giesker (cite). First, fix a line bundle $L$ of sufficiently high degree so that for all $E\in N(k,d)$ $E\otimes L$  is globally generated (and redefine $E$). Then, for some large $N$ we can write $E$ as a quotient:
	\begin{equation}
	\phi:\OO^N \to E.
	\end{equation}
	This quotient induces a map from $\wedge^k(\OO^N) \to \wedge^k(E)$ and the $SL(k,\C)$ structure induces an isomorphism $\mu:\wedge^k(E)\cong L^k$. Hence, a quotient of the trivial bundle induces an element $\hat{\beta}$ in 
	\begin{equation}
	V_1 := \Hom(H^0(\wedge^k(\OO^N)), H^0(L^2)).
	\end{equation}
	Now to pass to $\MM$ we quotient by $GL(k,\C)$. Suppose we have $\phi_2 = \Lambda^{-1} \phi_1 \Lambda$. Then
	\begin{equation}
		\hat{\beta}_2 = \mu(\wedge^k \phi_2) = \mu(\wedge^k \Lambda^{-1}\phi_1 \Lambda) = \det\Lambda \mu(\wedge^k \phi_1) = \det\Lambda~ \hat{\beta}_1.
	\end{equation}
	Therefore the orbits of $GL(k,\C)$ correspond to equivalence classes in $\PP(V_1)$. It is this mapping, which we will denote $\iota:\MM\to \PP(V_1)$, which we claim is an embedding. 
	\begin{lemma}
		The map $\iota:\MM \to \PP(V_1)$ is injective.
	\end{lemma}
	\begin{proof}
		 Suppose $(E_1, \dbar_{E_1})$ and $(E_2, \dbar_{E_2})$ are holomorphic vector bundles for which $\hat{\beta}_1 = \iota(E_1) = \iota(E_2) = \hat{\beta_2}$. Since the $SL(k,\C)$ structure is unique up to $\C^\ast$, this means that
		\begin{equation}
		\wedge^k \phi_1 = \lambda \wedge^k \phi_2
		\end{equation}
		for some $\lambda \in \C^\ast$. Fixing a local trivialization of $E_1$, $E_1|_U \cong U\times \C^k$, and picking a local frame $e_1,..,e_k$, choose any sections $s_1,...,s_k \in \OO^N|_{U}$ so that $\phi_1(e_i) = s_i$. Let $s = s_1 \wedge s_2 \wedge ... \wedge s_k$. Then
		\begin{equation}
		e_1\wedge...\wedge s_k=\phi_1(s_1)\wedge...\wedge \phi_1(s_k)= \wedge^k \phi_1(s) = \lambda \wedge^k \phi_2(s) = \lambda \phi_2(s_1)\wedge...\wedge \phi_2(s_k)
		\end{equation}
		Since $\{e_i\}$ was a local frame, the LHS is non-zero, and thus the RHS is non-zero. Therefore $\{\phi_2(s_i)\}$ is a local frame trivializing $E_2$ on $U$. This map $e_i \to \phi_2(s_2)$ gives an isomorphism of $E_1$ with $E_2$. Note that this isomorphism is not unique as we could have picked other sections $\tilde{s}_i$ with $\phi_1(\tilde{s}_i) = e_i$.
	\end{proof}
	\begin{theorem}
		The map $\iota:\MM \to \PP(V_1)$ is an embedding.
	\end{theorem}
	\begin{proof}
		Since the lemma shows it is injective, it remains to show its derivative is everywhere injective. We give a proof following Thaddeus (GIT and flips page 717).
	\end{proof}

	\subsection{Parabolic Vector Bundles}

	Just as we constructed the moduli space of vector bundles over a Riemann surface, let us now construct that of parabolic vector bundles. We base our work on that of Hurtubise and Jeffrey (cite). Again let $\Sigma$ be a compact connected Riemann surface and fix a line bundle $L$ of sufficiently high degree so that for all $E\in N(k,d), E\otimes L$ is globally generated (and redefine $E$). Then for some large $N$ we can write $E$ as a quotient:
	\begin{equation}
		\phi: \OO^N \to E.
	\end{equation}
	As before, we have a mapping $\hat{\beta}$ taking $E$ to $V_1$. Now we add the parabolic data. At a point $p_i$, the map $\alpha_i:E\to \C_{p_i}$ pulls back to $\hat{\alpha}_i = \alpha_i\circ \phi$ in 
	\begin{equation}
		V_2 = H^0(\OO^N)^\ast.
	\end{equation}
	Since we are only interested in $\alpha_i$ up to (independent) scaling, the parabolic bundle $(E,\alpha)$ represents an equivalence class in
	\begin{equation}
		Z := \PP(V_1) \times \PP(V_2) \times ... \times \PP(V_2),
	\end{equation}
	where there are $n$ copies of $\PP(V_2)$, one for each puncture. Then letting $\tilde{\MM}$ denote the space of parabolic vector bundles, and $\tilde{\iota}:\tilde{\MM}\to Z$ denote the map taking $(E,\alpha)$ to $(\hat{\beta},\hat{\alpha})$, we get a closed subvariety $X = \tilde{\iota}\left(\tilde{\MM}\right)$ in $Z$. 

	Now we also have weights $\gamma = (\gamma_1,...,\gamma_n)$ for the parabolic structure. These vary the choice of polarization of $X$, namely the choice of line bundles on which the action of $SL(N,\C)$ \textit{linearises}. Let us recall what this means:
	\begin{definition}
		Given a linear algebraic group $G$ and a $G$-variety $X$, a line bundle $p:L\to X$ \emph{linearises} if there is an action of $G$ on $L$ such that for all $l\in L$, $g\in G$,
		\begin{equation}
			p(g\cdot l) = g\cdot p(l),
		\end{equation}
		and which restricts to a linear isomorphism $L_x \cong L_{g\cdot x}$ on fibres.
	\end{definition}
	In this case $G\cong SL(N,\C)$ which acts on $V_1$ and each copy of $V_2$.
	
	
Let $\pi_1:Z \to \PP(V_1)$ and $\pi_{2,i}:Z\to \PP(V_2)$ denote the projections to the first factor and to the i-th factor of $Z$ respectively. Let
\begin{equation}
	L_0 = \pi_1^\ast(\OO(N)),~ \text{ and } ~ L_{1,i} = \pi_1^\ast(\OO(N-1))\otimes \pi_{2,i}^\ast (\OO(2)).
\end{equation}
Then the linearisation corresponding to weights $\gamma = (\gamma_1,...,\gamma_n)$ is 
\begin{equation}
	L_\gamma = (L_0)^{s_0} \otimes \left(
	\otimes_{i=1}^n (L_{1,i})^{s_{1,i}}
	\right),
\end{equation}
where $s_0(\gamma_i) = s_{1,i}(1-\gamma_i)$.
\begin{lemma}
	$L_i$ is a linearisation of the action of $G$ on $X$.
\end{lemma}
\begin{proof}
	????????????/
\end{proof}
In summary, what we have now are a collection of moduli spaces of parabolic bundles semi-stable with respect to weights $\gamma$, with $\gamma_i \neq 0,1$. We want to fit these spaces together, and in such a way that we can include $\gamma_i = 0,1$. To do this, we put a $(\PP^1)^n$-bundle over $X$,
\begin{equation}
	Y = \PP(L_0\oplus L_{1,1})\oplus \PP(L_0\oplus L_{1,2})\oplus ... \oplus \PP(L_0\oplus L_{1,n}).
\end{equation}
We endow $Y$ with the natural polarisation $\OO(1,1,...,1)$. Now $Y$ contains all the stable points for the various choice of weights, which correspond to the various possible holonomies of the unitary connections. We still need to account for gauge equivalence, which apriori suggests we take the $SL(N,\C)$ quotient. It will turn out that this is not quite correct, as we need to make an adjustment with $\gamma_i = 0,1$.

Consider the space of weighted parabolic bundles as quadruples $(E, \alpha_i, A_i, \gamma)$, where $\alpha_i:E\to \C_{p_i}$ is the parabolic structure, $A_i$ is a subspace of $E_{p_i}$, with $A_i = \ker\alpha_i$ whenever $\alpha_i \neq 0$ (equiv. $\gamma \neq 0$), and $\gamma$ are the weights as usual. When $\gamma_i \neq 0$, we have not added any new information. When $\gamma_i = 0$ we are adding a projective class $A_i$ for the parabolic structure even as the map vanishes.

On the other hand when $\gamma_1 = 1$, the sheaves can acquire torsion meaning they are no longer locally free. To handle this, we first need
\begin{lemma}[H\& J Lemma 4.11]
	Let $E_t$, $t\in\C$ be a family of coherent rank 2 sheaves over $\Sigma$, with $E_t$ locally free at $p$ for $t\neq 0$, and with $E_0$ having torsion subsheaf $\C_p$ near $p$. Let $\phi_t \in H^0(\Sigma, \wedge^2(E)^\ast)$ be a family of $SL(2,\C)$ structures on $E_t$. Then $\phi_0$ vanishes at $p$.
	\label{t:sl2-lemma}
\end{lemma}
\begin{proof}
	If $z$ is a local co-ordinate on $\Sigma$ on an open set containing $p$, such that $z(p)=0$, one can obtain $E_t$ locally (up to reparameterization) from the exact sequence 
	\begin{equation}
		\OO \xrightarrow{(0,t^k,z)} \OO \oplus \OO \oplus \OO \xrightarrow{~~~~~}E_t
	\end{equation}
	for some integer $k$. Then the $SL(2,\C)$ structures $\phi_t$ are multiples of $e_1^\ast \wedge (-ze^\ast_2 + t^k e^\ast_3)$, which vanishes when $z=t=0$.
\end{proof}
Since $Y$ is a projective bundle, we can freely tensor with line bundles and write $Y$ as the bundle
\begin{equation}
	Y = \bigoplus_{i=1}^n\PP\left(\pi_1^\ast\left(
	\OO(-1)
	\right)\oplus \pi_{2,i}^\ast\left(\OO(-2)\right)\right).
\end{equation}
In this form, when $\gamma \neq 1$ we have a natural lift of a parabolic bundle $(E,\alpha)\in X$ to $Y$, given by 
\begin{equation}
	\hat{E} = \left(
	(\hat{\beta}, \hat{\alpha}_1^2),(\hat{\beta},\hat{\alpha}_2^2),...,(\hat{\beta},\hat{\alpha}^2)
	\right),
\end{equation}
which we want to extend to the torsion case where $\gamma = 1$. When we have torsion at say $p_i$ ($\gamma_i = 1)$, we could rescale the torsion subsheaf of $E$, modifying $\hat{\alpha}_i^2$ to $c\hat{\alpha}_i^2$ for some $c$. This rescaling should remain in the same equivalence class, and if $\hat{\beta} \neq 0$ then since $\hat{\beta}$ is not rescaled this would not be the case. Therefore we want that the $i$-th component of the lift of $E$ in $Y$ should be $(0,\hat{\alpha}_i^2)$. This is achieved as follows; recall that $\hat{\beta}$ is defined by composing
\begin{equation}
	\wedge^2(\OO^N) \xrightarrow{\phi} H^0(\Sigma, \wedge^2 E) \xrightarrow{\xi} H^0(\Sigma, L^2)
\end{equation}
where $\phi:\OO^N \to E$ is a quotient defining $E$, and $\xi$ was its $SL(2,\C)$ structure. We have a commutative diagram:
\begin{equation}
	TBD
\end{equation}
Then for $\gamma \neq 1$ one has $\xi = \text{ev}_\OO^{-1}\circ \xi^{p_i} \circ \text{ev}_{\wedge^2(E)}$, and so for $\gamma_i=1$ we use this definition to define $\xi^{p_i}$ at the torsion points of $E$. Then, from lemma \ref{t:sl2-lemma}, one has that $\hat{\beta}_i$ vanishes when there is torsion at $p_i$. Using this definition we can define a lift for all $(E,\alpha)\in X$ to $Y$ by $
\hat{E} = \left(
(\hat{\beta}, \hat{\alpha}_1^2),(\hat{\beta},\hat{\alpha}_2^2),...,(\hat{\beta},\hat{\alpha}^2)
\right).$

Next Hurtubise and Jeffrey analyse which elements of $Y$ are stable or semistable as weighted parabolic bundles. In lemma (ref) we saw that torsion in the kernal of $\alpha_i$ destabilised $(E,\alpha) \in X$. The same is true in $Y$:
\begin{lemma}[H\&J Lemma 4.13]
	A semi-stable element $y\in Y$ corresponds to a bundle parabolic bundle $(E,\alpha)$ with no torsion in the kernal of $\alpha$.
\end{lemma}
To allow the the $\alpha_i$ to go to zero but still preserve the information of which projective class we have at $p_i$, we define a new map. Let $(E,\alpha)\in X$ with $SL(2,\C)$ structure $\phi$. Then we can map $(E,\alpha,\phi)\to Y$ by
\begin{equation}
	(E,\alpha,\phi) \to \bigoplus_{i=1}^n\left(
	(\phi(p_i)^N, \phi(p_i)^{N-1}\alpha_i^2
	\right),
\end{equation}
and we define $\hat{Y}$ to be the closure in $Y$ of the image of this map. In this closure, we can take $\alpha_i =0$, but since $\phi(p_i)\neq 0$ we preserve the information of a subspace of $E$ at $p_i$ (unsure?). Then since $\phi(p_i)\neq 0$, lemma \ref{t:sl2-lemma} guarantees $E$ is torsion free at $p_i$. Finally the proposition regarding stability is
\begin{theorem}[H\&J Prop 4.14]
	Let 
	\begin{equation}
		y = \left(
		(b_1, a_1), (b_2, a_2),...,(b_n, a_n)
		\right)
	\end{equation}
	be a point in $\hat{Y}$. Let $\Gamma(y)$ be the set of $\gamma_i \in [0,1]$ such that $\gamma_i =0$ if $a_i =0$ and $\gamma_i = 0$ if $b_i = 0$. Then $y$ is semi-stable if and only if for one element $\gamma \in \Gamma(y)$, $\pi(y)\in X$ is $\gamma$-semi-stable.
\end{theorem}
Therefore semi-stable elements in $\hat{Y}$ all project down to semi-stable elements of $X$ for some choice of weights, and the GIT quotient $\hat{Y}\sslash SL(N,\C)$ corresponds to equivalence classes of quadruples $(E,\alpha_i, \hat{\alpha}_i, \phi)$ where $(E,\alpha)$ is a parabolic bundle, $\hat{\alpha}_i$ is a subspace of $E|_{p_i}$ (which is the kernal of $\alpha_i$ when $\alpha_i \neq 0$) and $\phi$ is an $SL(2,\C)$ structure. However this is not quite the final moduli space we want to construct, as we have added the extra information of $\hat{\alpha}_i$ when $\alpha_i = 0$. At these points, $\hat{\alpha}_i \in \PP_1=\PP(E|_{p_i})$. We want to collapse these extra $\PP_1$s. To do this, embed $V_1$ into $W_1 = V_1^{\otimes N}$ so that a non-zero element $l$ in $L_{0} = \pi_1^\ast(\OO(N))$ can be thought of as an element in $W_1^\ast$ by taking $v_1\otimes...\otimes v_n$ to $l(v_1)l(v_2)...l(v_n)$. Similarly, embedding $V_2$ into $W_2 = V_1^{\otimes N-1}\otimes V_2\otimes V_2$ allows us to think of non-zero elements in $L_{1,i}$ as elements of $W_2^\ast$. Then this maps $\hat{Y}$ to a subvariety $\tilde{Y}$ in $\bigotimes_{i=1}^n \PP(W_1\otimes W_2)$, and the map collapses the unwanted $\PP_1$s while being an embedding otherwise. 

Finally, we let $\cP = \tilde{Y}\sslash SL(N,\C)$ be the geometric quotient, and we call it the \emph{moduli space of framed parabolic bundles}.
	
\end{document}