\documentclass[]{article}

%opening
\title{no title}
\date{\today}
\usepackage{amssymb}
\usepackage{amsmath}
\usepackage{amsthm}
\usepackage{tikz-cd}
\usepackage{quiver}
\usepackage{mathtools}

\newtheorem{lemma}{Lemma}
\newtheorem{theorem}{Theorem}
\newtheorem{definition}{Definition}

\newcommand{\C}{\mathbb{C}}
\newcommand{\Hom}{\text{Hom}}
\newcommand{\Ann}{\text{Ann}}
\newcommand{\OO}{\mathcal{O}}
\newcommand{\LL}{\mathcal{L}}
\newcommand{\MM}{\mathcal{M}}
\newcommand{\End}{\text{End}~}
\newcommand{\coker}{\text{coker}~}
\newcommand{\dbar}{\overline{\partial}}
\newcommand{\PP}{\mathbb{P}}
\newcommand{\Tr}{\text{Tr }}
\newcommand{\HH}{\mathbb{H}}
\newcommand{\sslash}{\mathbin{/\mkern-4mu/}}

\begin{document}
	\subsection{Projective Embedding of $\MM$}
	The moduli $\MM$ of flat $SL(n,\C)$ bundles can be embedded into projective space, as we will now describe following the work of Thaddeus and Giesker (cite). First, fix a line bundle $L$ of sufficiently high degree so that for all $E\in N(k,d)$ $E\otimes L$  is globally generated (and redefine $E$). Then, for some large $N$ we can write $E$ as a quotient:
	\begin{equation}
	\phi:\OO^N \to E.
	\end{equation}
	This quotient induces a map from $\wedge^k(\OO^N) \to \wedge^k(E)$ and the $SL(k,\C)$ structure induces an isomorphism $\mu:\wedge^k(E)\cong L^k$. Hence, a quotient of the trivial bundle induces an element $\hat{\beta}$ in 
	\begin{equation}
	V_1 := \Hom(H^0(\wedge^k(\OO^N)), H^0(L^2)).
	\end{equation}
	Now to pass to $\MM$ we quotient by $GL(k,\C)$. Suppose we have $\phi_2 = \Lambda^{-1} \phi_1 \Lambda$. Then
	\begin{equation}
		\hat{\beta}_2 = \mu(\wedge^k \phi_2) = \mu(\wedge^k \Lambda^{-1}\phi_1 \Lambda) = \det\Lambda \mu(\wedge^k \phi_1) = \det\Lambda~ \hat{\beta}_1.
	\end{equation}
	Therefore the orbits of $GL(k,\C)$ correspond to equivalence classes in $\PP(V_1)$. It is this mapping, which we will denote $\iota:\MM\to \PP(V_1)$, which we claim is an embedding. 
	\begin{lemma}
		The map $\iota:\MM \to \PP(V_1)$ is injective.
	\end{lemma}
	\begin{proof}
		 Suppose $(E_1, \dbar_{E_1})$ and $(E_2, \dbar_{E_2})$ are holomorphic vector bundles for which $\hat{\beta}_1 = \iota(E_1) = \iota(E_2) = \hat{\beta_2}$. Since the $SL(k,\C)$ structure is unique up to $\C^\ast$, this means that
		\begin{equation}
		\wedge^k \phi_1 = \lambda \wedge^k \phi_2
		\end{equation}
		for some $\lambda \in \C^\ast$. Fixing a local trivialization of $E_1$, $E_1|_U \cong U\times \C^k$, and picking a local frame $e_1,..,e_k$, choose any sections $s_1,...,s_k \in \OO^N|_{U}$ so that $\phi_1(e_i) = s_i$. Let $s = s_1 \wedge s_2 \wedge ... \wedge s_k$. Then
		\begin{equation}
		e_1\wedge...\wedge s_k=\phi_1(s_1)\wedge...\wedge \phi_1(s_k)= \wedge^k \phi_1(s) = \lambda \wedge^k \phi_2(s) = \lambda \phi_2(s_1)\wedge...\wedge \phi_2(s_k)
		\end{equation}
		Since $\{e_i\}$ was a local frame, the LHS is non-zero, and thus the RHS is non-zero. Therefore $\{\phi_2(s_i)\}$ is a local frame trivializing $E_2$ on $U$. This map $e_i \to \phi_2(s_2)$ gives an isomorphism of $E_1$ with $E_2$. Note that this isomorphism is not unique as we could have picked other sections $\tilde{s}_i$ with $\phi_1(\tilde{s}_i) = e_i$.
	\end{proof}
	\begin{theorem}
		The map $\iota:\MM \to \PP(V_1)$ is an embedding.
	\end{theorem}
	\begin{proof}
		Since the lemma shows it is injective, it remains to show its derivative is everywhere injective. We give a proof following Thaddeus (GIT and flips page 717).
	\end{proof}

	\subsection{Parabolic Vector Bundles}

	Just as we constructed the moduli space of vector bundles over a Riemann surface, let us now construct that of parabolic vector bundles. We base our work on that of Hurtubise and Jeffrey (cite). Again let $\Sigma$ be a compact connected Riemann surface and fix a line bundle $L$ of sufficiently high degree so that for all $E\in N(k,d), E\otimes L$ is globally generated (and redefine $E$). Then for some large $N$ we can write $E$ as a quotient:
	\begin{equation}
		\phi: \OO^N \to E.
	\end{equation}
	As before, we have a mapping $\hat{\beta}$ taking $E$ to $V_1$. Now we add the parabolic data. At a point $p_i$, the map $\alpha_i:E\to \C_{p_i}$ pulls back to $\hat{\alpha}_i = \alpha_i\circ \phi$ in 
	\begin{equation}
		V_2 = H^0(\OO^N)^\ast.
	\end{equation}
	Since we are only interested in $\alpha_i$ up to (independent) scaling, the parabolic bundle $(E,\alpha)$ represents an equivalence class in
	\begin{equation}
		Z := \PP(V_1) \times \PP(V_2) \times ... \times \PP(V_2),
	\end{equation}
	where there are $n$ copies of $\PP(V_2)$, one for each puncture. Then letting $\tilde{\MM}$ denote the space of parabolic vector bundles, and $\tilde{\iota}:\tilde{\MM}\to Z$ denote the map taking $(E,\alpha)$ to $(\hat{\beta},\hat{\alpha})$, we get a closed subvariety $X = \tilde{\iota}\left(\tilde{\MM}\right)$ in $Z$. 

	Now we also have weights $\gamma = (\gamma_1,...,\gamma_n)$ for the parabolic structure. These vary the choice of polarization of $X$, namely the choice of line bundles on which the action of $SL(N,\C)$ \textit{linearises}. Let us recall what this means:
	\begin{definition}
		Given a linear algebraic group $G$ and a $G$-variety $X$, a line bundle $p:L\to X$ \emph{linearises} if there is an action of $G$ on $L$ such that for all $l\in L$, $g\in G$,
		\begin{equation}
			p(g\cdot l) = g\cdot p(l),
		\end{equation}
		and which restricts to a linear isomorphism $L_x \cong L_{g\cdot x}$ on fibres.
	\end{definition}
	In this case $G\cong SL(N,\C)$ which acts on $V_1$ and each copy of $V_2$.
	
	
Let $\pi_1:Z \to \PP(V_1)$ and $\pi_{2,i}:Z\to \PP(V_2)$ denote the projections to the first factor and to the i-th factor of $Z$ respectively. Let
\begin{equation}
	L_0 = \pi_1^\ast(\OO(N)),~ \text{ and } ~ L_{1,i} = \pi_1^\ast(\OO(N-1))\otimes \pi_{2,i}^\ast (\OO(2)).
\end{equation}
Then the linearisation corresponding to weights $\gamma = (\gamma_1,...,\gamma_n)$ is 
\begin{equation}
	L_\gamma = (L_0)^{s_0} \otimes \left(
	\otimes_{i=1}^n (L_{1,i})^{s_{1,i}}
	\right),
\end{equation}
where $s_0(\gamma_i) = s_{1,i}(1-\gamma_i)$.
\begin{lemma}
	$L_i$ is a linearisation of the action of $G$ on $X$.
\end{lemma}
\begin{proof}
	????????????/
\end{proof}
In summary, what we have now are a collection of moduli spaces of parabolic bundles semi-stable with respect to weights $\gamma$, with $\gamma_i \neq 0,1$. We want to fit these spaces together, and in such a way that we can include $\gamma_i = 0,1$. To do this, we put a $(\PP^1)^n$-bundle over $X$,
\begin{equation}
	Y = \PP(L_0\oplus L_{1,1})\oplus \PP(L_0\oplus L_{1,2})\oplus ... \oplus \PP(L_0\oplus L_{1,n}).
\end{equation}
We endow $Y$ with the natural polarisation $\OO(1,1,...,1)$. 
\end{document}