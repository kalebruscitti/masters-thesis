\documentclass[]{article}

%opening
\title{no title}
\date{\today}
\usepackage{amssymb}
\usepackage{amsmath}
\usepackage{tikz-cd}
\usepackage{quiver}
\usepackage{mathtools}

\newcommand{\C}{\mathbb{C}}
\newcommand{\Hom}{\text{Hom}}
\newcommand{\Ann}{\text{Ann}}
\newcommand{\OO}{\mathcal{O}}
\newcommand{\LL}{\mathcal{L}}
\newcommand{\MM}{\mathcal{M}}
\newcommand{\End}{\text{End}~}
\newcommand{\coker}{\text{coker}~}
\newcommand{\dbar}{\overline{\partial}}
\newcommand{\PP}{\mathbb{P}}
\newcommand{\Tr}{\text{Tr }}
\newcommand{\HH}{\mathbb{H}}

\begin{document}
	\subsection{Projective Embedding of $\MM$}
	The moduli $\MM$ of flat $SL(n,\C)$ bundles can be embedded into projective space, as we will now describe following the work of Thaddeus, Giesker and Bertram, Daskalopoulos and Wentworth. First, fix a line bundle $L$ of sufficiently high degree so that for all $E\in N(k,d)$ $E\otimes L$  is globally generated (and redefine $E$). Then, for some large $N$ we can write $E$ as a quotient:
	\begin{equation}
	\phi:\OO^N \to E.
	\end{equation}
	This quotient induces a map from $\wedge^k(\OO^N) \to \wedge^k(E)$ and the $SL(k,\C)$ structure induces an isomorphism $\mu:\wedge^k(E)\cong L^k$. Hence, a quotient of the trivial bundle induces an element $\hat{\beta}$ in 
	\begin{equation}
	V_1 := \Hom(H^0(\wedge^k(\OO^N)), H^0(L^2)).
	\end{equation}
	Now to pass to $\MM$ we quotient by $GL(k,\C)$. Suppose we have $\phi_2 = \Lambda^{-1} \phi_1 \Lambda$. Then
	\begin{equation}
		\hat{\beta}_2 = \mu(\wedge^k \phi_2) = \mu(\wedge^k \Lambda^{-1}\phi_1 \Lambda) = \det\Lambda \mu(\wedge^k \phi_1) = \det\Lambda~ \hat{\beta}_1.
	\end{equation}
	Therefore the orbits of $GL(k,\C)$ correspond to equivalence classes in $\PP(V_1)$. It is this mapping, which we will denote $T:\MM\to \PP(V_1)$, which we claim is an embedding. Suppose that $T(E) = T(F)$ where $\phi_1:\OO^N \to E$ and $\phi_2:\OO^N \to F$. Taking out any point $p\in \Sigma$, $E$ and $F$ are free on $\Sigma - p$. Let $s_1,...,s_N$ be a set which globally generates $E$. Then the sections $\{s_{i_1}\wedge...\wedge s_{i_k}\}$ form a map from $\Sigma - p$ to the Grassmannian $G(N,N-k)$
	\iffalse
	The moduli $\MM$ of flat $SL(n,\C)$ bundles can be embedded into projective space, as we will now describe following the work of Thaddeus, Giesker and Bertram, Daskalopoulos and Wentworth. First, fix a line bundle $L$ of sufficiently high degree that all $E\otimes L$  are globally generated (and redefine $E$). Then, for some large $N$ we can write $E$ as a quotient:
	\begin{equation}
		\phi:\OO^N \to E.
	\end{equation}
	This quotient induces a map from $\wedge^2(\OO^N) \to \wedge^2(E)$ and the $SL(2,\C)$ structure induces an isomorphism $\wedge^2(E)\cong L^2$. Hence, a quotient of the trivial bundle induces an element $\hat{\beta}$ in 
	\begin{equation}
		V_1 := \Hom(H^0(\wedge^2(\OO^N)), H^0(L^2)).
	\end{equation}
	There exists a scheme $Quot_{\OO^N}$ which parameterizes the isomorphism classes of quotients of $\OO^N$ (citation grothendieck). Let $R^{ss}\subset Quot$ be those which are semi-stable. Let $F$ be a bundle over $Quot$ with fibre $V_1$, and let $\det:E\to L^2$ be the composition of the determinant with the $SL(2,\C)$ structure. Then we have a direct image bundle $Z = \det_\ast F$ over $\text{Jac}^d(\Sigma)$, with fiber at $L^2$ given by
	\begin{equation}
	Z_{\det E = L^2} = F_E = V_1.
	\end{equation}
	The map above taking the quotient $\phi$ to $\hat{\beta}$ can then be thought of as a map $T:R^{ss} \to Z$. We have an action of $SL(N,\C)$ on $R^{ss}$ and $Z$ and the map  $T:R^{ss}\to $ is a $SL(N,\C)$ morphism. Thaddeus claims there is a linearization $\LL$ on $Z$ such that $T^{-1}Z^{ss}(\LL) = R^{ss}$ and $T$ is finite. Thaddeus shows that this map is an embedding (c.f. page 717), and from GIT it shows $R^{ss}$ has a good quotient by $SL(N,\C)$, recovering the moduli space $\MM$ of semistable bundles on $\Sigma$.
	
	Suppose $\phi_1$ and $\phi_2$ differ by conjugation in $SL(N,\C)$. Then since $T$ involves taking $\det \phi$ we will have $T(\phi_1) = T(\phi_2)$. On the other hand, if $T(\phi_1) = \lambda T(\phi_2)$ for some $\lambda \in \C^\ast$, we can find 
	
	
	Suppose we have a deformation of $\phi$ by $\psi_1:\OO^N \to E$, namely $\phi(\epsilon) = \phi + \epsilon\psi_1$ with $\epsilon^2=0$. Suppose also we have a section $s_0 \in H^0(\OO^N)$ deformed by $s_1 \in H^0(\OO^N)$ to $s(\epsilon) = s_0 + \epsilon(s_1)$. Then we can compute the differential of the determinant map;
	\begin{align*}
		dT\psi_1(s_1\wedge s_2) :&= \frac{d}{d\epsilon}|_{\epsilon=0} T(\phi(\epsilon))(s_1\wedge s_2)\\
		&=\frac{d}{d\epsilon}|_{\epsilon=0}\phi(\epsilon)(s_1)\wedge \phi(\epsilon)(s_2)\\
		&= \phi(s_1) \psi_1(s_2) +(-??) \psi_1(s_1)\phi(s_2)\\
		&= \phi\wedge \psi_1
	\end{align*}
	Our map $\phi$ induces a map $\phi^\ast :H^1(\End E) \to H^1(E\otimes \OO^N)$. Since $\OO^N$ is trivial, $H^1(\OO^N) =0$. That is to say, given $\alpha \in H^1(\End E)$ we have that $\phi^\ast(\alpha) = \alpha \phi = \dbar_E \psi$ for some $\psi \in H^0(E\otimes \OO^N)$.  Suppose that there are two choices $\alpha_1$ and $\alpha_2$ for which $\alpha_1 \phi = \alpha_2 \phi = \dbar_E \psi$. Then $(\alpha_1 - \alpha_2)\phi = 0$. If $v\in \OO^N, v\neq 0$ and $(\alpha_1 - \alpha_2)\phi(v) = 0$ then either $\phi(v)=0$ or $\alpha_1(\phi(v)) = \alpha_2(\phi(v))$. Since $E := \OO^N / \ker(\phi)$, this means that $\alpha_1 = \alpha_2$ for every element of $E$. Thus, it makes sense to define a symplectic form on $R^{ss}$ by
	\begin{equation}
		\omega_R(\psi_1,\psi_2) = \omega_\MM(\alpha, \beta) = \int\limits_{\Sigma} \Tr(A^\dagger B) ~dz \wedge d\bar{z} = \int\limits_{\Sigma} \sum_{k=0}^{\dim E} g_E\langle Be_k, Ae_k\rangle ~dz\wedge d\bar{z}.
	\end{equation}
	\subsection{Fubini-Study Metric on $\PP(V_1)$}
	To define a Fubini-Study metric on $\PP(V_1)$ we first need to define an inner product on $V_1 = \Hom(H^0(\wedge^2 \OO^N), H^0(L^2))$. Typically, given vector spaces $V,W$ with inner products $\langle,\rangle_V$ and $\langle,\rangle_W$, one defines an inner product on $\Hom(V,W)$ by
	\begin{equation}
		\langle A, B \rangle = \Tr(A^\dagger B) = \sum_{k=0}^{\dim V} \langle A^\dagger B e_k, e_k\rangle_V = \sum_{k=0}^{\dim V} \langle Be_k, Ae_k\rangle_W, 
	\end{equation}
	where $\{e_k\}$ is an orthonormal basis for $V$ with respect to $\langle,\rangle_V$. Thus, to put an inner product on $V_1$, we first define ones for $H^0(\wedge^2 \OO^N)$ and $H^0(L^2)$. By assumption, we have a metric $g$ on $E$, which we can use to define a metric on $\wedge^2 E$ by 
	\begin{align*}
		g_{\wedge^2 E}(\alpha \wedge \beta, \sigma\wedge \tau) &= \frac{1}{2}g(\alpha \otimes \beta - \beta\otimes \alpha, \sigma\otimes \tau - \tau \wedge \sigma)\\
		&= \frac{1}{2}\left(
		g(\alpha, \sigma)g(\beta,\tau) - g(\beta,\sigma)g(\alpha,\tau) - g(\alpha,\tau)g(\beta,\sigma) + g(\beta,\tau)g(\alpha,\sigma)
		\right)\\
		&= g(\alpha,\sigma)g(\beta,\tau) - g(\beta,\sigma)g(\alpha,\tau).
	\end{align*}
	We pullback to $\phi^\ast g_{\wedge^2 \OO^N}$, and finally we can integrate over $\Sigma$ to get 
	\begin{equation}
		\langle s_1 \wedge s_2, t_1\wedge t_2 \rangle_{H^0(\wedge^2\OO^N)} := \int\limits_{\Sigma} g_{\wedge^2 E}(\phi(s_1)\wedge \phi(s_2), \phi(t_1)\wedge \phi(t_2))(z) ~dz\wedge d\bar{z}.
	\end{equation}
	To put an inner product on $H^0(L^2)$, we follow the exact same process starting from any choice of metric $g_L$. 
	\fi
	\iffalse
	We now aim to compute the pullback of the Fubini-Study metric on $\PP(V_1)$ to $R^{ss}$. The inner product on $V_1$ which we use comes from the metric $\langle, \rangle_E$ on $E$ as follows; we define $\langle v_1, v_2 \rangle_{\OO^N} = \langle \phi(v_1), \phi(v_2)\rangle_E$ as a metric on $\OO^N$, and then we define 
	\begin{equation}
		\langle v_1 \wedge v_2, w_1 \wedge w_2 \rangle = \langle v_1, w_1\rangle\langle v_2,w_2\rangle_{\OO^N} - \langle v_1,w_2\rangle\langle v_2,w_1\rangle_{\OO^N}.
	\end{equation}
	as a metric on $\wedge^2 \OO^N$, and finally we can use this to define an inner product on $H^0(\wedge^2 \OO^N)$ by integrating this metric over $\Sigma$. Choosing a metric $\langle, \rangle_L$ on $L$, we get one for $L^2$ in a similar manner. This allows us to define the $\dagger$ operator on $V_1$ in the standard manner; by requiring that for $A\in V_1$ and all $s_1 \wedge s_2 \in H^0(\wedge^2 \OO^N)$ and $t_1\wedge t_2 \in L^2$, we have
	\begin{equation}
		\langle A(s_1 \wedge s_2), t_1\wedge t_2 \rangle = \langle s_1 \wedge s_2, A^\dagger(t_1 \wedge t_2)\rangle.
	\end{equation}
	Finally, we can then define the inner product on $V_1$ as $\Tr(A^\dagger B)$. In particular, for tangent vectors $\psi_1, \psi_2$ to $R^{ss}$, we have:
	\begin{align*}
		\Tr((\phi \wedge \psi_1)^\dagger (\phi \wedge \psi_2)) &= \Tr(\phi^\dagger \phi)\Tr(\psi_1^\dagger \psi_2) \\
		&= \Tr(\phi^\dagger \phi)\sum_{k} \langle \psi_1^\dagger\psi_2(e_k), e_k\rangle\\
		&= \Tr(\phi^\dagger \phi)\int\limits_{\Sigma} \sum_k \langle \psi_1^\dagger \psi_2(e_k)(z), e_k(z)\rangle_{\OO^N}~dz\wedge d\bar{z}\\
		&= \Tr(\phi^\dagger \phi)\int\limits_{\Sigma} \sum_k \langle 
		\psi_2 (e_k), \psi_1(e_k)
		\rangle_E ~dz\wedge d\bar{z}\\
	\end{align*}
	On the other hand, given $\phi_1$ tangent to $R^{ss}$ at $\phi$, we can obtain an element of $H^1(\End E)$, a vector tangent to $E_\phi$ in $\MM$. The tangent space to $Quot$ is the first hypercohomology group of the complex [c.f. thaddeus pg 717]
	\begin{equation}
		0 \to \End E \xrightarrow{\phi} E\otimes \OO^N \to 0 \to...
	\end{equation}
	Let us pick injective objects ${I_{i}}$ so that we have the resolution
	\[\begin{tikzcd}
		{\End E} & {E\otimes \OO^N} & 0 & {...} \\
		{I_1} & {I_2} & 0 & {...}
		\arrow[from=2-2, to=2-3]
		\arrow["\phi", from=1-1, to=1-2]
		\arrow[from=1-2, to=1-3]
		\arrow["\phi", from=2-1, to=2-2]
		\arrow[from=1-1, to=2-1]
		\arrow[from=1-2, to=2-2]
		\arrow[from=1-3, to=1-4]
		\arrow[from=2-3, to=2-4]
	\end{tikzcd}\]
	Then $\HH^1(\End E\xrightarrow{\phi} E\otimes \OO^N) := I_2/\phi(I_1)$. So a tangent vector to $(E,\phi)$ is determined? However, we also have the short exact sequence
	\begin{equation}
		\ker \phi \hookrightarrow \End E \to E\otimes \OO^N
	\end{equation}
	which, gives us the exact sequence in cohomology
	\[\begin{tikzcd}
		{...} & {H^1(\ker \phi)} & {H^1(\End E)} & {H^1(E\otimes \OO^N)} & {...}
		\arrow[from=1-2, to=1-3]
		\arrow[from=1-3, to=1-4]
		\arrow[from=1-1, to=1-2]
		\arrow[from=1-4, to=1-5]
	\end{tikzcd}.\]
	Then, since $\OO^N$ is a trivial bundle, $H^1(E\otimes \OO^N) = 0$ and we have a surjection $H^1(\ker\phi) \to H^1(\End E)$. That is to say, given $\alpha \in H^1(\End E)$ we have that $\alpha \phi = \dbar_E \psi$ for some $\psi \in H^0(E\otimes \OO^N)$.
	
	$\psi \in C^0(E\otimes \OO^N)$ defines $\dbar \psi \in C^1(E\otimes \OO^N)$ and then we should have an $\alpha \in C^1(\End E)$ for which $\phi^\ast \alpha = \dbar \psi$. Suppose $\alpha = A~dz$, and $v\in \Gamma(\OO^N)$.
	\begin{equation}
	A(\phi(v)) ~dz =\phi^\ast A(v) = \dbar_E \psi(v)= \frac{\partial \psi(v)}{\partial \bar{z}} ~d\bar{z} + \theta(\phi(v)) ~dz
	\end{equation}
	want roughly $\psi(\phi^{-1}(e)) = A$ i.e. $\psi = A\circ \phi$
	
	So modulo some conjugations(?) $\alpha(\phi(v)) = \frac{\partial\psi}{\partial \bar{z}}(v)$. Then symplectic form becomes
	\begin{align*}
		\omega(\alpha,\beta)&=\int\limits_{\Sigma} \Tr (A^\dagger B) ~ dz\wedge d\bar{z}\\
		&= \int\limits_{\Sigma}\sum_{k} \langle A^\dagger B e_k, e_k\rangle_E ~dz \wedge d\bar{z}
	\end{align*}
	\fi
	\iffalse
	Now we add in the parabolic structure. At a puncture $p_i$, the map $\alpha_i:E\to \C_{p_i}$ acts on global sections $\sigma \in \Gamma(E)$ by
	\begin{equation}
		\alpha_i(\sigma) = \alpha_i(\sigma(p_i)),
	\end{equation}
	and we call this action $\hat{\alpha_i}$, in 
	\begin{equation}
		V_2 := H^0(\OO^N)^\ast.
	\end{equation}
	We get one copy of $V_2$ for each puncture. Furthermore, parabolic structures are only up to the independent scale of the $\alpha_i$, so we represent the equivalence class of $(E,\alpha)$ as \begin{equation}
	[\hat{\beta}],[\hat{\alpha}])\in \PP(V_1)\times_{i=1}^n \PP(V_2).
	\end{equation}
	Let $X$ denote the closed subvariety of this space consisting of the image of Quot under the above injection.
	\fi
	
\end{document}