\documentclass[]{article}

%opening
\title{no title}
\date{\today}
\usepackage{amssymb}
\usepackage{amsmath}
\usepackage{amsthm}
\usepackage{tikz-cd}
\usepackage{quiver}
\usepackage{mathtools}

\newtheorem{lemma}{Lemma}
\newtheorem{theorem}{Theorem}
\newtheorem{definition}{Definition}

\newcommand{\C}{\mathbb{C}}
\newcommand{\Hom}{\text{Hom}}
\newcommand{\Ann}{\text{Ann}}
\newcommand{\OO}{\mathcal{O}}
\newcommand{\LL}{\mathcal{L}}
\newcommand{\MM}{\mathcal{M}}
\newcommand{\End}{\text{End}~}
\newcommand{\coker}{\text{coker}~}
\newcommand{\dbar}{\overline{\partial}}
\newcommand{\PP}{\mathbb{P}}
\newcommand{\Tr}{\text{Tr }}
\newcommand{\HH}{\mathbb{H}}
\newcommand{\sslash}{\mathbin{/\mkern-4mu/}}

\begin{document}
	\subsection{Projective Embedding of $\MM$}
	The moduli $\MM$ of flat $SL(n,\C)$ bundles can be embedded into projective space, as we will now describe following the work of Thaddeus and Giesker (cite). First, fix a line bundle $L$ of sufficiently high degree so that for all $E\in N(k,d)$ $E\otimes L$  is globally generated (and redefine $E$). Then, for some large $N$ we can write $E$ as a quotient:
	\begin{equation}
	\phi:\OO^N \to E.
	\end{equation}
	This quotient induces a map from $\wedge^k(\OO^N) \to \wedge^k(E)$ and the $SL(k,\C)$ structure induces an isomorphism $\mu:\wedge^k(E)\cong L^k$. Hence, a quotient of the trivial bundle induces an element $\hat{\beta}$ in 
	\begin{equation}
	V_1 := \Hom(H^0(\wedge^k(\OO^N)), H^0(L^2)).
	\end{equation}
	Now to pass to $\MM$ we quotient by $GL(k,\C)$. Suppose we have $\phi_2 = \Lambda^{-1} \phi_1 \Lambda$. Then
	\begin{equation}
		\hat{\beta}_2 = \mu(\wedge^k \phi_2) = \mu(\wedge^k \Lambda^{-1}\phi_1 \Lambda) = \det\Lambda \mu(\wedge^k \phi_1) = \det\Lambda~ \hat{\beta}_1.
	\end{equation}
	Therefore the orbits of $GL(k,\C)$ correspond to equivalence classes in $\PP(V_1)$. It is this mapping, which we will denote $\iota:\MM\to \PP(V_1)$, which we claim is an embedding. 
	\begin{lemma}
		The map $\iota:\MM \to \PP(V_1)$ is injective.
	\end{lemma}
	\begin{proof}
		 Suppose $(E_1, \dbar_{E_1})$ and $(E_2, \dbar_{E_2})$ are holomorphic vector bundles for which $\hat{\beta}_1 = \iota(E_1) = \iota(E_2) = \hat{\beta_2}$. Since the $SL(k,\C)$ structure is unique up to $\C^\ast$, this means that
		\begin{equation}
		\wedge^k \phi_1 = \lambda \wedge^k \phi_2
		\end{equation}
		for some $\lambda \in \C^\ast$. Fixing a local trivialization of $E_1$, $E_1|_U \cong U\times \C^k$, and picking a local frame $e_1,..,e_k$, choose any sections $s_1,...,s_k \in \OO^N|_{U}$ so that $\phi_1(e_i) = s_i$. Let $s = s_1 \wedge s_2 \wedge ... \wedge s_k$. Then
		\begin{equation}
		e_1\wedge...\wedge s_k=\phi_1(s_1)\wedge...\wedge \phi_1(s_k)= \wedge^k \phi_1(s) = \lambda \wedge^k \phi_2(s) = \lambda \phi_2(s_1)\wedge...\wedge \phi_2(s_k)
		\end{equation}
		Since $\{e_i\}$ was a local frame, the LHS is non-zero, and thus the RHS is non-zero. Therefore $\{\phi_2(s_i)\}$ is a local frame trivializing $E_2$ on $U$. This map $e_i \to \phi_2(s_2)$ gives an isomorphism of $E_1$ with $E_2$. Note that this isomorphism is not unique as we could have picked other sections $\tilde{s}_i$ with $\phi_1(\tilde{s}_i) = e_i$.
	\end{proof}
	\begin{theorem}
		The map $\iota:\MM \to \PP(V_1)$ is an embedding.
	\end{theorem}
	\begin{proof}
		Since the lemma shows it is injective, it remains to show its derivative is everywhere injective. We give a proof following Thaddeus (GIT and flips page 717).
	\end{proof}

	\subsection{Parabolic Vector Bundles}
	Following section 4 of (Hurtubise  and Jeffrey), we describe the moduli space of framed parabolic vector bundles.
	\begin{definition}
		A \emph{parabolic bundle} over a complex manifold $\Sigma$ is a holomorphic vector bundle $E$ over $\Sigma$ with a \emph{parabolic structure}, which is a point of marked points $\{p_1,...,p_n\}$ and for each point, a flag of subspaces in the fibre $E_{p_k}$. 
	\end{definition}
	In particular if $E$ has rank 2, then a parabolic structure on $E$ is a choice of points $\{p_k\}$ and a sheaf homomorphism $\alpha:E\to S$ where $S := \bigoplus_{k} \C_{p_k}$.
	
	There is an adapted notion of stability for parabolic vector bundles.
	\begin{definition}
		Let $\gamma_1,...,\gamma_n \in [0,1]$ be a set of weights. For a subbundle (not neccesarily proper) $F$ of $E$ we set $\mu_i(F) = 1$ if $F_{p_i} \subset \ker\alpha_i$, and $\mu_i = 0$ otherwise. Define $\sigma(F) = \frac{1}{rk(E)}$ if $F=E$ and $0$ otherwise. Then we say a pair $(E,\alpha)$ is \emph{stable} with respect to $\gamma$ if 
		\begin{equation}
		rk(E)\deg(F) < rk(F)\left(\deg(E) - \sum_{i=1}^n \gamma_i\right) 
		+ rk(E)\sum_{i=1}^n(1 - \mu_i(F) + \sigma_i(F))\gamma_i.
		\end{equation}
		If the inequality is not strict, $(E,\alpha)$ is \emph{semi-stable}.
	\end{definition}
	
	Just as we constructed the moduli space of vector bundles, let us now construct that of parabolic vector bundles.  
	
\end{document}