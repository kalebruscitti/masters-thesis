\documentclass[]{article}

%opening
\title{no title}
\date{\today}
\usepackage{amssymb}
\usepackage{amsmath}
\usepackage{amsthm}
\usepackage{tikz-cd}
\usepackage{quiver}

\newtheorem{theorem}{Theorem}
\newtheorem{definition}{Definition}
\newtheorem{lemma}{Lemma}

\newcommand{\PP}{\mathbb{P}}
\newcommand{\C}{\mathbb{C}}
\newcommand{\Hom}{\text{Hom}}
\newcommand{\Ann}{\text{Ann}}
\newcommand{\OO}{\mathcal{O}}
\newcommand{\LL}{\mathcal{L}}
\newcommand{\MM}{\mathcal{M}}
\newcommand{\End}{\text{End }}
\newcommand{\coker}{\text{coker}~}
\newcommand{\dbar}{\overline{\partial}}
\newcommand{\cA}{\mathcal{A}}
\newcommand{\cG}{\mathcal{G}}
\newcommand{\Tr}{\text{Tr }}
\newcommand{\HH}{\mathbb{H}}
\newcommand{\XX}{\mathfrak{X}}
\newcommand{\sslash}{\mathbin{/\mkern-4mu/}}
\newcommand{\cP}{\mathcal{P}}
\begin{document}	
	\subsection{Toric Degeneration}
	Here we want to describe a result of Harada and Kaveh (cite), which constructs an integrable system from a \textit{toric degeneration} satisfying some additional hypotheses. 
	\begin{definition}
		\label{d:toricdegen}
		Let $X$ be an $n$-dimensional projective variety. We call $\pi: \XX\to\C$ a \emph{toric degeneration} of $X$ if:
		\begin{enumerate}
			\item $\pi:\XX\to\C$ is a flat family of irreducible varieties. In particular, each fibre $X_t := \pi^{-1}(t)$ is a reduced scheme.
			\item The family $\XX$ is trivial over $\C^\ast$, namely there exists a fibre-preserving isomorphism $\rho:X\times\C^\ast \to \XX - X_0$, such that for each $t\in \C^\ast$ $\rho_t:X\times\{t\} \to X_t$ is an isomorphism.
			\item The fibre $X_0$ is a toric variety with respect to an action of $(\C^\ast)^n := \mathbb{T}$.
		\end{enumerate}
	\end{definition}
	In this case, the moduli space $\MM$ of flat $SU(2)$ connections serve as our $n$-dimensional projective variety, and we want to degenerate to the moduli space of parabolic bundles which will serve as our toric variety $X_0$. Now let us describe the additional hypothesis that are required for Harada and Kaveh's result.
	
	Suppose $\pi:\XX\to\C$ is a toric degeneration of $X$, and $X$ has a Kahler form $\omega$. Let $\Omega$ denote a constant multiple of a Fubini-Study Kahler form on $\mathbb{P}^N$, and equip $\mathbb{P}^N\times\C$ with the Kahler structure $\Omega \times \omega_{std}$. Assume that:
	\begin{enumerate}
		\item The family $\XX$ is smooth away from the zero fibre $X_0$.
		\item The family $\XX$ is embedded in $\mathbb{P}^N\times \C$ as an algebraic subvariety, for some projective space $\mathbb{P}^N$ such that:
			\begin{itemize}
				\item the map $\pi:\XX\to\C$ is the restriction of $\XX$ to the projection of $\mathbb{P}^N\times \C$ to $\C$;
				\item the action of $\mathbb{T}$ on $X_0$ extends to a linear action on $\mathbb{P}^N$.
			\end{itemize}
		Let $\omega_t$ denote the restriction of $\Omega\times \omega_{std}$ to the fibre $X_t$ embedded in $\mathbb{P}^N \times {t}$. Then
		\item The map $\rho_1 : X\to X_1$ is an isomorphism of Kahler manifolds; $\rho_1^\ast(\omega_1) = \omega$;
		\item Let $T = (S^1)^n$ denote the compact subtorus of $\mathbb{T}$. The Kahler form $\Omega$ on $\mathbb{P}^N$ is $T$-invariant and in particular the restriction $\omega_0$ to the toric variety $X_0$ is a $T$-invariant Kahler form.
	\end{enumerate}

	\subsection{Embedding the Degeneration into $\mathbb{P}^n\times \C$}
	In previous section we saw that we can embed $\MM$ into a projective space by considering a twist by a fixed line bundle and taking the top exterior power. We will go over this embedding again in a way that makes the symplectic structure clear. Fix a line bundle $L$ over $\Sigma$ of sufficiently high degree that $E\otimes L$ is globally generated. Then, we can define a line bundle $\LL$ over $\MM$ as follows. Given $(E,\dbar_E)\in \MM$ we have a line $\wedge^2 (E\otimes L)$. (remains to show this is really a bundle)
	
	\begin{theorem}
		Claim: $\Gamma(\LL, \MM) = V_1 := \Hom(H^0(\wedge^2 \OO^N), H^0(L^2))$.
	\end{theorem}
	\begin{proof}
		Let $\tilde{\sigma} \in \Gamma(\LL, \MM)$. Then $\tilde{\sigma}$ induces a map $\sigma:\wedge^2 E \to L^2$ by $\sigma(e_1\wedge e_2) = \tilde{\sigma}(e_1)\tilde{\sigma}(e_2)$. Since $E\otimes L$ is globally generated, there is a projection $\pi:\OO^N \to E$, inducing a map $\pi:\wedge^2 \OO^N \to \wedge^2 E$. Then let
		\begin{equation}
			\beta := \sigma\circ \pi: \wedge^2 \OO^N \to L^2
		\end{equation}
		this gives an element $\beta \in V_1$. Suppose $\sigma \circ \pi = \tau \circ \pi$ for $\sigma,\tau \in \Gamma(L^2, \MM)$. Then $(\sigma - \tau) \circ \pi = 0$, meaning 
		\begin{equation}
		\pi(\wedge^2 \OO^N) = \wedge^2 E \subset \ker(\sigma-\tau)
		\end{equation}
		Since the domain of $\sigma-\tau$ is all of $\wedge^2 E$, this tells us that $\ker(\sigma-\tau) = \wedge^2 E$ and hence $\sigma - \tau$ is the zero section; $\sigma = \tau$. Thus we have $\Gamma(\LL, \MM) \hookrightarrow V_1$.
		
		On the other hand, given any section $\beta \in V_1$, we can define a map $\wedge^2 E \to L^2$ by simply picking any embedding $\pi^{-1}:E\hookrightarrow \OO^N$ and letting $\sigma = \beta\circ \pi^{-1}:\wedge^2 E \to L^2$. Then precomposing this with the map $E \to \wedge^2 E$ gives a section in $\Gamma(\LL, M)$, which will map to $\beta$.  
		**some sketchy details about sections here**
	\end{proof}
	A non-zero global section $\sigma$ of $\LL$, which must exist since $\LL$ is high-degree, gives us local isomorphisms of $\LL$ with $\C$ and hence a natural hermitian metric $h$ on $\LL$. The Chern connection of this metric will have curvature (cite)
	\begin{equation}
		\Omega = \dbar\partial \log h.
	\end{equation}
	Since $\Gamma(\LL,\MM) = V_1$, this metric can be thought of as an inner product on $V_1$. In which case, $\Omega =\dbar\partial \log h$ is interpreted as the Fubini-Study metric on $\PP(V_1)$. unfounded claim: hence $\LL$ is a very ample line bundle, which shows again that we have embedding $\MM \hookrightarrow \PP(V_1)$.
	
	However, we want to embed the entire degeneration into the same projective space $\PP^N$. We have an embedding of $\MM$ into $\PP(V_1)$ and an embedding of $\cP$ into $\bigotimes_{i=1}^n \PP(W_1 \otimes W_2)$. We can further embed $\bigotimes_{i=1}^n \PP(W_1 \otimes W_2)$ into a very large $\PP^N$ using the Segre embedding, which has the important property that $\Omega_{fs} \PP^N = \sum \Omega_{fs} \PP(W_1\otimes W_2$. If we have $(E,\dbar_E)\in\MM$, we can keep track of what will become the parabolic data under the degeneration, and use that to fit $E$ into $\PP^N$.
	
	Let $(E,\dbar_E)\in \MM$ correspond to the flat $SU(2)$ connection $A$. Fix a trinion decomposition with boundary curves $C_i$ and suppose $A$ is adapted to trinion decomposition. Then at each curve, we have holonomy $Hol_{C_i}(A) \in SU(2)$. The holonomy has eigenvalues $\gamma_{i,1}$ and $\gamma_{i,2}$ and since the determinant must be 1, $\gamma_{i,1}\gamma_{i,2} = 1$. When $|\gamma_{i,1}|\neq 1$ (hence $\gamma_{i,1} \neq \gamma_{i,2}$) the eigenspaces are distinct, and we can define $\alpha_i:E|_{V_i} \to \C$ to be the projection to subspace of the largest eigenvalue. We can interpret this as a parabolic structure $(E,\alpha)$ at a choice of marked points $p_i$ on the boundary curves $C_i$, with weights $\gamma_i$ given by the largest eigenvalues.
	
	When the eigenvalues are both 1 $(\gamma_i = 1)$ the picture is more complicated, and corresponds to the parabolic sheaves from section (label) that acquire torsion. 
\end{document}