\documentclass[]{article}

%opening
\title{no title}
\date{\today}
\usepackage{amssymb}
\usepackage{amsmath}
\usepackage{tikz-cd}
\usepackage{quiver}

\newtheorem{theorem}{Theorem}
\newtheorem{definition}{Definition}
\newtheorem{lemma}{Lemma}

\newcommand{\C}{\mathbb{C}}
\newcommand{\Hom}{\text{Hom}}
\newcommand{\Ann}{\text{Ann}}
\newcommand{\OO}{\mathcal{O}}
\newcommand{\LL}{\mathcal{L}}
\newcommand{\MM}{\mathcal{M}}
\newcommand{\End}{\text{End }}
\newcommand{\coker}{\text{coker}~}
\newcommand{\dbar}{\overline{\partial}}
\newcommand{\cA}{\mathcal{A}}
\newcommand{\cG}{\mathcal{G}}
\newcommand{\Tr}{\text{Tr }}
\newcommand{\HH}{\mathbb{H}}
\newcommand{\XX}{\mathfrak{X}}
\newcommand{\sslash}{\mathbin{/\mkern-4mu/}}
\newcommand{\cP}{\mathcal{P}}
\begin{document}	
	\subsection{Toric Degeneration}
	Here we want to describe a result of Harada and Kaveh (cite), which constructs an integrable system from a \textit{toric degeneration} satisfying some additional hypotheses. 
	\begin{definition}
		\label{d:toricdegen}
		Let $X$ be an $n$-dimensional projective variety. We call $\pi: \XX\to\C$ a \emph{toric degeneration} of $X$ if:
		\begin{enumerate}
			\item $\pi:\XX\to\C$ is a flat family of irreducible varieties. In particular, each fibre $X_t := \pi^{-1}(t)$ is a reduced scheme.
			\item The family $\XX$ is trivial over $\C^\ast$, namely there exists a fibre-preserving isomorphism $\rho:X\times\C^\ast \to \XX - X_0$, such that for each $t\in \C^\ast$ $\rho_t:X\times\{t\} \to X_t$ is an isomorphism.
			\item The fibre $X_0$ is a toric variety with respect to an action of $(\C^\ast)^n := \mathbb{T}$.
		\end{enumerate}
	\end{definition}
	In this case, the moduli space $\MM$ of flat $SU(2)$ connections serve as our $n$-dimensional projective variety, and we want to degenerate to the moduli space of parabolic bundles which will serve as our toric variety $X_0$. Now let us describe the additional hypothesis that are required for Harada and Kaveh's result.
	
	Suppose $\pi:\XX\to\C$ is a toric degeneration of $X$, and $X$ has a Kahler form $\omega$. Let $\Omega$ denote a constant multiple of a Fubini-Study Kahler form on $\mathbb{P}^N$, and equip $\mathbb{P}^N\times\C$ with the Kahler structure $\Omega \times \omega_{std}$. Assume that:
	\begin{enumerate}
		\item The family $\XX$ is smooth away from the zero fibre $X_0$.
		\item The family $\XX$ is embedded in $\mathbb{P}^N\times \C$ as an algebraic subvariety, for some projective space $\mathbb{P}^N$ such that:
			\begin{itemize}
				\item the map $\pi:\XX\to\C$ is the restriction of $\XX$ to the projection of $\mathbb{P}^N\times \C$ to $\C$;
				\item the action of $\mathbb{T}$ on $X_0$ extends to a linear action on $\mathbb{P}^N$.
			\end{itemize}
		Let $\omega_t$ denote the restriction of $\Omega\times \omega_{std}$ to the fibre $X_t$ embedded in $\mathbb{P}^N \times {t}$. Then
		\item The map $\rho_1 : X\to X_1$ is an isomorphism of Kahler manifolds; $\rho_1^\ast(\omega_1) = \omega$;
		\item Let $T = (S^1)^n$ denote the compact subtorus of $\mathbb{T}$. The Kahler form $\Omega$ on $\mathbb{P}^N$ is $T$-invariant and in particular the restriction $\omega_0$ to the toric variety $X_0$ is a $T$-invariant Kahler form.
	\end{enumerate}

	\subsection{Embedding the Degeneration into $\mathbb{P}^n\times \C$}
	
\end{document}