

	Moduli spaces of principal $G$-connections on Riemann surfaces have been a topic of much mathematical research, both due to their interesting complex geometry and their connections with gauge theories in physics. Of particular note are unitary connections $(G=U(n) \text{ or }SU(n))$, as unitary groups arise as the structure groups of the gauge theories of bosons in the standard model, and have particularly tractable moduli spaces. A pioneering work in this area is that of Atiyah and Bott \cite{atiyah_yang-mills_1983}, which we will reference frequently. 
	
	Understanding the quantization of these gauge theories is a ongoing area of research. One notable work in this area is a paper of Jeffrey and Weitsman \cite{jeffrey_bohr-sommerfeld_1992} which discusses the geometric quantization of the space $\MM$ of flat $SU(2)$ connections on a compact Riemann surface. In their paper, they describe a real polarization of $\MM$ by decomposing the surface into \textit{trinions}, or \textit{pairs of pants}. At the boundary of two trinions in the decomposition, one has a closed curve in the surface, and a real polarization of the moduli space can be given by the holonomy of connections around these curves. These holonomies also give rise to Goldman flows, which almost give a toric Hamiltonian action on the space, but there are problems for connections $A$ which have a holonomy that is central in $SU(2)$ around one of the decomposition curves. Their paper counts the number of Bohr-Sommerfeld points in $\MM$, showing that it is given by the Verlinde dimension. If their real polarizaton had been a smooth fibration, then a theorem of Sniyaticki could be applied which says that the dimension of the quantization is given by the number of Bohr-Sommerfeld points in $\MM$. This would match the known result for a Kahler polarization of $\MM$, that the dimension of the quantization is given by the Verlinde dimension (cite). Unfortunately, due to the connections with central holonomies, this is not the case.
	
	In order to complete the proof, one can try and build a smooth moduli space which is a toric variety, with a prequantum line bundle whose sections are the same as those which we wish to compute. Hurtubise and Jeffrey \cite{hurtubise_moduli_2005}\cite{hurtubise_representations_2000} construct a moduli space $P$ using symplectic implosion, which is a toric variety, with a Hamiltonian system having the same moment polytope as that of the Hamiltonain system on $\MM$. Furthermore, they also give a holomorphic description of the moduli space. Mehta and Seshadri (cite) tell us that unitary connections on a punctured Riemann surface with fixed holonomy at the fibres are in correspondence with the \textit{parabolic vector bundles} on the unpunctured space. Considering a trinion as a thrice-punctured Riemann surface, we can study the moduli of unitary connections on a trinion in terms of parabolic vector bundles. Since we want to study unitary connections with any holonomies, we have to find a space $\cP$ which includes all the parabolic structures with any holonomies, and this space will allow us to include the singular fibres of the real polarisation, at the cost of considering instead \emph{framed parabolic sheaves}. Finally, Hurtubise and Jeffrey exhibit an isomorphism between $P$ and $\cP$.
	
	Therefore, we have the moduli space $(\MM,\omega)$ with prequantum line bundle $(\LL,\nabla)$ for which we wish to compute the dimension of the polarization, and the parabolic moduli space $P$, over which the dimension of the sections of a corresponding line bundle can be computed using the theory of toric varieties to match the Verlinde formula. Furthermore, $P \cong \cP$, and therefore the dimension of sections of line bundles over $\cP$ is also given by the Verlinde formula. Therefore, if we can put a prequantum line bundle $(\LL_P, \nabla_P)$ on $P$, whose quantization has the same dimension as that of the prequantum system on $\MM$, we will be able to complete the proof of the Verlinde formula for the quantization in the real polarization.
	
	The relationship between the spaces $\MM$ and $P$ is given in terms of a \textit{degeneration} of the smooth Riemann surface to the punctured one, and the induced degeneration of the moduli spaces. Biswas and Hurtubise \cite{biswas_degenerations_2021} provide a model for the degeneration of the Riemann surfaces and a corresponding degeneration of the moduli space of vector bundles. The degeneration of surfaces is a family over a neighbourhood of $0$ in $\C$ of Riemann surfaces, which are smooth for $t\neq0$ and which approach the punctured surface at $t=0$. For the corresponding degeneration of moduli spaces, at $t=0$, one obtains $\cP$ and at $t\neq 0$ we have the moduli space $\MM$ of holomorphic vector bundles which we hope to quantize. 
	
	
	Thus, the aim of this thesis is to construct the corresponding prequantum system on $P$, and show that its quantization has the same dimension as that of $\MM$, thus obtaining a new proof of the Verlinde formula for the moduli space of $SU(2)$ connections on a Riemann surface. This document proceeds by introducing the moduli space $\MM$ of unitary connections on a Riemann surface and its geometric quantization (Section \ref{s:vectorbundles}), then describing the space of parabolic sheaves $\cP$ following Hurtubise and Jeffrey (Section 3). Afterwards, we describe the degeneration of Biswas and Hurtubise, and how to embed it into a projective space (Section 4), and finally build the prequantum line bundle over $P$ and show its sections match those of the line bundle over $\MM$. (Section 5). Finally, we conclude with a summary of the results and potential avenues for continued research (Section 6).

