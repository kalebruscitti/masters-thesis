\section{Introduction}
\label{s:intro}

	Moduli spaces of principal $G$-connections on Riemann surfaces have been a topic of much mathematical research, both due to their interesting complex geometry and their connections with gauge theories in physics. Of particular note are unitary $(G=U(n) \text{ or }SU(n))$ connections, which arise as the structure groups of the gauge theories of bosons in the standard model, and have particularly tractable moduli spaces, such as studied by Atiyah and Bott \cite{atiyah_yang-mills_1983}. 
	
	One notable work in this area is a paper of Jeffrey and Weitsman \cite{jeffrey_bohr-sommerfeld_1992} which discusses the geometric quantization of the space $\MM$ of flat $SU(2)$ connections on a compact Riemann surface. In their paper, they describe $\MM$ by decomposing the surface into \textit{trinions}, or \textit{pairs of pants}. At the boundary of two trinions in the decomposition, one has a closed curve in the surface, and the moduli space $\MM$ can be described in terms of the holonomy of connections around these curves, with proper gluing conditions. They use the holomony to construct functions on $\MM$, which almost give a toric Hamiltonian action on the space, but there are problems for connections $A$ which have a holonomy that is central in $SU(2)$. Their paper proves that the dimension of the geometric quantization of $\MM$ is given by the Verlinde dimension, which for toric varieties can be computed by a point count in the simplex associated to the toric variety. If their action had been toric, then it would give a proof of the Verlinde formula for $\MM$, but it remains to study the points with central holonomy.
	
	To build these points into the moduli space and obtain a toric variety, Hurtubise and Jeffrey \cite{hurtubise_moduli_2005}\cite{hurtubise_representations_2000} construct a moduli space $P$ using symplectic implosion, which is toric and has the same moment polytope as our candidate for $\MM$. Furthermore, they also give a holomorphic description of the moduli space. Mehta and Seshadri (cite) tell us that unitary connections on a punctured Riemann surface with fixed holonomy at the fibres are in correspondence with the \textit{parabolic vector bundles} on the unpunctured space. Considering a trinion as a thrice-punctured Riemann surface, we can study the moduli of unitary connections on a trinion in terms of parabolic vector bundles. Since we want to study unitary connections with any holonomies, we have to find a space $\cP$ which includes all the parabolic structures with any holonomies, and this space will allow us to include the non-smooth fibres, at the cost of considering instead \emph{parabolic sheaves}. Finally, they exhibit an isomorphism between $P$ and $\cP$.
	
	Therefore, we have $\MM$ for which we wish to compute the dimension of the space of sections of line bundles, which fits into a bigger space $P$, over which the dimension of the sections of the corresponding line bundle is given by the theory of toric varities by the Verlinde formula. Furthermore, $P \cong \cP$, and therefore the dimension of sections of line bundles over $\cP$ is also given by the Verlinde formula. What remains to be proven is that the dimensions over $\MM$ and over $\cP$ are the same. The relationship between these two spaces is given in terms of a \textit{degeneration} of the smooth Riemann surface to the punctured one, and the induced degeneration of the moduli spaces. If this degeneration preserves the space of sections of line bundles over the moduli space, then one may obtain a new proof of the Verlinde formula. 
	
	Towards this goal, Biswas and Hurtubise \cite{biswas_degenerations_2021} describe a degeneration of the Riemann surfaces and a corresponding degeneration of the moduli space of vector bundles. The degeneration of surfaces is a family over $\mathbb{C}$ of Riemann surfaces, which are smooth for $t\neq0$ and which approach the punctured surface at $t=0$. For the corresponding degeneration of moduli spaces, at $t=0$, one obtains $\cP$ and at $t\neq 0$ we have the moduli space $\MM$ of holomorphic vector bundles which we hope to quantize. To show that this degeneration preserves the symplectic structure and the quantum data, we turn to a theorem of Harada and Kaveh \cite{harada_integrable_2015}. Their theorem shows that if the degeneration is \textit{toric} and there is an embedding of the degeneration into projective space, such that the pullback of a Fubini-Study metric gives us the symplectic structure on our space, then the Hamiltonian system on the surface at $t=0$ gives us a Hamiltonian system for $t\neq 0$. 
	
	Therefore, the aim of this thesis is to show that the degeneration of Biswas and Hurtubise satisfies the conditions of the Harada-Kaveh theorem in the $G=SU(2)$ case, and thus obtain a new proof of the Verlinde formula for the moduli space of $SU(2)$ connections on a Riemann surface. This document proceeds by introducing the moduli space $\MM$ of unitary connections on a Riemann surface and its geometric quantization (Section \ref{s:vectorbundles}), then describing the space of parabolic sheaves $\cP$ following Hurtubise and Jeffrey (Section 3). Afterwards, we describe the degeneration of Biswas and Hurtubise, and how to embed it into a projective space (Section 4), and (god willing) we verify that it satisfies the conditions of the Harada-Kaveh theorem (Section 5). Finally, we conclude with a summary of the results and potential avenues for continued research (Section 6).

