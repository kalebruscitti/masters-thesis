\section{Introduction}
\label{s:intro}

	Moduli spaces of principal $G$-connections on Riemann surfaces have been a topic of much mathematical research, both due to their interesting complex geometry and their connections with gauge theories in physics. Of particular note are unitary $(G=U(n) \text{ or }SU(N))$ connections, which arise as the structure groups of the gauge theories of bosons in the standard model, and have particularly tractable moduli spaces, such as studied by Atiyah and Bott (cite). 
	
	One notable work in this area is a paper of Jeffrey and Weitsman (cite) which discusses the geometric quantization of the space $\MM$ of flat $SU(2)$ connections on a compact oriented Riemann surface. In their paper, they describe a torus action on $\MM$ exhibiting it as a toric variety. They also show how to describe $\MM$ by decomposing the surface into \textit{trinions}, or \textit{pairs of pants}. At the boundary of two trinions in the decomposition, one has a closed loop in the surface, and the moduli space $\MM$ can be described in terms of the holonomy of connections around these curves, with proper gluing conditions. Their paper strongly suggests, but is not a complete proof, that the dimension of the quantum system associated to their quantization is given by the Verlinde formula, which is computed in terms of a point count of the simplex associated to $\MM$ as a toric variety. The problem is that there are some fibres of the polarization used to quantize $\MM$ which are not smooth, those which correspond to trivial holonomies, which need to be accounted for.
	
	To better understand this moduli space and its non-smooth fibres, one can try to transform the data. Due to Mehta and Seshadri, the unitary connections on a punctured Riemann surface with fixed holonomy at the fibres are in correspondence with the \textit{parabolic vector bundles} on the unpunctured space. Considering a trinion as a thrice-punctured Riemann surface, we can study the moduli of unitary connections on a trinion in terms of parabolic vector bundles. Since we want to study unitary connections with any holonomies, we have to find a space which includes all the parabolic structures with any holonomies, and this space will allow us to include the non-smooth fibres. This approach is used by Hurtubise and Jeffrey (cite) who describe this moduli space in both a complex and a symplectic picture. Their description fits the problem fibres of the polarization neatly into the space, and thus gives us a proper geometric quantization of the moduli space. However, it is not clear that in this transformation of the problem, we have preserved the sections of the line bundles over the moduli space, which is the quantum data of the geometric quantization.
	
	The relationship between these two pictures is given in terms of a \textit{degeneration} of the smooth Riemann surface to the punctured one, and the induced degeneration of the moduli spaces. If this degeneration preserves the space of sections of line bundles over the moduli space, then one may obtain a new proof of the Verlinde formula. Towards this goal, Biswas and Hurtubise describe a degeneration of the Riemann surfaces and a corresponding degeneration of the moduli space of vector bundles. The degeneration of surfaces is a family over $\mathbb{C}$ of Riemann surfaces, which are smooth for $t\neq0$ and which approach the punctured surface at $t=0$. For the corresponding degeneration of moduli spaces, at $t=0$, one obtains the toric variety $\MM$ of Jeffrey and Weitsman, and at $t\neq 0$ we have the moduli space of holomorphic vector bundles which we hope to quantize. To show that this degeneration preserves the symplectic structure and the quantum data, we turn to a theorem of Harada and Kaveh (cite). Their theorem shows that if the degeneration is \textit{toric} and there is an embedding of the degeneration into projective space, such that the pullback of a Fubini-Study metric gives us the symplectic structure on our space, then the Hamiltonian system on the surface at $t=0$ gives us a Hamiltonian system for $t\neq 0$. 
	
	Therefore, the aim of this thesis is to show that the degeneration of Biswas and Hurtubise, in the $G=SU(2)$ case, satisfies the conditions of the Harada-Kaveh theorem, and thus obtain a new proof of the Verlinde formula for the moduli space of $SU(2)$ connections on a Riemann surface. This document proceeds by introducing the moduli space of unitary connections on a Riemann surface (Section \ref{s:vectorbundles}), then describing the geometric quantization of this space, as constructed by Jeffrey, Weitsman and Hurtubise (Section 3). Afterwards, we describe the degeneration of Biswas and Hurtubise, and how to embed it into a projective space (Section 4), and (god willing) we verify that it satisfies the conditions of the Harada-Kaveh theorem (Section 5). Finally, we conclude with a summary of the results and potential avenues for continued research (Section 6).

