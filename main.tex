\documentclass[]{article}
\parskip = \baselineskip

%opening
\title{no title}
\date{\today}
\usepackage{amssymb}
\usepackage{amsmath}
\usepackage{amsthm}
\usepackage{tikz-cd}
\usepackage{quiver}
\usepackage{dsfont}
\usepackage{biblatex}
\addbibresource{citations.bib}

\newtheorem{theorem}{Theorem}
\newtheorem{definition}{Definition}
\newtheorem{lemma}{Lemma}


\newcommand{\C}{\mathbb{C}}
\newcommand{\Hom}{\text{Hom}}
\newcommand{\Hol}{\text{Hol}}
\newcommand{\Ann}{\text{Ann}}
\newcommand{\OO}{\mathcal{O}}
\newcommand{\LL}{\mathcal{L}}
\newcommand{\MM}{\mathcal{M}}
\newcommand{\End}{\text{End }}
\newcommand{\coker}{\text{coker}~}
\newcommand{\dbar}{\overline{\partial}}
\newcommand{\cA}{\mathcal{A}}
\newcommand{\cG}{\mathcal{G}}
\newcommand{\cP}{\mathcal{P}}
\newcommand{\Tr}{\text{Tr }}
\newcommand{\cR}{\mathcal{R}}
\newcommand{\PP}{\mathbb{P}}
\newcommand{\HH}{\mathbb{H}}
\newcommand{\ad}{\text{ad~}}
\newcommand{\sslash}{\mathbin{/\mkern-4mu/}}

\begin{document}


	Moduli spaces of principal $G$-connections on Riemann surfaces have been a topic of much mathematical research, both due to their interesting complex geometry and their connections with gauge theories in physics. Of particular note are unitary connections $(G=U(n) \text{ or }SU(n))$, as unitary groups arise as the structure groups of the gauge theories of bosons in the standard model, and have particularly tractable moduli spaces. A pioneering work in this area is that of Atiyah and Bott \cite{atiyah_yang-mills_1983}, which we will reference frequently. 
	
	Understanding the quantization of these gauge theories is a ongoing area of research. One notable work in this area is a paper of Jeffrey and Weitsman \cite{jeffrey_bohr-sommerfeld_1992} which discusses the geometric quantization of the space $\MM$ of flat $SU(2)$ connections on a compact Riemann surface. In their paper, they describe a real polarization of $\MM$ by decomposing the surface into \textit{trinions}, or \textit{pairs of pants}. At the boundary of two trinions in the decomposition, one has a closed curve in the surface, and a real polarization of the moduli space can be given by the holonomy of connections around these curves. These holonomies also give rise to Goldman flows, which almost give a toric Hamiltonian action on the space, but there are problems for connections $A$ which have a holonomy that is central in $SU(2)$ around one of the decomposition curves. Their paper counts the number of Bohr-Sommerfeld points in $\MM$, showing that it is given by the Verlinde dimension. If their real polarizaton had been a smooth fibration, then a theorem of Sniyaticki could be applied which says that the dimension of the quantization is given by the number of Bohr-Sommerfeld points in $\MM$. This would match the known result for a Kahler polarization of $\MM$, that the dimension of the quantization is given by the Verlinde dimension (cite). Unfortunately, due to the connections with central holonomies, this is not the case.
	
	In order to complete the proof, one can try and build a smooth moduli space which is a toric variety, with a prequantum line bundle whose sections are the same as those which we wish to compute. Hurtubise and Jeffrey \cite{hurtubise_moduli_2005}\cite{hurtubise_representations_2000} construct a moduli space $P$ using symplectic implosion, which is a toric variety, with a Hamiltonian system having the same moment polytope as that of the Hamiltonain system on $\MM$. Furthermore, they also give a holomorphic description of the moduli space. Mehta and Seshadri (cite) tell us that unitary connections on a punctured Riemann surface with fixed holonomy at the fibres are in correspondence with the \textit{parabolic vector bundles} on the unpunctured space. Considering a trinion as a thrice-punctured Riemann surface, we can study the moduli of unitary connections on a trinion in terms of parabolic vector bundles. Since we want to study unitary connections with any holonomies, we have to find a space $\cP$ which includes all the parabolic structures with any holonomies, and this space will allow us to include the singular fibres of the real polarisation, at the cost of considering instead \emph{framed parabolic sheaves}. Finally, Hurtubise and Jeffrey exhibit an isomorphism between $P$ and $\cP$.
	
	Therefore, we have the moduli space $(\MM,\omega)$ with prequantum line bundle $(\LL,\nabla)$ for which we wish to compute the dimension of the polarization, and the parabolic moduli space $P$, over which the dimension of the sections of a corresponding line bundle can be computed using the theory of toric varieties to match the Verlinde formula. Furthermore, $P \cong \cP$, and therefore the dimension of sections of line bundles over $\cP$ is also given by the Verlinde formula. Therefore, if we can put a prequantum line bundle $(\LL_P, \nabla_P)$ on $P$, whose quantization has the same dimension as that of the prequantum system on $\MM$, we will be able to complete the proof of the Verlinde formula for the quantization in the real polarization.
	
	The relationship between the spaces $\MM$ and $P$ is given in terms of a \textit{degeneration} of the smooth Riemann surface to the punctured one, and the induced degeneration of the moduli spaces. Biswas and Hurtubise \cite{biswas_degenerations_2021} provide a model for the degeneration of the Riemann surfaces and a corresponding degeneration of the moduli space of vector bundles. The degeneration of surfaces is a family over a neighbourhood of $0$ in $\C$ of Riemann surfaces, which are smooth for $t\neq0$ and which approach the punctured surface at $t=0$. For the corresponding degeneration of moduli spaces, at $t=0$, one obtains $\cP$ and at $t\neq 0$ we have the moduli space $\MM$ of holomorphic vector bundles which we hope to quantize. 
	
	
	Thus, the aim of this thesis is to construct the corresponding prequantum system on $P$, and show that its quantization has the same dimension as that of $\MM$, thus obtaining a new proof of the Verlinde formula for the moduli space of $SU(2)$ connections on a Riemann surface. This document proceeds by introducing the moduli space $\MM$ of unitary connections on a Riemann surface and its geometric quantization (Section \ref{s:vectorbundles}), then describing the space of parabolic sheaves $\cP$ following Hurtubise and Jeffrey (Section 3). Afterwards, we describe the degeneration of Biswas and Hurtubise, and how to embed it into a projective space (Section 4), and finally build the prequantum line bundle over $P$ and show its sections match those of the line bundle over $\MM$. (Section 5). Finally, we conclude with a summary of the results and potential avenues for continued research (Section 6).


\section{Moduli Space of Flat Unitary Connections on a Riemann Surface}
\label{s:vectorbundles}
	Given a Riemann surface $\Sigma$ and a unitary group $G=U(n)$ or $G=SU(n)$, we are interested in the moduli space $\MM$ of connections on a principal $G$ bundle over $\Sigma$, up to gauge equivalence. A detailed study of this moduli space was made by Atiyah and Bott \cite{atiyah_yang-mills_1983}, from which we take much of the following discussion.
	
	Thanks to the work of Narasimhan and Seshadri and Donaldson \cite{donaldson_new_1983}(cite NS) there are multiple ways in which one can view $\MM$. One equivalence is between flat unitary connections and irreducible representations of $\pi_1(\Sigma)$ into $G$. The gauge equivalence is then accounted for with a Geometric Invariant Theory quotient $\Hom(\pi_1(\Sigma), G)\sslash G$ \cite{thaddeus_geometric_1996}(section 7 for example). Another equivalence is with holomorphic $SL(n,\C)$ bundles, which we think of in terms of Doubeault operators $\dbar_E$ on a smooth complex vector bundle $E$. Different aspects of the geometry of $\MM$ become clear in different pictures. As connections, the symplectic structure is clear, whereas as holomorphic bundles the complex structure is clear.
	\subsection{Representation Variety Picture}
	First let us sketch the idea of the correspondence between flat connections and representations of the fundamental group. Given any $G$-connection $A$ on a manifold $\Sigma$, the holonomy of $A$ around a loop $\gamma$ based at $p\in \Sigma$ gives us a map $\text{Hol}_A(\gamma):\text{Loops}(p,\Sigma)\to G$. Generically, the holonomy is not invariant up to homotopy, so this map doesn't pass to a map on $\pi_1(\Sigma) \to G$. However, \emph{flat} connections are defined to be those for which the holonomy for any contractible loop is trivial. In this case, we can pass to the quotient to get a map $\text{Hol}_A(\gamma):\pi_1(p,\Sigma)\to G$. Picking a different basepoint $p'$ conjugates the resulting morphism in $G$, so that we can associate to any flat connection $A$ a map $\pi_1(\Sigma)\to G$ up to conjugation. It turns out that this map actually determines $A$ up to gauge equivalence.
	
	Let $\cA$ denote the space of flat connections on the trivial principal bundle $P=G\times \Sigma$, let $\cG = \C^\infty(P, G)^G$ be the gauge group, and let $\Phi:\cA \to \Hom(\pi_1(\Sigma),G)/G$ denote the map taking $A$ to $\text{Hol}_A$. 
	\begin{lemma}
		If $A,B\in \cA$ and $\Phi(A) = \Phi(B)$, then $A \cong B \mod \cG$.
	\end{lemma}
	\begin{proof}
		Suppose $A,B\in \cA$ are both flat connections, for which $\Phi(A) = \Phi(B)$. Explicitly, given any loop $\gamma\in \pi_1(\Sigma)$ based at say $p\in \Sigma$, we have 
		\begin{equation*}
			\Hol_A(\gamma) = \Hol_B(\gamma), 
		\end{equation*}
		for some $h\in G$. We want to construct a gauge equivalence $f\in \cG$ between $A$ and $B$. Pick any other point $q\in \Sigma$ and a curve $\sigma:p\to q$. Then for any $g\in G$ denote by $\Pi_\sigma^Ag$ the parallel transport of $g$ along $\sigma$. I claim that $f(q) = \Pi_\sigma^B\left(\Pi_\sigma^A\right)^{-1}$ defined this way for all $q\in \Sigma$ gives a gauge equivalence between $A$ and $B$. First we must show $f$ is well defined; if $\tau$ is another curve from $p\to q$ then:
		\begin{align*}
			f(q)\Pi_\tau^A &= \Pi_\sigma^B\left(\Pi_\sigma^A\right)^{-1}\Pi_\tau^A\\
			&= \Pi_\sigma^B\left(\Pi_\sigma^A\right)^{-1}\Pi_\sigma^A\Hol_A(\delta^{-1}\circ \tau)\\
			&= \Pi_\sigma^B \Hol_B(\sigma^{-1}\circ \tau)\\
			&= \Pi_\tau^B\\
			f(q) &= \Pi_\tau^B (\Pi_\tau^A)^{-1}
		\end{align*}
		Therefore $f$ is well defined, and moreover this calculation shows it takes $A$-horizontal vectors to $B$-horizontal vectors. It is easy to see $f$ is smooth, and since it maps horizontal vectors to horizontal vectors, it must be an isomorphism of connections between $A$ and $B$. 
	\end{proof}
	This lemma tells us connections are determined up to gauge equivalence by their holonomy. Note that the proof did not use flatness of $A$ or $B$, so it is true of all connections. 
	
	To complete the correspondance between $\MM = \cA/\cG$ and $\Hom(\pi_1(G),G)/G$, one needs to show that given any map $\phi\in\Hom(\pi_1(G),G)$ we can find a connection whose holonomy matches $\phi$.
	\begin{lemma}
			$\Phi:\cA \to \Hom(\Pi_1(\Sigma),G)$ is surjective.
	\end{lemma}
	\begin{proof}
		Let $\phi:\pi_1(\Sigma)\to G$ be a group homomorphism. We think of $\Sigma$'s universal cover $\tilde{\Sigma}$ as a $\pi_1(\Sigma)$ bundle $\pi:\pi_1(\Sigma)\times \Sigma \to \Sigma$. Then at a point $x\in \tilde{\Sigma}$, $\pi_1(\Sigma, \pi(x))$ acts on $\tilde{\Sigma}$ by monodromy, and $\pi_1(\Sigma)$ acts on $G$ by
		\begin{equation}
			\gamma \cdot g = \phi(\gamma) g.
		\end{equation}
		Then we can define a principal $G$-bundle over $\Sigma$ using an associated bundle type construction:
		\begin{equation}
			P = \tilde{\Sigma} \times G / \pi_1(\Sigma).
		\end{equation}
		The monodromy action is proper and free on the universal cover, and left multiplication in $G$ is proper and free, so this quotient is a well-defined smooth manifold. Finally, we can put a connection on $P$ with the correct holonomy. To do so, we define a connection on $\tilde{\Sigma}\times G$ by picking the horizontal bundle in $T(\tilde{\Sigma}\times G)$ to be all vectors of the form $(v, 0)$, i.e. with no $G$ component. Then $\pi_1(\Sigma)$ preserves this space and the image in the quotient is a horizontal bundle defining a connection $A$ on $P$. 
		
		Let $\gamma$ be a loop in $\Sigma$ starting at $x$. Then let $\gamma':[0,1]\to P$ be defined by
		\begin{equation}
			\gamma'(t) = (\psi(\gamma), \gamma(t))).
		\end{equation}
		If we can check this is horizontal, we will be done as then $\Hol_A(\gamma) = \gamma'(t) = \psi(\gamma)$. If we lift under the quotient of $\pi_1(\Sigma)$ we get $\tilde{\gamma} \in \tilde{\Sigma}\times G$ with
		\begin{equation}
			\tilde{\gamma}(t) = ((v(t),\gamma(x)), \psi(\gamma))
		\end{equation}
		and $d/dt(\tilde{\gamma}) = (v'(t), 0)$, meaning it is horizontal. Note finally that since $\psi$ is a group homomorphism, $\Hol_A(\gamma) = \Hol_A(e) = \mathds{1}$ for any contractible loop, so $A$ is flat.
	\end{proof}
	Combining the previous two lemmas we have:
	\begin{theorem}
		The map $\Phi:\MM \to \Hom(\pi_1(\Sigma), G)/G$ taking $A$ to $\Hol_A(-) \mod G$ is a bijection
	\end{theorem}
	Thus, at least set-wise, we can study the moduli space of flat connections by studying the \textit{representation variety} $\Hom(\pi_1(\Sigma),G)/G$. If $\pi_1(\Sigma)$ is finitely generated, then we can consider $\Hom(\pi_1(\Sigma),G)$ as a subset of $G^n$ by taking the generators to their images under any homomorphism. Then $\Hom(\pi_1(\Sigma),G)$ inherits a topology from the Lie group topology on $G$, and $\Hom(\pi_1(\Sigma),G)/G$ gets the quotient topology. We will see soon that in order to get a \emph{fine} moduli space, we will need to impose that $G$ is a unitary group and that our representations are irreducible.

	\subsection{Holomorphic Bundles Picture}
	When $\Sigma$ is a Riemann surface, we can use the complex structure to study $\MM$ in another way. Differential forms on a Riemann surface have a splitting, $\Omega^1(\Sigma) = \Omega^{1,0}(\Sigma)\oplus \Omega^{0,1}(\Sigma)$ which induces a splitting on the space $\cA$ of connections on complex vector bundles $E$ over $\Sigma$. As we will discuss, the $(0,1)$ part of a connection $A\in \Omega^1(\Sigma)\oplus \mathfrak{gl}(n,\C)$ defines a \emph{holomorphic structure} on $E$, and the space of such structures is exactly $\MM$ again.
	
	\begin{definition}
		A \textit{holomorphic structure} on a complex vector bundle $E$ is a choice of trivializations $\{U_\alpha, \phi_\alpha\}$ for $E$, such that the transition functions
		\begin{equation*}
		T_{\alpha,\beta} = \phi_\alpha \circ \phi^{-1}_\beta: E|_{U_\alpha \cap U_\beta} \to E|_{U_\alpha \cap U_\beta},
		\end{equation*}
		are biholomorphic. 
	\end{definition}
	An equivalent and convenient characterization is as follows. Given a holomorphic structure, in every chart $\{U_\alpha\}$, with local frame $\{e_1,...,e_n\}$ for $E$, we can define a local operator taking a section $s = s^i e_i$ to
	\begin{equation*}
	\dbar_E(s) = \dbar(s^i)\otimes e_i,
	\end{equation*}
	where $\dbar$ is the usual Cauchy-Riemann operator on $\mathbb{C}$. Now, let us check this operator is well defined globally on $E$. On the intersection $U_\alpha \cap U_\beta$, with local frames $\{e_i\}$ and $\{f_i\}$, we have $s = s^i e_i = \tilde{s}^i f_i$, with $s^i = {T_{\alpha\beta}}^i_j\tilde{s}^j.$ Since $T_{\alpha\beta}$ is biholomorphic, we have:
	\begin{align*}
	\dbar_E(s) = \dbar(s^i)\otimes e_i &= \dbar({T_{\alpha\beta}}^i_j \tilde{s}^j)\otimes f_i\\
	&= {T_{\alpha\beta}}^i_j \dbar(\tilde{s}^j)\otimes f_i.
	\end{align*}
	Hence $\dbar_E$ transforms with $T_{\alpha\beta}$ and it is globally well defined. We call $\dbar_E$ the \textit{Dolbeault Operator} corresponding to the holomorphic structure on $E$. Conversely, if we have a differential operator $\dbar_E:\Gamma(E) \to \Omega^{0,1}(\Sigma)\otimes \Gamma(E)$, we can define a holomorphic structure on $E$. (pf or citation needed)
	
	Therefore, in order to study the space of holomorphic structures on $E$, we can equivalently study the space of Dolbeault operators on $E$. In a smooth local trivialization of $E$, we can write 
	\begin{equation*}
	\dbar_E = \dbar + B,
	\end{equation*}
	where $\dbar$ is the usual Cauchy-Riemann operator and $B \in \Omega^{0,1}(M, \End(E))$. Since $\dim \Sigma = 1$ there is no constraints on $B$ (citation needed) and so our space is an affine complex space with translations $\Omega^{0,1}(M,\End(E))$. We have an action of $\text{Aut}(E)$ on this space by change of basis, whose orbits are (by definition of Aut) isomorphism classes of holomorphic bundles $E$ of rank $n$ and degree $k$, and it is the space of these orbits, $N(n,k)$, that we wish to describe. 
	
	However, in order to get a well-behaved moduli space, we must introduce a condition called stability. 
	\begin{definition}
		Let the \emph{slope} of a bundle $E$ be
		$$\mu := c_1(E)/\text{rank}(E),$$
		where $c_1(E)$ denotes the first Chern class. Then $E$ is said to be \textit{stable} if, for every proper subbundle $F$ of $E$, $\mu(F) < \mu(E)$. If the inequality is not strict, $E$ is \textit{semi-stable}.
	\end{definition}
	Restricting our moduli space to stable bundles is required to guarantee that once we quotient our space by $\text{Aut}(E)$, we end up with a space that is Hausdorff (Mumford,Fogarty GIT book). The proof is extensive, so instead we show only a motivating example of two isomorphism classes of holomorphic bundle which intersect at an unstable bundle. Let $\Sigma = \mathbb{P}^1$ and $E = \OO(1)\oplus \OO(-1)$. Then
	\begin{equation}
	c_1(E) = c_1(\OO(1))c_1(\OO(-1)) = -1,
	\end{equation} 
	therefore $\mu(E) = -1/2$. However $\OO(1)$ is a proper subbundle of $E$ and $\mu(\OO(1)) = 1$, so $E$ is unstable. Letting $U_0 = \mathbb{P}^1 - \{\infty\}$ and $U_1 = \mathbb{P}^1 - \{0\}$, $E$ has transition function $T$ on $U_0 \cap U_1$ given by
	\begin{equation*}
	T(z) = \begin{bmatrix}
	z & 0\\
	0 & \frac{1}{z}
	\end{bmatrix}, z \neq 0, z\neq \infty.
	\end{equation*}
	Now consider conjugating $E$ by the following family in $Aut(E)$
	\begin{align*}
	\begin{bmatrix}
	1 & b\\
	c & 1\\
	\end{bmatrix}
	\begin{bmatrix}
	z & 0\\
	0 & \frac{1}{z}\\
	\end{bmatrix}
	\begin{bmatrix}
	1 & -c\\
	-b & 1\\
	\end{bmatrix} = \begin{bmatrix}
	z - \frac{b^2}{z} & \frac{b}{z}-cz\\
	cz - \frac{b}{z} & \frac{1}{z}-c^2 z\\
	\end{bmatrix}.
	\end{align*}
	If we let $c=0$ and $b=t$ we get a family $F(t)$ of vector bundles with transition matrices
	\begin{equation*}
	T_{F}(t) = \begin{bmatrix}
	z-t^2/z & t/z\\
	-t/z & 1/z
	\end{bmatrix},
	\end{equation*}
	and if we let $c=t$ and $b=0$ we get a family $G(t)$ of vector bundles with transition matrices
	\begin{equation*}
	T_{G}(t) = \begin{bmatrix}
	z & -tz\\
	tz & \frac{1}{z}-t^2 z
	\end{bmatrix}.
	\end{equation*}
	As $t\to 0$, both $F(t)$ and $G(t)$ go to $E$. Therefore, $E$ is in the orbit of both $F(1)$ and $G(1)$ under the action of Aut$(E)$. This means any open set containing $[F(1)]$ must contain $E$, and any open set containing $[G(1)]$ must contain $E$, so there are no disjoint open sets in $\Omega^{0,1}(M,\End E)/\text{Aut}(E)$ separating these families, contradicting Hausdorffness.
	
	For this reason, we will restrict our attention to only the subset of $N(n,k)$ consisting of semi-stable bundles. In particular, we will denote the space of degree $0$ semi-stable bundles as $\MM$. 
	
	Since stability is an open condition, we are safe to consider deformations. At a bundle $(E,\dbar_E)$ with holomorphic structure given by transition functions $T_{\alpha,\beta}$, we can consider deforming the holomorphic structure to 
	\begin{equation}
	T_{\alpha,\beta}(\epsilon) = T_{\alpha,\beta} + \epsilon t_{\alpha,\beta},
	\end{equation}
	where $t_{\alpha,\beta} \in \Omega^{0,1}(M,\End(E))$ and $\epsilon^2=0$. For this to remain a well-defined holomorphic structure, we require that $T_{\alpha,\beta}(\epsilon)$ satisfies the cocycle condition for all $\epsilon$. That is, on $U_{\alpha}\cap U_{\beta}\cap U_{\gamma}$,
	\begin{align*}
	T_{\alpha,\beta}(\epsilon)T_{\beta,\gamma}(\epsilon) &= T_{\alpha,\gamma}(\epsilon)\\
	\left(T_{\alpha,\beta} + \epsilon t_{\alpha,\beta} \right)
	\left(T_{\beta,\gamma} + \epsilon t_{\beta,\gamma} \right) &=
	\left(T_{\alpha,\gamma} + \epsilon t_{\alpha,\gamma} \right)\\
	T_{\alpha,\beta}T_{\beta,\gamma} + \epsilon(t_{\alpha,\beta}T_{\beta,\gamma} + T_{\alpha,\beta} t_{\beta,\gamma}) + \epsilon^2 t_{\alpha,\beta}t_{\beta,\gamma} &= T_{\alpha,\gamma} + \epsilon t_{\alpha,\gamma}
	\end{align*}
	using that $\epsilon^2 = 0$ and $T_{\alpha,\beta}$ satisfy the cocycle condition, we have
	\begin{align*}
	T_{\alpha,\gamma} + \epsilon(t_{\alpha,\beta} T_{\beta,\gamma} + T_{\alpha,\beta} t_{\beta,\gamma}) &= T_{\alpha,\gamma} + \epsilon t_{\alpha,\gamma}\\
	t_{\alpha,\beta} T_{\beta,\gamma} + T_{\alpha,\beta} t_{\beta,\gamma} &= t_{\alpha,\gamma}.
	\end{align*}
	This condition tells us that $t_{\alpha,\beta}$ is a 1 cocycle in the sheaf $\End(E)$. When we quotient the action of of $\text{Aut}(E)$, we find that the tangent space to $N(n,k)$ is $H^1(\End(E))$. 
	
	When $E$ has a hermitian metric $h:E\otimes E \to \C$, the conjugate Hodge star $\bar{\star}:\Omega^{0,1}(\Sigma) \to \Omega^{1,0}(\Sigma)$ combined with $h$ allows us to define a hermitian inner product on $H^1(\End (E))$.  First $h$ defines a metric on $\End E$; if $A,B\in \End E$, let
	\begin{equation}
	g(A,B) = \Tr(A^\dagger B),
	\end{equation}
	where $\dagger$ is defined in terms of $h$, by $h(Ae,e) = h(e,A^\dagger e)$ for all $e\in E$. Then for any $\alpha = A \otimes a$, $A\in \End E$ and $a \in \Omega^{0,1}(\Sigma)$, we define:
	\begin{align}
	\bar{\ast}_E \alpha = g(A,-)\otimes \bar{\ast}a, 
	\end{align}
	and
	\begin{equation}
	\langle \alpha, \beta \rangle = \int\limits_\Sigma \alpha \wedge_g \bar{\ast}_E \beta =\int\limits_\Sigma g(A,B)~a\wedge \bar{\ast} b.
	\end{equation}
	This is just another form of Serre Duality (citation). In a local co-ordinate chart where $\alpha = Adz$ and $\beta = Bdz$, this takes the form
	\begin{equation}
	\langle \alpha, \beta \rangle = \int\limits_\Sigma \Tr(A^\dagger B)~dz\wedge d\bar{z}.
	\end{equation}
	The relationship between this space of connections and the space of holomorphic vector bundles is described by the Narasimhan-Seshadri theorem. In one direction, given a flat connection $A$ on $P$, inducing a connection on the associated bundle $E$, we can decompose $A = A^{0,1} + A^{1,0}$. This allows us to take $\dbar_E = A^{0,1}$ as a complex structure on $E$ corresponding to the flat connection $A$. The Narasimhan-Seshadri theorem guarantees that this structure will define a stable bundle, and also gives the converse direction; that flat stable structures $\dbar_E$ define flat unitary connections $A$. 

	\subsection{Symplectic Picture}
	 Let $\Sigma$ be the compact connected Riemann surface of genus $g$. Consider the trivial principal $G=SU(2)$ bundle $P$, over $\Sigma$ and let $\cA$ denote the space of smooth principal connections on $P$. In a fixed trivialization $P \cong G\times \Sigma$, a connection is determined by a form $A \in \Omega^1(\Sigma)\otimes \mathfrak{g}$. If we let $\cA_0$ denote the subspace of flat connections $(0 = F:=dA + A\wedge A)$, we can identify
	\begin{equation}
	\cA_0 = \{A \in \Omega^1(\Sigma)\otimes\mathfrak{g} (=\cA) ~|~ dA + A\wedge A = 0\}.
	\end{equation}
	Then the gauge group, $\cG = \Hom(\Sigma, G)$ acts on $\cA$ as follows: for $g\in \cG$,
	\begin{equation}
	g\circlearrowright A := g^{-1}Ag + g^{-1}dg.
	\end{equation}
	The vector space $\cA$ has a natural symplectic structure, which comes from the inner product (the Killing form) on the Lie Algebra $\mathfrak{g}$, $K:\mathfrak{g}\otimes\mathfrak{g}\to\mathbb{C}$. If $A = \alpha \otimes X$ and $B = \beta \otimes Y$ then we can define
	\begin{equation}
	\omega(A,B) = \int\limits_\Sigma K(X,Y)\alpha\wedge \beta.
	\end{equation}
	This symplectic stucture is preserved by the action of $\cG$; we have
	\begin{align*}
	(g^{-1} A g + g^{-1}dg) \wedge (g^{-1} B g + g^{-1}dg) &= g^{-1}A\wedge B g + g^{-1}(dg\wedge Bg) + g^{-1}A\wedge dg + g^{-1}dg \wedge g^{-1}dg \\
	&= g^{-1} A\wedge B g + g^{-1}(d(gB) + d(Ag)g^{-1} - dA - gdB)g
	\end{align*}
	
	Importantly, the curvature $F = dA + A\wedge A$ is actually a moment map for the action of $\cG$ (also to be shown here). This lets us acquire the moduli space of flat connections by symplectic reduction:
	\begin{equation}
	\MM = \cA\sslash\cG = F^{-1}(0)/ \cG= \cA_0/ \cG.
	\end{equation} 
	Symplectic reduction also gives us a symplectic structure on the quotient space $\MM$, such that under the pullback by the quotient map, $q:\cA_0 \to \MM$, we recover the symplectic form $\omega$ for $\cA$.


\section{Geometric Quantization when $G=SU(2)$}
\documentclass[]{article}

%opening
\title{no title}
\date{\today}
\usepackage{amssymb}
\usepackage{amsmath}
\usepackage{tikz-cd}
\usepackage{quiver}

\newcommand{\C}{\mathbb{C}}
\newcommand{\Hom}{\text{Hom}}
\newcommand{\Ann}{\text{Ann}}
\newcommand{\OO}{\mathcal{O}}
\newcommand{\LL}{\mathcal{L}}
\newcommand{\MM}{\mathcal{M}}
\newcommand{\End}{\text{End}}
\newcommand{\coker}{\text{coker}~}
\newcommand{\dbar}{\overline{\partial}}
\newcommand{\cA}{\mathcal{A}}
\newcommand{\cG}{\mathcal{G}}
\newcommand{\Tr}{\text{Tr }}
\newcommand{\HH}{\mathbb{H}}
\newcommand{\sslash}{\mathbin{/\mkern-4mu/}}


\begin{document}
\subsection{$SL(2,\C)$ bundles on punctured surface.}
	As before, let $\Sigma$ be a compact Riemann surface and $\MM$ the moduli space of flat connections on $\Sigma$, which here we consider in the symplectic picture. Following Jeffrey and Weitsman, we describe a Hamiltonian system on this moduli space, exhibiting it as a toric variety. Let $C$ be a closed oriented curve in $\Sigma$ and pick a basepoint $y\in C$. We can define a function $\tilde{f}_C:\cA \to \mathbb{R}$ by 
	\begin{equation}
		\tilde{f}_C(A) = \frac{1}{2}\text{hol}_C(A),
	\end{equation}
	where hol$_C(A)$ means the holonomy of $A$ around $C$ from $y$ to $y$. Since the holonomy is $\cG$ invariant, this passes to $f_C:\MM \to \mathbb{R}$. $\Sigma$ admits a decomposition into \textit{trinions} or \textit{pairs of pants}, which are copies of a disc with two holes:
	\begin{equation}
		D = \{z \in \mathbb{C}~|~ |z|\leq 2 \} - \{z~|~|z-1|<1/2\}\cup \{z~|~ |z+1| < 1/2\},
	\end{equation}
	with marked points on the boundary of $D$. Suppose we are given such a decomposition of $\Sigma$ into $2g-2$ trinions $D_\gamma$, $\gamma\in\{1,2,...,2g-2\}$, joined along their boundaries and with the marked points on the boundaries coinciding for any trinions with non-trivial intersection. Then the boundary circles of $D_\gamma$ give a collection $C_i$, $i\in\{1,2,...,3g-3\}$ of closed oriented curves in $\Sigma$ for which we get corresponding functions $f_i = f_{C_i}:\MM \to \mathbb{R}$ using the above definition.
\end{document}

%\printbibliography
\end{document}
