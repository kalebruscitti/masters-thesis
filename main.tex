\documentclass[]{article}
\parskip = \baselineskip

%opening
\title{no title}
\date{\today}
\usepackage{amssymb}
\usepackage{amsmath}
\usepackage{amsthm}
\usepackage{tikz-cd}
\usepackage{quiver}
\usepackage{dsfont}
\usepackage{biblatex}
\addbibresource{citations.bib}

\newtheorem{theorem}{Theorem}
\newtheorem{definition}{Definition}
\newtheorem{lemma}{Lemma}


\newcommand{\C}{\mathbb{C}}
\newcommand{\Hom}{\text{Hom}}
\newcommand{\Hol}{\text{Hol}}
\newcommand{\Ann}{\text{Ann}}
\newcommand{\OO}{\mathcal{O}}
\newcommand{\LL}{\mathcal{L}}
\newcommand{\MM}{\mathcal{M}}
\newcommand{\End}{\text{End }}
\newcommand{\coker}{\text{coker}~}
\newcommand{\dbar}{\overline{\partial}}
\newcommand{\cA}{\mathcal{A}}
\newcommand{\cG}{\mathcal{G}}
\newcommand{\cP}{\mathcal{P}}
\newcommand{\Tr}{\text{Tr }}
\newcommand{\cR}{\mathcal{R}}
\newcommand{\PP}{\mathbb{P}}
\newcommand{\HH}{\mathbb{H}}
\newcommand{\ad}{\text{ad~}}
\newcommand{\sslash}{\mathbin{/\mkern-4mu/}}

\begin{document}
\section{Introduction}
\label{s:intro}

	Moduli spaces of principal $G$-connections on Riemann surfaces have been a topic of much mathematical research, both due to their interesting complex geometry and their connections with gauge theories in physics. Of particular note are unitary $(G=U(n) \text{ or }SU(n))$ connections, which arise as the structure groups of the gauge theories of bosons in the standard model, and have particularly tractable moduli spaces, such as studied by Atiyah and Bott \cite{atiyah_yang-mills_1983}. 
	
	One notable work in this area is a paper of Jeffrey and Weitsman \cite{jeffrey_bohr-sommerfeld_1992} which discusses the geometric quantization of the space $\MM$ of flat $SU(2)$ connections on a compact Riemann surface. In their paper, they describe $\MM$ by decomposing the surface into \textit{trinions}, or \textit{pairs of pants}. At the boundary of two trinions in the decomposition, one has a closed curve in the surface, and the moduli space $\MM$ can be described in terms of the holonomy of connections around these curves, with proper gluing conditions. They use the holomony to construct functions on $\MM$, which almost give a toric Hamiltonian action on the space, but there are problems for connections $A$ which have a holonomy that is central in $SU(2)$. Their paper proves that the dimension of the geometric quantization of $\MM$ is given by the Verlinde dimension, which for toric varieties can be computed by a point count in the simplex associated to the toric variety. If their action had been toric, then it would give a proof of the Verlinde formula for $\MM$, but it remains to study the points with central holonomy.
	
	To build these points into the moduli space and obtain a toric variety, Hurtubise and Jeffrey \cite{hurtubise_moduli_2005}\cite{hurtubise_representations_2000} construct a moduli space $P$ using symplectic implosion, which is toric and has the same moment polytope as our candidate for $\MM$. Furthermore, they also give a holomorphic description of the moduli space. Mehta and Seshadri (cite) tell us that unitary connections on a punctured Riemann surface with fixed holonomy at the fibres are in correspondence with the \textit{parabolic vector bundles} on the unpunctured space. Considering a trinion as a thrice-punctured Riemann surface, we can study the moduli of unitary connections on a trinion in terms of parabolic vector bundles. Since we want to study unitary connections with any holonomies, we have to find a space $\cP$ which includes all the parabolic structures with any holonomies, and this space will allow us to include the non-smooth fibres, at the cost of considering instead \emph{parabolic sheaves}. Finally, they exhibit an isomorphism between $P$ and $\cP$.
	
	Therefore, we have $\MM$ for which we wish to compute the dimension of the space of sections of line bundles, which fits into a bigger space $P$, over which the dimension of the sections of the corresponding line bundle is given by the theory of toric varities by the Verlinde formula. Furthermore, $P \cong \cP$, and therefore the dimension of sections of line bundles over $\cP$ is also given by the Verlinde formula. What remains to be proven is that the dimensions over $\MM$ and over $\cP$ are the same. The relationship between these two spaces is given in terms of a \textit{degeneration} of the smooth Riemann surface to the punctured one, and the induced degeneration of the moduli spaces. If this degeneration preserves the space of sections of line bundles over the moduli space, then one may obtain a new proof of the Verlinde formula. 
	
	Towards this goal, Biswas and Hurtubise \cite{biswas_degenerations_2021} describe a degeneration of the Riemann surfaces and a corresponding degeneration of the moduli space of vector bundles. The degeneration of surfaces is a family over $\mathbb{C}$ of Riemann surfaces, which are smooth for $t\neq0$ and which approach the punctured surface at $t=0$. For the corresponding degeneration of moduli spaces, at $t=0$, one obtains $\cP$ and at $t\neq 0$ we have the moduli space $\MM$ of holomorphic vector bundles which we hope to quantize. To show that this degeneration preserves the symplectic structure and the quantum data, we turn to a theorem of Harada and Kaveh \cite{harada_integrable_2015}. Their theorem shows that if the degeneration is \textit{toric} and there is an embedding of the degeneration into projective space, such that the pullback of a Fubini-Study metric gives us the symplectic structure on our space, then the Hamiltonian system on the surface at $t=0$ gives us a Hamiltonian system for $t\neq 0$. 
	
	Therefore, the aim of this thesis is to show that the degeneration of Biswas and Hurtubise satisfies the conditions of the Harada-Kaveh theorem in the $G=SU(2)$ case, and thus obtain a new proof of the Verlinde formula for the moduli space of $SU(2)$ connections on a Riemann surface. This document proceeds by introducing the moduli space $\MM$ of unitary connections on a Riemann surface and its geometric quantization (Section \ref{s:vectorbundles}), then describing the space of parabolic sheaves $\cP$ following Hurtubise and Jeffrey (Section 3). Afterwards, we describe the degeneration of Biswas and Hurtubise, and how to embed it into a projective space (Section 4), and (god willing) we verify that it satisfies the conditions of the Harada-Kaveh theorem (Section 5). Finally, we conclude with a summary of the results and potential avenues for continued research (Section 6).



	Given a Riemann surface $\Sigma$ and a unitary group $G=U(n)$ or $G=SU(n)$, we are interested in the moduli space $\MM$ of connections on a principal $G$ bundle over $\Sigma$, up to gauge equivalence. A detailed study of these spaces was made by Atiyah and Bott \cite{atiyah_yang-mills_1983}, from which we take much of the following discussion.
	
	Thanks to the work of Narasimhan and Seshadri and Donaldson \cite{donaldson_new_1983}\cite{narasimhan_stable_1965} there are multiple ways in which one can view $\MM$. One equivalence is between flat unitary connections and irreducible representations of $\pi_1(\Sigma)$ into $G$.
	Gauge equivalence for the connections is accounted for by a quotient: $\Hom(\pi_1(\Sigma), G)/ G$. Another equivalence is with holomorphic $SL(n,\C)$ bundles over $\Sigma$, which we think of as Doubeault operators $\dbar_E$ on a smooth complex vector bundle $E$. Different aspects of the geometry of $\MM$ become clear in different pictures, so we will explain each of them here. 
	
	\section{Flat Connections as Representations of the Fundamental Group}
	\label{s:vectorbundles}
	We begin with the correspondence between flat connections and representations of the fundamental group. For a Lie group $G$, given any $G$-connection $A$ on a manifold $\Sigma$, the holonomy of $A$ around a loop $\gamma$ based at $p\in \Sigma$ gives us a map $\text{Hol}_A(\gamma):\text{Loops}(p,\Sigma)\to G$. Generically, the holonomy is not invariant up to homotopy, so this map does not pass to a map on $\pi_1(\Sigma) \to G$. However, if one restricts to \emph{flat} connections, which are those whose holonomy around any contractible loop is trivial, then one can pass to the quotient to get a map $\text{Hol}_A(\gamma):\pi_1(p,\Sigma)\to G$. Picking a different basepoint or trivialization conjugates the resulting morphism in $G$, so that we can associate to any flat connection $A$ a map $\pi_1(\Sigma)\to G$ up to conjugation. This \emph{holonomy representation} determines $A$ up to gauge equivalence.
	
	Let $\cA$ denote the space of flat connections on the trivial principal bundle $P=G\times \Sigma$, let $\cG = \C^\infty(P, G)^G$ be the gauge group, and let $\Phi:\cA \to \Hom(\pi_1(\Sigma),G)/G$ denote the map taking $A$ to $\text{Hol}_A$. 
	\begin{lemma}
		The map $\Phi$ is injective up to conjugation in $G$. That is, for any two connections $A,B\in \cA$, if $\Phi(A) = \Phi(B)$, then $A \cong B \mod \cG$.
	\end{lemma}
	\begin{proof}
		Suppose $A,B\in \cA$ are flat connections with $\Phi(A) = \Phi(B) \mod G$. Explicitly, given any loop $\gamma\in \pi_1(\Sigma)$ based at $p\in \Sigma$, there exists an $h\in G$ such that
		\begin{equation}
			h^{-1}\Hol_A(\gamma)h = \Hol_B(\gamma). 
		\end{equation}
	 	To prove the lemma, we construct a gauge equivalence $f\in \cG$ between $A$ and $B$. For any $q\in \Sigma$ pick a curve $\sigma:p\to q$. Then for any $g\in G$ denote by $\Pi_\sigma^Ag$ the parallel transport of $g$ along $\sigma$. Let $f(q)=\Pi_\sigma^B\left(\Pi_\sigma^A\right)^{-1}$ for all $q\in \Sigma$, which we will show gives the required gauge equivalence. First we must show $f$ is well defined; if $\tau$ is another curve from $p\to q$ then:
		\begin{align*}
			f(q)\Pi_\tau^A &= \Pi_\sigma^B\left(\Pi_\sigma^A\right)^{-1}\Pi_\tau^A\\
			&= \Pi_\sigma^B\left(\Pi_\sigma^A\right)^{-1}\Pi_\sigma^A\Hol_A(\delta^{-1}\circ \tau)\\
			&= \Pi_\sigma^B \Hol_B(\sigma^{-1}\circ \tau)\\
			&= \Pi_\tau^B\\
			f(q) &= \Pi_\tau^B (\Pi_\tau^A)^{-1}
		\end{align*}
		Therefore $f$ is well defined, and moreover this calculation shows it takes $A$-horizontal vectors to $B$-horizontal vectors. It is easy to see $f$ is smooth, and since it maps horizontal vectors to horizontal vectors, it must be an isomorphism of connections between $A$ and $B$. 
	\end{proof}
	This lemma tells us connections are determined up to gauge equivalence by their holonomy. Note that the proof did not use flatness of $A$ or $B$, so it is true for all connections. To complete the correspondance between $\MM = \cA/\cG$ and $\Hom(\pi_1(\Sigma),G)/G$, it remains to show that given any map $\phi\in\Hom(\pi_1(\Sigma),G)$ one can find a connection whose holonomy matches $\phi$.
	\begin{lemma}
			The map $\Phi:\cA \to \Hom(\Pi_1(\Sigma),G)$ is surjective.
	\end{lemma}
	\begin{proof}
		Let $\phi:\pi_1(\Sigma)\to G$ be a group homomorphism. The universal cover $\tilde{\Sigma}$ of $\Sigma$ is a $\pi_1(\Sigma)$ bundle $\pi:\pi_1(\Sigma)\times \Sigma \to \Sigma$. Then at a point $x\in \tilde{\Sigma}$, $\pi_1(\Sigma, \pi(x))$ acts on $\tilde{\Sigma}$ by monodromy, and $\pi_1(\Sigma)$ acts on $G$ by
		\begin{equation}
			\gamma \cdot g = \phi(\gamma) g.
		\end{equation}
		Thus, define a principal $G$-bundle over $\Sigma$ by quotienting out this action:
		\begin{equation}
			P = \tilde{\Sigma} \times G / \pi_1(\Sigma).
		\end{equation}
		The monodromy action is proper and free on the universal cover, and left multiplication in $G$ is proper and free, so this quotient is a well-defined smooth manifold. Finally, one can put a connection on $P$ with the correct holonomy. To do so, define a connection on $\tilde{\Sigma}\times G$ by picking the horizontal bundle in $T(\tilde{\Sigma}\times G)$ to be all vectors of the form $(v, 0)$; those with no $G$ component. Then $\pi_1(\Sigma)$ preserves this space and the image in the quotient is a horizontal bundle defining a connection $A$ on $P$. 
		
		Let $\gamma$ be a loop in $\Sigma$ starting at $x$. Then let $\gamma':[0,1]\to P$ be defined by
		\begin{equation}
			\gamma'(t) = (\psi(\gamma), \gamma(t))).
		\end{equation}
		If this is horizontal, then $\Hol_A(\gamma) = \gamma'(t) = \psi(\gamma)$ which completes the proof. If we lift under the quotient of $\pi_1(\Sigma)$ we get $\tilde{\gamma} \in \tilde{\Sigma}\times G$ with
		\begin{equation}
			\tilde{\gamma}(t) = ((v(t),\gamma(x)), \psi(\gamma))
		\end{equation}
		and $d/dt(\tilde{\gamma}) = (v'(t), 0)$, meaning $\gamma'$ is horizontal. Note finally that since $\psi$ is a group homomorphism, $\Hol_A(\gamma) = \Hol_A(e) = \mathds{1}$ for any contractible loop, so $A$ is flat.
	\end{proof}
	Combining the previous two lemmas we have:
	\begin{theorem}
		\label{t:phi-bijection}
		The map $\Phi:\MM \to \Hom(\pi_1(\Sigma), G)/G$ taking $A$ to $\Hol_A(-) \mod G$ is a bijection
	\end{theorem}
	Thus, one may identify the set of flat connections with the set $\Hom(\pi_1(\Sigma),G)/G$. To build a moduli space, we want to endow $\MM$ with a topology and some geometric structure. If $\pi_1(\Sigma)$ is finitely presented as
	\begin{equation}
		\pi_1(\Sigma) = \langle a_1,...,a_N ~|~ R_1,...R_N \rangle,
	\end{equation}
	then consider $\Hom(\pi_1(\Sigma),G)$ as a subset of $G^N$ by taking the generators to their images under any homomorphism. This lets $\Hom(\pi_1(\Sigma),G)$ inherit a topology from the Lie group topology on $G$, and $\Hom(\pi_1(\Sigma),G)/G$ can be given the quotient topology. 
	
	Geometrically, $\Hom(\pi_1(\Sigma),G)$ corresponds to $G[a_1,...,a_n]/\langle R_1,...,R_N\rangle$, so when $G$ is an algebraic group, $\Hom(\pi_1(\Sigma),G)$ is a variety. Then $\MM = \Hom(\pi_1(\Sigma),G)/ G$ is a quotient variety.
	
	
	
	\section{Unitary Representations on a Riemann Surface}
	\label{s:moduli-as-reps}
	Now we specialize to a compact connected Riemann surface $\Sigma$ of genus $g$, and $G=U(n)$. Then the fundamental group is 
	\begin{equation}
		\pi_1(\Sigma) = \{a_1, b_1,...,a_g, b_g ~|~ \Pi_{i=1}^g a_ib_ia_i^{-1}b_i^{-1} = e
		\}.
	\end{equation}
	Define $\xi:U(n)^{2g}\to U(n)$ by $\xi(A_1,B_1,...,A_i,B_i)=\Pi_{i=1}^g A_iB_iA_i^{-1}B_i^{-1}$. Then $\MM = \Hom(\pi_1(\Sigma),U(n))/U(n)$ is $\xi^{-1}(e)/U(n)\subset U^{2g}/U(n)$ and inherits a quotient topology from the topology of $U^{2g}$. In general, $\MM$ is not smooth, but the subset of $\MM$ consisting of irreducible representations will be a smooth manifold. 

	\begin{lemma}
		\label{l:irrep-lemma}
		Let $\rho(a_i) = A_i, \rho(b_i) = B_i$, for the generators $(a_1,b_1,...,a_g,b_g)$ of $\pi_1(\Sigma)$. Then a representation $\rho:\pi_1(\Sigma) \to U(n)$ is reducible if and only if all elements in the set $\{A_1,B_1,...,A_2,B_2\}$ pairwise commute.
	\end{lemma}
	\begin{proof}
		Suppose the $A_i,B_i$ all pairwise commute. Then by the spectral theorem for unitary matrices, they are all simultaneously diagonalizable. Thus they share at least one eigenspace $W$, which is invariant under all the $A_i$ and $B_i$, and so $\rho$ is reducible.
		
		On the other hand, suppose $\rho$ is reducible. Since unitary representations are semisimple, we can write the representation as $\bigoplus_{j=0}^k W_j$, with each $W_j$ an irreducible subspace which is invariant under $\rho$. Then each $W_j$ must be an eigenspace of each matrix $A_i$ and $B_i$, and thus the matrices have the same eigenspaces and are simultaneously diagonalizable. Since simultaneously diagonalizable matrices commute, this means the $A_i$ and $B_i$ pairwise commute. 
	\end{proof}
	Let $\cR$ denote the subset of $\MM$ consisting of reducible representations, and $\MM_s$ denote the subset of irreducible points. The condition that $[A,B] =0$ is a closed condition, so $\MM_s$ is open in $\MM$ and $\cR$ is closed.
	\begin{lemma}
		$\cR$ is compact.
	\end{lemma}
	\begin{proof}
		Let $p:\Hom(\pi_1(\Sigma),U(n))\to \MM$ denote the quotient by $U(n)$. Let $\tilde{\cR} = p^{-1}(\cR)$. Then $\tilde{\cR} \subset U(n)^{2g}$ is closed and thus since $U(n)$ is compact, $\tilde{\cR}$ is compact. Then $\cR = \tilde{\cR}/U(n)$ is also compact.
	\end{proof}
	Using this one can characterize the topology of $\cR$.
	\begin{theorem}
		\label{t:reducibletorus}
		The reducible part $\cR$ of the moduli space $\MM$ is homeomorphic to
		\begin{equation}
			T^{2g}/W(T),
		\end{equation}
		where $T\subset G$ is a maximal torus and $W(T)$ is its Weyl group, acting by the $2g$-diagonal action.
	\end{theorem}
	\begin{proof}
		Let $\{a_i,b_i\}_{i=1}^{g}$ generate $\pi_1(\Sigma)$. For $[\rho]\in \cR$, let $A_i = \rho(a_i)$ and $B_i = \rho(b_i)$. By Lemma $\ref{l:irrep-lemma}$, $\rho\in \cR$ implies the $A_i$ and $B_i$ pairwise commute, and are hence contained in some maximal torus $T$, thus $(A_1,B_1,...,A_g,B_g)\in T^{2g}$. To pass to the quotient $[\rho]$ under conjugation by $U(n)$, we need to quotient the Weyl group $W(T)$. Thus $[A_1,B_1,...,A_g,B_g] \in T^{2g}/W(T)$. Diagrammatically, we have:
		\[\begin{tikzcd}
		{\Hom(\pi_1(\Sigma),U(n))} & {T^{2g}} \\
		\MM & {T^{2g}/W(T)}
		\arrow["p", from=1-1, to=2-1]
		\arrow[from=1-1, to=1-2]
		\arrow["q"', from=1-2, to=2-2]
		\arrow[dashed, from=2-1, to=2-2]
		\end{tikzcd}\]
		Since the topology of $\Hom(\pi_1(\Sigma),U(n))$ and $T^{2g}$ are their subspace topologies in $U^{2g}$, the upper arrow is continuous. Its composition with the quotient $q$ gives a continuous map $\Hom(\pi_1(\Sigma), U(n))\to T^{2g}/W(T)$, and by the universal property of the quotient topology, this means the map $\MM\to T^{2g}/W(T)$ is continuous.
		
		Next we show the map is bijective. For surjectivity, given any $(t_1,...,t_{2g})$ define $\rho(a_i) = t_{2i-1}$ and $\rho(b_i) = t_{2i}$. The torus' commutativity $[t_i,t_j]=0$ guarantees $\rho(a_i)$ will be a well-defined reducible representation of $\pi_1(\Sigma)$. For injectivity, if $\rho$ and $\rho'$ map to $[A_1,...B_g]$ and $[A'_1,...,B'_g]$ which are equal in in $T^{2g}/W(T)$ then it means there is an element $t\in W(T)$ for which $A'_i = tA_it^{-1}$ and $B'_i = tB_i t^{-1}$. Therefore $\rho' = t\rho t$ and so $[\rho]=[\rho']$.
		
		Finally, since $W(T)$ is finite, $T^{2g}/W(T)$ is Hausdorff; since $\cR$ is compact, our mapping is a continuous bijection from a compact space to a Hausdorff space, hence a homeomorphism.
	\end{proof}
	When $G$ or $\pi_1(\Sigma)$ is Abelian, $\MM = \cR$ and therefore Theorem \ref{t:reducibletorus} determines the entire moduli space. When $G=U(1)$ which is Abelian, $T = U(1) = \C^{\ast}$ and $W(T) = {e}$ so:
	\begin{equation}
		\label{e:jacobian-torus}
		\MM = \cR \cong (\C^{\ast})^{2g}.
	\end{equation}
	In this case, $\MM$ is the Jacobian variety of $\Sigma$, and equation \ref{e:jacobian-torus} is the well-known result that the Jacobian of a compact connected Riemann surface is a torus. 
	
	When $\Sigma$ has genus 1, $\pi_1(\Sigma) = \mathbb{Z}^2$ which is Abelian. Then
	\begin{equation}
		\MM = \cR \cong \frac{T^{2}}{W(T)}.
	\end{equation}
	Now we would like to address the irreducible points. In general, $\MM_0$ will be a smooth manifold \cite[\S7]{atiyah_yang-mills_1983} but here we only prove it for $G=SU(2)$. 
	\begin{theorem}
		When $G=SU(2)$, $\MM_0$ is a smooth manifold of (real) dimension $6g-6$.
	\end{theorem}
	\begin{proof}
		This proof follows that of Michiels \cite[Thm 96]{michiels_moduli_nodate}. The strategy is to first show the map $\xi$ is submersive on $\MM_0$, so that $\xi^{-1}(e)$ is a smooth manifold, and then prove that $\MM_0$ is a quotient of $\xi^{-1}(e)$ under a free action of a compact group with dimension $\dim SU(2)$. This will give a dimension count of
		\begin{equation}
			\dim \MM_0 = \dim(\xi^{-1}(e)) - \dim(SU(2)) = (2g-1)\dim(SU(2)) - \dim(SU(2)) = 6g-6.
		\end{equation} 
		Proving $\xi:SU(2)^{2g} \to SU(2)$ is submersive requires showing the rank of $\xi$ is 3 at all irreducible points. Let $(A_1(t),...,B_g(t)) = (A_1 + ta_1,..., B_g + tb_g)$ for some $(A_1,...B_g) \in SU^{2g}$ and some $(a_1,...,b_g) \in \mathfrak{su}(2)^{2g}$. Then composing with $\xi$ gives the curve
		\begin{equation}
			t\to \gamma(t):= \prod_{i=1}^g A_i(t)B_i(t)A_i(t)^{-1}B_i(t)^{-1}.
		\end{equation}
		We compute the differential, first considering just one factor:
		\begin{align*}
			\frac{d}{dt}|_{t=0} A_i(t)B_i(t)A_i(t)^{-1}B_i(t)^{-1} &= \Ad_{B_iA_iB_i^{-1}}(a_i) + \Ad_{B_iA_i}(b_i) - \Ad_{B_iA_i}(a_i) - \Ad_{B_i}(b_i)\\
			&= \Ad_{B_iA_i}\left(
			(\Ad_{B_i^{-1}}-1)a_i + (1-\Ad_{A_i^{-1}})b_i
			\right)
		\end{align*}
		Then the derivative of the entire product is given by the product rule:
		\begin{equation}
			\frac{d}{dt}|_{t=0} \gamma(t) = \sum_{i=1}^g \left[
			\Ad_{\left(\prod_{j>i} A_jB_jA_j^{-1}B_j^{-1}\right)^{-1}B_iA_i} \left((\Ad_{B_i^{-1}}-1)a_i + (1-\Ad_{A_i^{-1}})b_i\right)
			\right].
		\end{equation}
		Fixing one value of $i\in {1,...,g}$, we can take $a_j = b_j =0$ for $i\neq j$, to obtain
		\begin{equation}
			d\xi(a_1,...,b_g) = \Ad_{\left(\prod_{j>i} A_jB_jA_j^{-1}B_j^{-1}\right)^{-1}B_iA_i} \left((\Ad_{B_i^{-1}}-1)a_i + (1-\Ad_{A_i^{-1}})b_i\right).
		\end{equation}
		If for any $i$ the map $\mathfrak{g}^2\to\mathfrak{g}$:
		\begin{equation}
			(a,b) \to (\Ad_{B_i^{-1}} - 1)a + (1-\Ad_{A_i^{-1}})b
		\end{equation}
		is surjective, then by varying $a_i$ and $b_i$ one obtains all of $\mathfrak{g}$, implying $\xi$ would be surjective. Therefore, if instead $\xi$ does not have full rank at an irreducible point $(A_1,...,B_g)$, then for all $i$ the above map $\mathfrak{g}^2\to\mathfrak{g}$ is not surjective.
		
		For $G=SU(2)$, the non-surjectivity of this map implies that $A_i$ and $B_i$ commute. If either is $\pm\mathds{1}$ then they commute. Otherwise, $(\Ad_{B_i}^{-1}-1)$ and $(1-\Ad_{A_i}^{-1})$ have images given by the two planes perpendicular to the rotation axes of $\Ad_{B_i}$ and $\Ad_{A_i}$. Since their sum is not surjective and $\dim\mathfrak{g}=3$, their sum is dimension $2$, meaning these planes coincide. Hence $\Ad_{A_i}$ and $\Ad_{B_i}$ share the same axis of rotation, implying $A_i$ and $B_i$ commute.
		
		Since we can repeat this argument for each $i$, we conclude that if $\xi$ is not full rank at $(A_1,...,B_g)$ then $[A_i,B_i]=0$ for all $i$ and hence we can simplify the differential to
		\begin{equation}
			d\xi(a_1,...,b_g) = \sum_{i=1}^g \left[
			\Ad_{B_iA_i}\left((\Ad_{B_i^{-1}}-1)a_i + (1-\Ad_{A_i^{-1}})b_i\right)
			\right].
		\end{equation}
		Since $[A_j,B_j]=0$, the $j$th term in this sum has image given by the plane perpendicular to $A_i$ (which is the same as that of $B_i$). Furthermore, because $d\xi$ is not full rank, we must have that for each $j$, the image is the same plane, as otherwise by the same dimensional count as above we'd have a contradiction. Thus, the $\{A_i,B_i\}_{i=1}^{g}$ all pairwise commute and so $(A_1,...,B_g)$ is reducible. By the contrapositive, $\xi:G^{2g}\to G$ is a submersion on the irreducible points. 
		
		The action of $SU(2)$ on $\xi^{-1}(e)$ by conjugation is not free since $-\mathds{1}$ acts trivially. Thus we define an action of $SU(2)/{\pm \mathds{1}}$ by conjugation, which does act freely. Suppose $[C]\in SU(2)/{\pm \mathds{1}}$ acts trivially on $(A_1,...,B_g)\in \xi^{-1}(e)$. Then $C$ commutes with all $A_i$ and $B_i$, and since the point is irreducible, there is some pair in $(A_1,...,B_g)$ that does not commute; call that pair $(X,Y)$. Then $C$ commutes with $X$ and $Y$, which do not commute with eachother, so $\Ad_X$ and $\Ad_Y$ have different rotation axes, and $\Ad_C$ cannot have both; $C$ must be $\pm 1$. Thus the action of $SU(2)/{\pm \mathds{1}}$ on $\xi^{-1}(e)$ is free.
		
		Finally, the quotient $SU(2)/{\pm \mathds{1}} \cong SO(3)$ is compact, and $\MM_0 = \xi^{-1}(e)/SO(3)$. Since $\xi$ is a submersion and $SO(3)$ is a compact group acting freely on it, the quotient $\MM_0$ is a smooth manifold, with dimension $6g-6$ as computed at the beginning of the proof.
	\end{proof}
	Now that we have some understanding of the moduli space $\MM$, we will pass to a holomorphic description of $\MM$ in terms of \emph{semi-stable} holomorphic vector bundles over $\Sigma$.
	\section{Semi-stable Holomorphic Bundles}
	\label{s:ss-bundles}
	When $\Sigma$ is a Riemann surface, one can use its complex structure to augment the study of $\MM$. Differential forms on a Riemann surface have a splitting, $\Omega^1(\Sigma) = \Omega^{1,0}(\Sigma)\oplus \Omega^{0,1}(\Sigma)$ which induces a splitting on the space $\cA$ of connections on complex vector bundles $E$ over $\Sigma$. As we will discuss, the $(0,1)$ part of a connection $A\in \Omega^1(\Sigma)\otimes \mathfrak{gl}(n,\C)$ defines a \emph{holomorphic structure} on $E$, and we can describe the moduli spaces of connections in terms of holomorphic structures.
	
	\begin{definition}
		A \textit{holomorphic structure} on a complex vector bundle $E$ is a choice of trivializations $\{U_\alpha, \phi_\alpha\}$ for $E$, such that the transition functions
		\begin{equation*}
		T_{\alpha,\beta} = \phi_\alpha \circ \phi^{-1}_\beta: E|_{U_\alpha \cap U_\beta} \to E|_{U_\alpha \cap U_\beta},
		\end{equation*}
		are biholomorphic. 
	\end{definition}
	An equivalent and convenient characterization is as follows. Given a holomorphic structure, in every chart $\{U_\alpha\}$, with local frame $\{e_1,...,e_n\}$ for $E$, one can define a local operator taking a section $s = s^i e_i$ to
	\begin{equation*}
	\dbar_E(s) = \dbar(s^i)\otimes e_i,
	\end{equation*}
	where $\dbar$ is the usual Cauchy-Riemann operator on $\mathbb{C}$. Let us check this operator is well defined globally on $E$. On the intersection $U_\alpha \cap U_\beta$, with local frames $\{e_i\}$ and $\{f_i\}$, we have $s = s^i e_i = \tilde{s}^i f_i$, with $s^i = {T_{\alpha\beta}}^i_j\tilde{s}^j.$ Since $T_{\alpha\beta}$ is biholomorphic, we have:
	\begin{align*}
	\dbar_E(s) = \dbar(s^i)\otimes e_i &= \dbar({T_{\alpha\beta}}^i_j \tilde{s}^j)\otimes f_i\\
	&= {T_{\alpha\beta}}^i_j \dbar(\tilde{s}^j)\otimes f_i.
	\end{align*}
	Hence $\dbar_E$ transforms with $T_{\alpha\beta}$ and it is globally well defined. We call $\dbar_E$ the \textit{Dolbeault Operator} corresponding to the holomorphic structure on $E$. Conversely, if we have a differential operator $\dbar_E:\Gamma(E) \to \Omega^{0,1}(\Sigma)\otimes \Gamma(E)$, we can define a holomorphic structure on $E$ as operator defines local holomorphic structure by defining $s$ to be holomorphic if $\dbar_E(s)=0$, and these local structures can always be glued to give a global structure when $\Sigma$ is a Riemann surface \cite[\S5]{atiyah_yang-mills_1983}.
	
	Therefore, in order to study the space of holomorphic structures on $E$, we can equivalently study the space of Dolbeault operators on $E$. In a smooth local trivialization of $E$, we can write 
	\begin{equation*}
	\dbar_E = \dbar + B,
	\end{equation*}
	where $\dbar$ is the usual Cauchy-Riemann operator and $B \in \Omega^{0,1}(E, \End E)$.
	
	On an arbitrary complex manifold, there may be an obstruction to $B$'s integrability, which lives in $\Omega^{0,2}(\Sigma)$. However $\dim \Sigma = 1$, so $\Omega^{0,2}(\Sigma)=0$ and there is no constraints on $B$. Therefore the set of structures is an affine complex space with translations $\Omega^{0,1}(M,\End E)$. We want to consider only equivalence classes of hermitian vector bundles, so we want to quotient out the action of $\text{Aut}(E) = \C^\infty(\Sigma, GL_n\C)$ by change of basis. It is the space of such isomorphism classes, $N(n,k)$, that we wish to describe. 
	
	In order to put geometric structure on this space, we need to add an additional constraint.
	\begin{definition}
		\label{d:stable}
		Let the \emph{slope} of a bundle $E$ be
		$$\mu := \deg(E)/\text{rank}(E),$$
		where $\deg(E)$ denotes the first Chern class of the line bundle $\det E$. Then $E$ is said to be \textit{stable} if, for every proper subbundle $F$ of $E$, $\mu(F) < \mu(E)$. If the inequality is not strict, $E$ is \textit{semi-stable}.
	\end{definition}

	The Narasimhan-Seshadri correspondance tells us that to study the moduli space of flat $U(n)$ connections, one should restrict their focus to the subspace of semi-stable bundles.
	\begin{theorem}[Narasimhan-Seshadri]
		\label{t:n-s}
		Let $\Sigma$ be a compact connected Riemann surface with $g\geq 2$ and $G=U(n)$. Then 
		\begin{enumerate}
			\item There is a correspondence between representations $\rho$ up to conjugation and semi-stable holomorphic bundles $E$ of degree zero up to gauge equivalence.
			\item $E$ is stable if and only if $\rho$ is irreducible.
		\end{enumerate}
	\end{theorem}
	\begin{proof}
		The original proof of Narasimhan and Seshadri \cite{narasimhan_stable_1965} is algebraic, and there is more recent proof of Donaldson \cite{donaldson_new_1983} using the Yang-Mills functional on connections.
	\end{proof}
	This theorem in combination with Theorem \ref{t:phi-bijection} tells us that there are three equivalent sets we can use to describe the moduli space of flat connections. We can look at flat connections, representations of the fundamental group, or semi-stable holomorphic bundles.
	
	For this reason, we will restrict our attention to only the subset of $N(n,k)$ consisting of semi-stable bundles; $N_{ss}(n,k)$. In particular, motivated by Theorem \ref{t:n-s}, we will denote the space of degree $0$ semi-stable $SL(n,\C)$ bundles as $\MM$. The next result tells us that $N_{ss}(n,k)$ has a well-defined geometric structure. 
	\begin{theorem}
		For a compact connected Riemann surface $\Sigma$ of genus $g$, there exists a connected complex projective variety $N_{ss}(n,k)$ of semi-stable  holomorphic bundles. When $n$ and $k$ are co-prime, $N_{ss}(n,k)$ is a smooth manifold. 
	\end{theorem}
	\begin{proof}
		Originally proven by Mumford \cite{mumford_projective_2004}, see also an outline given by Thaddeus \cite[4]{andersen_introduction_2021}.
	\end{proof}
	Remark: For $g=0$ there are no stable holomorphic bundles. It is a theorem of Grothendieck \cite[Theorem 2.1]{grothendieck_sur_1957} that any holomorphic bundle $E$ over $\mathbb{P}^1$ can be written as $E \cong \oplus_{i=1}^{\text{rank} E} \OO(n_i)$, which lets us verify that $E$ is at best semi-stable. 

	Now let us focus only on $\MM$. Just as in the representation picture, we will write $\MM_0$ to denote the stable bundles. $\MM_0$ is a smooth manifold \cite[\S7]{atiyah_yang-mills_1983}, and we can talk about its geometry. Being degree 0 means that the line bundle $\det E$ is topologically trivial, and a choice of global trivialization gives us an $SL(n,\C)$ structure on $E$. To preserve this trivialization, we will restrict $\text{Aut}(E)$ and $\End E$ to their intersections in $SL(n,\C)$ and $\mathfrak{sl}(n,\C)$ respectively.
	
	An important property of stable bundles which we will make use of is the \emph{stable implies simple lemma}:
	\begin{lemma}[Stable implies simple]
		If $E$ is stable, then $H^0(\Sigma,\End E) = \C$, and $H^0(\Sigma,\mathfrak{sl}(E)) = 0$.
		\label{l:stablesimple}
	\end{lemma}
	\begin{proof}
		Suppose $f\in H^0(\Sigma,\End(E))$, $\lambda \in \mathbb{C}$. Then $\ker(f)$ and im$(f)$ are subbundles of $E$ and we have the exact sequence
		\begin{equation}
			0\to \ker(f) \to E \to \text{im}(f)\to 0,
		\end{equation}
		therefore $c_1(\ker(f))c_1(\text{im})(f) = c_1(E)$ and $\text{rank}(\ker(f)) + \text{rank im}(f) = n$. This forces that either $\mu(\ker(f))$ or $\mu(\text{im}(f))$ must be greater than or equal to $\mu(E)$, and hence either $\ker(f) = E$ or $\text{im}(f)=E$ since $E$ is stable.
		
		Now for $\mathds{1}\in H^0(\Sigma,\End(E))$, the argument above applied to $f-\lambda \mathds{1}$, $\lambda \in \mathbb{C}$ shows that $f = \lambda \mathds{1}$ and therefore $H^0(\Sigma,\End(E)) =\C$. Since $\mathds{1}$ is not traceless, it is not in $\mathfrak{sl}(E)$, and hence $H^0(\Sigma,\mathfrak{sl}(E)) =0$.
	\end{proof}
	Since stability is an open condition, one may consider deformations to compute the tangent space. At a bundle $(E,\dbar_E)$ with holomorphic structure given by transition functions $T_{\alpha,\beta}$, we can consider deforming the holomorphic structure to 
	\begin{equation}
	T_{\alpha,\beta}(\epsilon) = T_{\alpha,\beta} + \epsilon t_{\alpha,\beta},
	\end{equation}
	where $t_{\alpha,\beta}$ is a Čech 1-cochain in $\End(E)$ and $\epsilon^2=0$. For this to remain a well-defined holomorphic structure, we require that $T_{\alpha,\beta}(\epsilon)$ satisfies the cocycle condition for all $\epsilon$. That is, on $U_{\alpha}\cap U_{\beta}\cap U_{\gamma}$,
	\begin{align*}
	T_{\alpha,\beta}(\epsilon)T_{\beta,\gamma}(\epsilon) &= T_{\alpha,\gamma}(\epsilon)\\
	\left(T_{\alpha,\beta} + \epsilon t_{\alpha,\beta} \right)
	\left(T_{\beta,\gamma} + \epsilon t_{\beta,\gamma} \right) &=
	\left(T_{\alpha,\gamma} + \epsilon t_{\alpha,\gamma} \right)\\
	T_{\alpha,\beta}T_{\beta,\gamma} + \epsilon(t_{\alpha,\beta}T_{\beta,\gamma} + T_{\alpha,\beta} t_{\beta,\gamma}) + \epsilon^2 t_{\alpha,\beta}t_{\beta,\gamma} &= T_{\alpha,\gamma} + \epsilon t_{\alpha,\gamma}
	\end{align*}
	using that $\epsilon^2 = 0$ and $T_{\alpha,\beta}$ satisfy the cocycle condition, we have
	\begin{align*}
	T_{\alpha,\gamma} + \epsilon(t_{\alpha,\beta} T_{\beta,\gamma} + T_{\alpha,\beta} t_{\beta,\gamma}) &= T_{\alpha,\gamma} + \epsilon t_{\alpha,\gamma}\\
	t_{\alpha,\beta} T_{\beta,\gamma} + T_{\alpha,\beta} t_{\beta,\gamma} &= t_{\alpha,\gamma}.
	\end{align*}
	This condition tells us that $t_{\alpha,\beta}$ is a 1-cocycle in the sheaf $\End(E)$. When we quotient the action of of $\text{Aut}(E)$, we find that the tangent space to $N(n,k)$ is $H^1(\Sigma,\End(E))$. Similarly, if we include an $SL(n,\C)$ structure, we get the tangent space of $\MM$, $T_E\MM = H^1(\Sigma,\mathfrak{sl}(E))$.
	
	\begin{theorem}
		If $\Sigma$ has genus $g\geq 2$ and $E$ is stable, then $\dim H^1(\Sigma,\End E) = n^2(g-1)+1$, and $\dim H^1(\Sigma,\mathfrak{sl}(E)) = (n^2-1)(g-1)$.
	\end{theorem}
	\begin{proof}
		We can compute the dimension of $H^1(\Sigma,\End(E))$ via Hirzebruch-Riemann-Roch. 
		\begin{equation}
			\label{e:hirz-rr}
			\dim H^0(\End(E)) - \dim H^1(\End(E)) = \int_\Sigma ch(L)\text{Td}(\Sigma),
		\end{equation}
		where $ch(V)$ is the Chern character and $\text{Td}(\Sigma)$ is the Todd class of $T\Sigma$. We know from the stable implies simple lemma (\ref{l:stablesimple}) that $H^0(\End E)=\C$. For a compact Riemann surface, the Todd class is $1+c_1(T\Sigma)/2 = 1+(1-g) = 2-g$, and for a vector bundle $V$ the Chern character is $\text{rank}(V) + c_1(V)$. 
		
		Since $\End E = E\otimes E^\ast$, its rank is $n^2$ and its Chern class is
		\begin{align*}
			c_1(\End E) &= (\text{rank} E) c_1(E) + (\text{rank} E^\ast)\\
			&= n c_1(E) - nc_1(E)\\
			&= 0
		\end{align*}
		Using these computations, equation \ref{e:hirz-rr} becomes 
		\begin{align*}
			c_1(\End(E)) + \text{rank}(\End(E))(1-g) &= \dim H^0(\End(E)) - \dim H^1(\End(E))\\
			\dim H^1(\End(E)) &= 1 - n^2(1-g) = n^2(g-1) + 1.
		\end{align*}
		For $H^1(\Sigma,\mathfrak{sl}(E))$, we instead have from lemma \ref{l:stablesimple} that the dimension of $H^0 = 0$ and $\text{rank}~\mathfrak{sl}(E) = (n^2-1)$ so we obtain:
		\begin{align*}
			\dim H^1(\mathfrak{sl}(n,\C)) &=  0-(n-1)^2(1-g) = (n-1)^2(g-1).
		\end{align*}
	\end{proof}

	When $E$ has a hermitian metric $h:E\otimes E \to \C$, the conjugate Hodge star $\bar{\star}:\Omega^{0,1}(\Sigma) \to \Omega^{1,0}(\Sigma)$ combined with $h$ allows us to define a hermitian inner product on $H^1(\End (E))$.  First $h$ defines a metric on $\End E$; if $A,B\in \End E$, let
	\begin{equation}
	g(A,B) = \Tr(A^\dagger B),
	\end{equation}
	where $\dagger$ is defined in terms of $h$, by $h(Ae,e) = h(e,A^\dagger e)$ for all $e\in E$. Then for any $\alpha = A \otimes a$, $A\in \End E$ and $a \in \Omega^{0,1}(\Sigma)$, we define:
	\begin{align}
	\bar{\ast}_E \alpha = g(A,-)\otimes \bar{\ast}a, 
	\end{align}
	and
	\begin{equation}
	\langle \alpha, \beta \rangle = \int\limits_\Sigma \alpha \wedge_g \bar{\ast}_E \beta =\int\limits_\Sigma g(A,B)~a\wedge \bar{\ast} b.
	\end{equation}
	In a local co-ordinate chart where $\alpha = Adz$ and $\beta = Bdz$, this takes the form
	\begin{equation}
	\langle \alpha, \beta \rangle = \int\limits_\Sigma \Tr(A^\dagger B)~dz\wedge d\bar{z}.
	\end{equation}
	The relationship between this space of connections and the space of holomorphic vector bundles is described by the Narasimhan-Seshadri theorem (\ref{t:n-s}). In one direction, given a flat connection $A$ on $P$, inducing a connection on the associated bundle $E$, we can decompose $A = A^{0,1} + A^{1,0}$. This allows us to take $\dbar_E = A^{0,1}$ as a complex structure on $E$ corresponding to the flat connection $A$. The Narasimhan-Seshadri theorem guarantees that this structure will define a stable bundle, and also gives the converse direction; that flat stable structures $\dbar_E$ define flat unitary connections $A$. 

	\section{Symplectic Picture}
	Another description of the moduli space $\MM$ is in terms of a symplectic reduction, which makes the symplectic structure more clear. We will be considering an infinite dimensional symplectic manifold and taking a symplectic quotient, which requires more careful consideration then we provide here. Rigourous details of this picture can be read in Atiyah-Bott \cite{atiyah_yang-mills_1983}.
	
	Again let $\Sigma$ be the compact connected Riemann surface of genus $g$. Consider the trivial principal $G=SU(2)$ bundle $P$, over $\Sigma$ and let $\cA$ denote the space of smooth principal connections on $P$. In a fixed trivialization $P \cong G\times \Sigma$, a connection is determined by a form $A \in \Omega^1(\Sigma)\otimes \mathfrak{g}$. A connection is flat if and only if it has zero curvature, $0 = F_A:= dA + A \wedge A$. The gauge group $\cG = \Hom(\Sigma, G)$ acts on $\cA$ as follows: for $g\in \cG$,
	 \begin{equation}
	 g\circlearrowright A := g^{-1}Ag + g^{-1}dg.
	 \end{equation}
	 Therefore to find the moduli space of gauge equivalence classes of connections, we want to consider a quotient $\cA/\cG$. This quotient will not be finite dimensional in general, so we want to impose the further constraint that $F_A = 0$. 
	
	The vector space $\cA$ has a natural symplectic structure, which comes from the inner product (the Killing form) on the Lie Algebra $\mathfrak{g}$, $K:\mathfrak{g}\otimes\mathfrak{g}\to\mathbb{C}$. If $A = \alpha \otimes X$ and $B = \beta \otimes Y$ then we can define
	\begin{equation}
	\label{e:ab-form}
	\omega(A,B) = \int\limits_\Sigma K(X,Y)\alpha\wedge \beta.
	\end{equation}
	If $\cA$ were a finite dimensional symplectic manifold, to obtain the quotient of the flat connections $\cA_0$ with $F_A =0$ by $\cG$, we could check that $F_A$ is a moment map for the action of $\cG$ and that $\omega$ is preserved by the action, to then obtain the symplectic quotient $\MM = F_A^{-1}(0)/\cG = \cA_0/\cG$. Although $\cA$ is infinite-dimensional, this process still works, and yields a finite dimensional moduli space.
	\begin{theorem}
		The symplectic structure $\omega$ defined above is invariant under the action of $\cG$ on $\cA$. Furthermore, the curvature $F_A$ is a moment map for this action.
	\end{theorem}
	\begin{proof}
	Atiyah and Bott \cite[\S9]{atiyah_yang-mills_1983}, at the end of section 9.
	\end{proof}
	Then we consider the moduli space of flat connections by symplectic reduction:
	\begin{equation}
	\MM = \cA\sslash\cG = F^{-1}(0)/ \cG= \cA_0/ \cG.
	\end{equation} 
	Symplectic reduction also gives us a symplectic structure on the quotient space $\MM$, such that under the pullback by the quotient map, $q:\cA_0 \to \MM$, we recover the symplectic form $\omega$ for $\cA$. This symplectic structure on $\MM$ will be called the \emph{Atiyah-Bott symplectic form} when we need to distinguish it from other forms on $\MM$. 

		
	
\section{Geometric Quantization when $G=SU(2)$}

	Using any of the descriptions above, we have the space $\MM$ of flat $SU(2)$ connections on a compact connected Riemann surface $\Sigma$ of genus $g\geq 2$, which is a quasiprojective scheme with dimension $3g-3$. Furthermore, $\MM$ is equipped with the Atiyah-Bott symplectic form $\omega$, and there exists a line bundle $\LL$ over $\MM$ with curvature $2\pi i \omega$ \cite{quillen_determinants_1985}\cite{ramadas_comments_1989}. This is the data of a prequantum system to which we can apply geometric quantization.
	
	Jeffrey and Weitsman describe this quantization following an approach of \cite{weitsman_real_1992}, which is for a compact symplectic manifold $(M,\omega)$ and line bundle $\LL$, using a real polarization of $M$. A real polarization of $M$ is a map $\pi:M\to B$ onto a manifold of half dimension, such that $\omega|_{\pi^{-1}(b)} =0$ for all $b\in B$. Supposing $\pi:M\to B$ is also a fibration, there will be a finite set of \textit{Bohr-Sommerfeld points} $b_i$ for which $\LL$ restricted to the fibers $L_{b_i}$ of $\pi$ possesses global covariant constant sections. Denote $J_\pi$ denote the sheaf of sections of $\LL$ which are covariant constant along the fibres of $\pi$. Then the quantization of a prequantum system $(M,\omega,\LL)$ is the vector space
	\begin{equation}
		\mathcal{H} = \bigoplus_{i=0}^{\dim M} H^i(M,J_\pi).
	\end{equation}
	It is the dimension of $\mathcal{H}$ which we hope to compute. If $B_s$ is the set of all Bohr-Sommerfeld points, and for each $b\in B_s$ $S_b$ is the space of global covariant constant sections of $\LL|_{\pi^{-1}(b)}$, then Sniatycki (cite) proves that there is a natural isomorphism:
	\begin{equation}
		\mathcal{H} \cong \bigoplus_{b\in B_s} S_b.
	\end{equation}
	Since each $S_b$ is one dimensional, counting $\dim \mathcal{H}$ boils down to counting the Bohr-Sommerfeld points. 
	
	In this case, the above theorem does not apply because $\MM$ is not a smooth manifold and the polarisation we will describe is not a fibration. Sniatycki's theorem simply provides inspiration for investigating the Bohr-Sommerfeld set in $\MM$, and Jeffrey and Weitsman show that Bohr-Sommerfeld fibres are associated to marked trivalent graphs satisfying the quantum Clebsch-Gordan conditions, and the number of such graphs is called the \emph{Verlinde dimension}, counted by the \emph{Verlinde formula} \cite[Thm. 8.1]{jeffrey_bohr-sommerfeld_1992}.

\subsection{Polarisation of the Moduli Space}	
	As before, let $\Sigma$ be a compact Riemann surface and $\MM$ the moduli space of flat $G=SU(2)$ connections on $\Sigma$. Following Jeffrey and Weitsman, we describe an action of $T^{3g-3}$ on $\MM$. Let $C$ be a closed oriented curve in $\Sigma$ and pick a basepoint $y\in C$. We can define a function $\tilde{f}_C:\cA \to \mathbb{R}$ by 
	\begin{equation}
		\tilde{f}_C(A) = \frac{1}{2}\text{hol}_C(A),
	\end{equation}
	where hol$_C(A)$ means the holonomy of $A$ around $C$ from $y$ to $y$. Since the holonomy is $\cG$ invariant, this passes to $f_C:\MM \to \mathbb{R}$. $\Sigma$ admits a decomposition into \textit{trinions} or \textit{pairs of pants}, which are copies of a disc with two holes:
	\begin{equation}
		D = \{z \in \mathbb{C}~|~ |z|\leq 2 \} - \{z~|~|z-1|<1/2\}\cup \{z~|~ |z+1| < 1/2\},
	\end{equation}
	with marked points on the boundary of $D$. Suppose we are given such a decomposition of $\Sigma$ into $2g-2$ trinions $D_\gamma$, $\gamma\in\{1,2,...,2g-2\}$, joined along their boundaries and with the marked points on the boundaries coinciding for any trinions with non-trivial intersection. Then the boundary circles of $D_\gamma$ give a collection $C_i$, $i\in\{1,2,...,3g-3\}$ of closed oriented curves in $\Sigma$ for which we get corresponding functions $f_i = f_{C_i}:\MM \to \mathbb{R}$ using the above definition. Since these functions are the trace of $SU(2)$ matrices, they can be described by cosine of angles $\theta_i$,
	\begin{equation}
		\theta_i(A) = \cos^{-1}(f_i(A)),
	\end{equation}
	where $\theta_i$ is taken to lie in $[0,\pi]$. This defines a map $\theta = (\theta_1,...,\theta_{3g-3}):\MM \to \mathbb{R}^{3g-3}$. These $\theta_i$ are smooth on $U_i := \theta_i^{-1}(0,\pi) \subset \MM$, which is open and dense. Thus, the Hamiltonian flows of each $\theta_i$ are defined on $\MM^{s} = \bigcap_{i=1}^{3g-3} U_i \subset \MM$. These Hamiltonian flows are periodic with constant period, which means they induce a torus action on $\MM^{s}$. Explicitly, if we let $X_i$ denote the Hamiltonian vector field of $\theta_i$, defined by
	\begin{equation}
		\iota_{X_i}\omega = d\theta_i,
	\end{equation}
	and let $e^{tX_i}$ be the corresponding vector field flow, then the action is given by $g = (\alpha_1,...,\alpha_{3g-3}) \in T^{3g-3}$ acts by
	\begin{equation}
		A \to e^{\alpha_1 X_1 + ... + \alpha_{3g-3}X_{3g-3}}A.
	\end{equation}
	 The Lie algebra of $T^{3g-3}$ is $
	\mathbb{R}^3$ and we interpret $\theta(A)$ as being dual by $\langle \theta, X \rangle = \sum \theta_i X_i$. Then
	\begin{equation}
		d\left(\langle \theta(A),X\rangle\right) = d\sum\theta_i X_i = \sum X_i d\theta_i = \iota_{X}\omega,
	\end{equation}
	which means $\theta$ is the moment map for the torus action. These functions $f_i$ also give us a real polarization of $\MM$. Let $B \subset \mathbb{R}^{3g-3}$ be the image of the $f_i$,
	\begin{equation}
		B = \{(f_i(E),...,f_{3g-3}(E))~|~ E \in \MM\},
		\label{e:B-def}
	\end{equation}
	then the fibers of the map $\pi = (f_1,...,f_{3g-3})$ foliate the smooth locus of $\MM$, and the generic fibre is a Lagrangian subvariety (cite J\&W). 
	
	Alternatively, one can describe the polarization using the picture of connections as representations of the fundamental group $\pi_1(\Sigma)$. First, a preliminary result. Let $T\subset SU(2)$ be a maximal torus.
	\begin{definition}
		A connection $A$ on $\Sigma^g$ is said to be \emph{adapted to a trinion decomposition} (a.t.d.) if there is a tubular neighbourhood $V_i \cong (-1,1)\times S^1$ of each boundary circle $C_i$ in the decomposition, such that in co-ordinates $(s,\theta)$ for $V_i$,
		\begin{equation}
			A|_{V_i} = X_i d\theta, 
		\end{equation}
		where $X_i$ is a constant element in $\mathfrak{t} = \text{Lie}(T)$.
	\end{definition}
	\begin{theorem}
		For all $y\in \pi^{-1}(b)$, there exists an adapted to trinion decomposition connection $A$ in the gauge equivalence class $y$.
	\end{theorem}
	\begin{proof}
	tbd
	\end{proof}
	
	This lets us define subgroups of $G=SU(2)$, which correspond to stabilizers of flat connections. Suppose $A$ is an a.t.d connection. Then the stabilizer of $A|_{C_i}$ in $\cG(C_i) = \Hom(C_i, G)$ consists of constant maps, and can thus be identified with a subgroup $H_i$ in $G$. If $\theta_i(A) \in \{0,\pi\}$, then $\text{hol}_{C_i}(A) = \pm\text{Id}$ and so $H_i = G$. Otherwise, $H_i = T$. 
	
	We can describe the fibre $\pi^{-1}(b)$ using these subgroups. Suppose $A$ is a.t.d. and $[A] \in \pi^{-1}(b)$. Let $\tau_i \in H_i$ for each circle $C_i$, $i\in (1,2,...,3g-3)$. Then define the map
	\begin{equation}
		\label{e:psiA}
		\psi_A : \prod_{i=1}^{3g-3} \to \pi^{-1}(b)
	\end{equation}
	as follows. Denote the trinions composing $\Sigma$ as $D_{\gamma}$, $\gamma\in{1,2,...,2g-2}$. For any circle $C_i$, let $D_{\gamma(i)}$, $D_{\gamma'(i)}$ be the trinions on either side. For $\tau=(\tau_1,\tau_2,...,\tau_{3g-3})$, choose a collection of maps $\zeta_\gamma : D_\gamma \to g$ such that for every $C_i$, $\zeta_{\gamma(i)}$ and $\zeta_{\gamma'(i)}$ are constant on a tubular neighbourhood of $C_i$, and such that
	\begin{equation}
		\zeta_{\gamma(i)}|_{C_i} = \tau_i \zeta_{\gamma'(i)}|_{C_i}.
	\end{equation}
	Here, adopt the convention that the orientation of the tubular neighbourhood is $v\wedge w$, where $w$ is tangent to the oriented circle $C_i$ and $v$ is transverse to $c_i$ and pointing \textit{into} $D_{\gamma(i)}$, thus away from $D_{\gamma'(i)}$.
	
	Now we define a connection $A_\tau$ on $\Sigma$ by defining $A_\tau$ on each trinion: $A_\tau|_{D_\gamma} := \zeta_\gamma \circlearrowright A|_{D_\gamma}$. Finally define $\psi_A(\tau) = [A_\tau]$. Next we ask, for $\tau,\tau' \in \prod_{i=1}^{3g-3} H_i$, when are $A_\tau$ and $A_{\tau'}$ gauge equivalent? 
	
	Let $J_\gamma$ be the stabilizer of $A|_{D_\gamma}$ under $\cG|_{D_\gamma} = \Hom(D_\gamma,G)$. Since $A$ is a.t.d., this also consists of constant maps. $J_\gamma = Z(G) = \{\pm \text{Id}\}$ if the holonomy is an irreducible representation of $SU(2)$, and otherwise $J_\gamma =T$ (resp $G$) if the holonomy reduces to $T$ (resp $Z(G)$).  
	
	Jeffrey and Weitsmasn give us the following lemma and theorem:
	\begin{theorem}
		If $\tau,\tau'$ are in $\prod_{i=1}^{3g-3} H_i$, then $[A_\tau] = [A_{\tau'}]$ if and only if there is a set of gauge transformations $\Phi_\gamma:D_\gamma \to G$ such that:
		\begin{enumerate}
			\item $\Phi_\gamma \in J_\gamma$ for all $\gamma$.
			\item For each boundary circle $C_i$, we have
			$$
				\Phi_{\gamma'(i)}|_{C_i} \tau_i = \tau_i' \Phi_{\gamma(i)}|_{C_i}.
			$$
		\end{enumerate}
	\end{theorem} 
	\begin{theorem}
		The map $\psi_A:\prod_i H_i \to \pi^{-1}(b)$ is surjective and the group $\prod_\gamma J_\gamma$ has a natural action on $\prod_i H_i$ so that
		\begin{equation}
			\pi^{-1}(b) = \left(\prod_i H_i\right)/\left(\prod_\gamma J_\gamma\right).
		\end{equation}
	\end{theorem}
	\begin{proof}
		jeffrey and weitsman page 600
	\end{proof}

\subsection{Moduli of Connections on a Trinion}
 	Now we've seen that the space $\MM$ can be constructed by glueing together connections defined along a trinion decomposition, so the natural question is what the space of connections on a trinion $D$, denoted $\MM(D)$, looks like. The space $\MM(D)$, like $\MM$, can be described the quotient of representations of the fundamental group into $G$ under conjugation by $G$. For a trinion, 
	\begin{equation}
		\pi_1(D) = \left\{
		[C_1], [C_2], [C_3] ~|~ [C_1][C_2][C_3]  =1
		\right\},
	\end{equation}
	where $C_i$ are the three boundary curves of the trinion. We can again define the holonomy angle functions, first letting $\tilde{\theta_i}:\Hom(\pi_1(D),G) \to [0,\pi]$ be
	\begin{equation}
		\tilde{\theta_i}(\rho) = \cos^{-1}\left(\frac{1}{2}\Tr(\rho[C_i])\right),
	\end{equation}
	and these maps will descend under the quotient by $G$ to maps $\theta_i:\MM(D)\to [0,\pi]$. Then Jeffrey and Weitsman prove:
	\begin{theorem}
		The map $\theta = (\theta_1, \theta_2, \theta_3):\MM(D)\to[0,\pi]^3$ sends $\MM(D)$ bijectively to the set satifying the inequalities
		\begin{equation}
			|\theta_1 - \theta_2| \leq \theta_3 \leq \min(\theta_1 + \theta_2, 2\pi - (\theta_1 + \theta_2)).
			\label{e:trinion-ineqs}
		\end{equation}
		Remark that this set is a 3D simplex (insert figure?)
	\end{theorem}
	\begin{proof}
		J\&W prop 3.1
	\end{proof}
	Using this result, and the gluing process described in the last section, the image of $\MM$ under the holonomy angles $\theta_1,...,\theta_{3g-3}$ are the values satisfying the inequalities (\ref{e:trinion-ineqs}) on every trinion. Applying a theorem of Guillemin and Steinberg (cite) to this case, one obtains:
	\begin{theorem}
		\label{t:torusfibres}
		Suppose $x\in \pi(\MM) \subset B$ (defined in eqn. \ref{e:B-def}) Then
		\begin{itemize}
			\item The Hamiltonian vector fields corresponding to the functions $\theta_i$ are linearly independent on the fibre $\pi^{-1}(x)$, if and only if $x$ is a point where all the inequalities (\ref{e:trinion-ineqs}) are strict.
			\item In general, the number of linearly independent Hamiltonian vector fields on the fibre $\pi^{-1}(x)$ is equal to $3g-3-s$, where $s$ is the number of independent linear equations out of the following satisfied by $\theta(x)$:
				\begin{align*}
					\theta_{i_{\sigma(1)}(\gamma)}(x) + \theta_{i_{\sigma(2)}(\gamma)}(x)-\theta_{i{_\sigma(3)}(\gamma)}(x)&=0,\\
					\theta_{i_1(\gamma)}(x) + \theta_{i_2(\gamma)}(x) + \theta_{i_3(\gamma)}(x) = 2\pi.
					\label{e:theta-ineqs}
				\end{align*}
			where $\sigma:{1,2,3}\to{1,2,3}$ is any cyclic permutation. These inequalities correspond to $(\ref{e:trinion-ineqs})$.
		\end{itemize}
	\end{theorem}
	Furthermore
	\begin{lemma}
		Let $x\in \MM(D)$ and let $\theta(x)$ be the holonomy angles of $x$ around the three boundary curves of $D$. Then $x$ corresponds to a conjugacy class of reducible representations of $\pi_1(D)$ if and only if at least one of the equations (\ref{e:theta-ineqs}) above is satisfied.
	\end{lemma}
	Motivated by this lemma, one defines \emph{interior triples} in $[0,\pi]^3$ to be those for which none of the equations is satisfied, i.e those on the interior of the simplex. These triples correspond to points in $\MM(D)$ which are conjugacy classes of irreducible representations of $\pi_1(D)$. Theorem \ref{t:torusfibres} tells us that the fibre $\pi^{-1}(x)$ is a torus if and only if $\theta(x)$ is an interior triple.
	\begin{theorem}
		Let $x\in B$ and let $A$ be a flat a.t.d. connection whose gauge equivalence class is in $\pi^{-1}(x)$. Further, assume that on every trinion the holonomy angles of $x$ is an interior triple. Then the fibre $\pi^{-1}(x)$ is identified with $T^{3g-3}/(\mathbb{Z}_2)^{2g-2}$ under the map $\psi_A$ defined in eqn. (\ref{e:psiA}).
	\end{theorem}
	Thus, the fibres corresponding to interior triples, which are the generic fibres, are tori of dimension $3g-3$. It is the non-generic fibres, corresponding to reducible points, which cause the result of Sniatycki to fail here.
	
	
%\printbibliography
\end{document}
