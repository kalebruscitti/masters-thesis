	
	
	\section{Extended Moduli Spaces of Connections}
	As always, let $\Sigma$ be a compact connected genus $g$ Riemann surface. Recall that one construction of our moduli space $\MM$ was as the representations $\Hom(\pi_1(\Sigma),G)/G$, which has its subset topology as $\xi^{-1}(e)/G \subset G^{2g}/G$, where 
	\begin{equation}
		\xi(A_1,B_1,...,A_g,B_g) = \prod_{i=1}^g A_iB_iA_i^{-1}B_i^{-1}.
	\end{equation} 
	Now let us generalize to the case where $\Sigma$ is permitted to have $n$ punctures, $\{p_i\}_{i=1}^n$. Let $c_i$ be a loop around $p_i$, and let $\{d_i\}_{i=2}^n$ be curves $d_i:p_1\to p_i$ in $\Sigma$. Then the fundamental group of $\Sigma$ can be written \cite{hurtubise_representations_2000}[2.2]:
	\begin{equation}
		\pi_1(\Sigma) = \langle a_i, b_i, c_1, d_jc_jd_j^{-1} ~|~ \prod_{i=1}^{g}a_ib_ia_i^{-1}b_i^{-1}c_1\prod_{j=2}^{n}d_jc_jd_j^{-1}=e\rangle.
	\end{equation}
	Thus, we will redefine $\xi:G^{2g+2n-1}\to G$ as
	\begin{equation}
		\xi(a_1,b_1,...,a_g,b_g,c_1,...,c_n,d_2,...,d_n) = \prod_{i=1}^{g}a_ib_ia_i^{-1}b_i^{-1}c_1\prod_{j=2}^{n}d_jc_jd_j^{-1}.
	\end{equation}
	Suppose $\{C_i\}_{i=1}^{2g-g}$ is a trinion decomposition for an unpunctured surface $\tilde{\Sigma}$. Let $\Sigma_i$ be the (singular) surface obtained by shrinking $C_i$ to a point $p_i$ (needs clarification). Then proposition \ref{t:atd-thm} tells us that any element $A$ in $\MM$ can be represented by an A.T.D. connection, meaning that $A$'s holonomy around the curve $c_i$ corresponding to the point $p_i$ is in the fundamental alcove $\Delta\subset \mathfrak{t}$. Repeating this for each $C_i$ tells us that our moduli space corresponds to the $T$-extended moduli space:
	\begin{equation}
		\MM = \MM^T := \{(A_i,B_i,D_j,C_j)\in G^{2g+n-1}\times \exp(\Delta)^n ~|~ \xi(A_i,B_i,D_j,C_j) = \mathds{1}\}.
	\end{equation}
	We can also define the $G$-extended moduli space:
	\begin{equation}
	\MM^G := \{(A_i,B_i,D_j,C_j)\in G^{2g+n-1}\times G^n ~|~ \xi(A_i,B_i,D_j,C_j) = \mathds{1}\},
	\end{equation}
	and it is clear that $\MM^T \subset \MM^G$ by including $\exp(\Delta)\subset G$. There is an action of $G^n$ on these spaces. For $(A_i,B_i,D_j,C_j)\in \MM^G$, an element $\sigma$ in the first copy of $G$ acts by:
	\begin{align*}
		A_j &\to \sigma A_j \sigma^{-1},& 	B_j&\to \sigma B_j\sigma^{-1}
	\end{align*}
	\begin{align*}
		D_j&\to \sigma D_j, & C_1 &\to \sigma C_1\sigma^{-1}
	\end{align*}
	and an element $\sigma$ in the $l$th copy $(l=2,...,n)$ of $G$ acts by
	\begin{align*}
	D_l&\to D_l\sigma^{-1}, & C_l &\to \sigma C_l\sigma^{-1}.
	\end{align*}
	This action restricts to give an action of $G^n$ on $\MM^T$. 
	
	\section{Symplectic Implosion}
	To proceed, we will state some definitions and results of (citations here):
	
	
	\begin{definition}
		For $\MM^G$ defined above, with fundamental alcove $\Delta\subset G$, we define the imploded cross section 
		\begin{equation}
			P := \prod_{\sigma} \frac{\Phi^{-1}(\sigma)}{[G_\sigma, G_\sigma]},
		\end{equation}
		where $\sigma$ ranges over the faces of $\Delta$, and $G_\sigma$ is the stabilizer of $\sigma$ under the action of $G$ by conjugation at a point in the interior of $\sigma$.
	\end{definition}
	TODO: all of this section \pagebreak
	

	
	Now associated to the compact Riemann surface $\Sigma$ we have a symplectic variety $P$ of representations with weighted frames, which is toric for $G=SU(2)$ and has a prequantum line bundle $\LL_P$ corresponding to $\LL$ on $\MM$. Now we will go in detail on the other half of Jeffrey and Hurtubise's construction, building the complex variety $\cP$ of \emph{framed parabolic bundles} over a singular curve $\tilde{\Sigma}$ corresponding to contracting the loops in a trinion decomposition of $\Sigma$. We will see that $P$ and $\cP$ are diffeomorphic, meaning that we can compute sections of prequantum line bundles over $\cP$ using the structure of $P$.
	
	First we will see how to embed $\MM$ into projective space, as this construction will have us transform the data of $\MM$ from holomorphic bundles over $\Sigma$ to sheaves, and will lay the groundwork for the embedding we will require later as a hypothesis of the theorem of Harada and Kaveh. Then we will describe the construction of $\cP$ and the diffeomorphism with $P$.
	\section{Projective Embedding of $\MM$}
	\label{s:m-embedding}
	The moduli $\MM$ of flat $SL(n,\C)$ bundles can be embedded into projective space, as we will now describe following the work of Thaddeus and Giesker (cite). First, fix a line bundle $L$ of sufficiently high degree so that for all $E\in N(k,d)$ $E\otimes L$  is globally generated (and redefine $E$). Then, for some large $N$ we can write $E$ as a quotient:
	\begin{equation}
	\phi:\OO^N \to E.
	\end{equation}
	This quotient induces a map from $\wedge^k(\OO^N) \to \wedge^k(E)$ and the $SL(k,\C)$ structure induces an isomorphism $\mu:\wedge^k(E)\cong L^k$. Hence, a quotient of the trivial bundle induces an element $\hat{\beta}$ in 
	\begin{equation}
	V_1 := \Hom(H^0(\wedge^k(\OO^N)), H^0(L^2)).
	\end{equation}
	Now to pass to $\MM$ we quotient by $GL(k,\C)$. Suppose we have $\phi_2 = \Lambda^{-1} \phi_1 \Lambda$. Then
	\begin{equation}
		\hat{\beta}_2 = \mu(\wedge^k \phi_2) = \mu(\wedge^k \Lambda^{-1}\phi_1 \Lambda) = \det\Lambda \mu(\wedge^k \phi_1) = \det\Lambda~ \hat{\beta}_1.
	\end{equation}
	Therefore the orbits of $GL(k,\C)$ correspond to equivalence classes in $\PP(V_1)$. It is this mapping, which we will denote $\iota:\MM\to \PP(V_1)$, which we claim is an embedding. 
	\begin{lemma}
		The map $\iota:\MM \to \PP(V_1)$ is injective.
	\end{lemma}
	\begin{proof}
		 Suppose $(E_1, \dbar_{E_1})$ and $(E_2, \dbar_{E_2})$ are holomorphic vector bundles for which $\hat{\beta}_1 = \iota(E_1) = \iota(E_2) = \hat{\beta_2}$. Since the $SL(k,\C)$ structure is unique up to $\C^\ast$, this means that
		\begin{equation}
		\wedge^k \phi_1 = \lambda \wedge^k \phi_2
		\end{equation}
		for some $\lambda \in \C^\ast$. Fixing a local trivialization of $E_1$, $E_1|_U \cong U\times \C^k$, and picking a local frame $e_1,..,e_k$, choose any sections $s_1,...,s_k \in \OO^N|_{U}$ so that $\phi_1(e_i) = s_i$. Let $s = s_1 \wedge s_2 \wedge ... \wedge s_k$. Then
		\begin{equation}
		e_1\wedge...\wedge s_k=\phi_1(s_1)\wedge...\wedge \phi_1(s_k)= \wedge^k \phi_1(s) = \lambda \wedge^k \phi_2(s) = \lambda \phi_2(s_1)\wedge...\wedge \phi_2(s_k)
		\end{equation}
		Since $\{e_i\}$ was a local frame, the LHS is non-zero, and thus the RHS is non-zero. Therefore $\{\phi_2(s_i)\}$ is a local frame trivializing $E_2$ on $U$. This map $e_i \to \phi_2(s_2)$ gives an isomorphism of $E_1$ with $E_2$. Note that this isomorphism is not unique as we could have picked other sections $\tilde{s}_i$ with $\phi_1(\tilde{s}_i) = e_i$.
	\end{proof}
	\begin{theorem}
		The map $\iota:\MM \to \PP(V_1)$ is an embedding.
	\end{theorem}
	\begin{proof}
		Since the lemma shows it is injective, it remains to show its derivative is everywhere injective. We give a proof following Thaddeus (GIT and flips page 717).
	\end{proof}

	\section{Parabolic Vector Bundles}
	Given a trinion decomposition of the Riemann surface $\Sigma$, let us fix the holonomies around each boundary loop. If we think of each trinion as a thrice-punctured sphere and consider the space of connections with fixed holonomies on the trinion, then due to Mehta and Seshadri (cite) there is a correspondence between the moduli of $\pi_1(\Sigma)$ representations into $SU(n)$ and that of rank-$n$ holomorphic bundles with an $SL(n,\C)$ structure and a parabolic structure at the punctures of $\Sigma$, which we call a parabolic bundle. Under this correspondence, the eigenvalues of the holonomy get translated into a set of weights for the parabolic structure. 
	
	We want to consider the moduli space of connections with all possible holonomies, and therefore we will want to fit all these moduli of parabolic bundles together, and in such a way that we can even include the $\theta_i = 0,\pi$ cases, which will correspond to weights $0$ and $1$. This is the construction of Hurtubise and Jeffrey which we will describe in the next section. First we lay out the basic definitions and results about parabolic bundles.
	\begin{definition}
		A \emph{parabolic bundle} over a complex manifold $\Sigma$ is a holomorphic vector bundle $E$ over $\Sigma$ with a \emph{parabolic structure}, which is a point of marked points $\{p_1,...,p_n\}$ and for each point, a flag of subspaces in the fibre $E_{p_k}$. 
	\end{definition}
	In particular if $E$ has rank 2, then a parabolic structure on $E$ is a choice of points $\{p_k\}$ and a sheaf homomorphism $\alpha:E\to S$ where $S := \bigoplus_{k} \C_{p_k}$.
	
	There is an adapted notion of stability for parabolic vector bundles.
	\begin{definition}
		Let $\gamma_1,...,\gamma_n \in [0,1]$ be a set of weights. For a subbundle (not neccesarily proper) $F$ of $E$ we set $\mu_i(F) = 1$ if $F_{p_i} \subset \ker\alpha_i$, and $\mu_i = 0$ otherwise. Define $\sigma(F) = \frac{1}{rk(E)}$ if $F=E$ and $0$ otherwise. Then we say a pair $(E,\alpha)$ is \emph{stable} with respect to $\gamma$ if 
		\begin{equation}
		rk(E)\deg(F) < rk(F)\left(\deg(E) - \sum_{i=1}^n \gamma_i\right) 
		+ rk(E)\sum_{i=1}^n(1 - \mu_i(F) + \sigma_i(F))\gamma_i.
		\end{equation}
		If the inequality is not strict, $(E,\alpha)$ is \emph{semi-stable}.
	\end{definition}
	This lemma of Hurtubise and Jeffrey (cite) summarizes some important results about tweighted parabolic bundles.
	\begin{lemma}
		\label{l:ss-lemma}
		If $(E,\alpha)$ is a parabolic bundle semi-stable with respect to weights $\gamma$, then:
		\begin{enumerate}
			\item The kernal of $\alpha$ is torsion free, and the torsion subsheaf of $E$ is non-zero only at the $p_i$, equalling $0$ or $\C$ at each $p_i$.
			\item If $\gamma_i >0$, then $\alpha_i \neq 0$.
			\item If $\gamma_i < 1$, then $E$ is torsion free at $p_i$.
			\item If $\gamma_i \in (0,1)$, one has a parabolic structure at $p_i$, and if all the weights are in $(0,1)$, the stability condition is identical to that of parabolic bundles with weights $(1-\gamma_i)/2$ and $(1+\gamma_i)/2$.
			\item If $(E,\alpha)$ is locally free at $p_i$ and $\alpha \neq 0$, then for $\gamma_i = 0$, there is a family $(E_t, \alpha_t)$, $t\in \C$, of semi-stable pairs such that $(E_t,\alpha_t)\cong (E,\alpha)$ for $t\neq 0$, and $\alpha_0 = 0$. 
			\item If $(E,\alpha)$ is locally free at $p_i$ and $\alpha \neq 0$, then for $\gamma_i = 1$, there is a family $(E_t,\alpha_t)$, $t\in \C$, such that $(E_t, \alpha_t)\cong (E,\alpha)$, $t\neq 0$ and $E_0$ has torsion at $p_i$.
		\end{enumerate}
	\end{lemma}
	\begin{proof}
		Hurtubise and Jeffrey (cite) lemma 4.3
	\end{proof}
	In particular, this lemma tells us that when $\gamma_i\in (0,1)$, which will correspond to connections with holonomy in $(0,\pi)$, we get well-behaved parabolic structures without torsion. It also tells us there are two edge cases to consider. When $\gamma_i = 0$, the parabolic structure vanishes, and when $\gamma_i = 1$ we acquire torsion. To fit all possible holonomies into a moduli space, each of these cases will need to be dealt with.
	
	\section{Moduli of Framed Parabolic Bundles}
	Just as we constructed the moduli space of vector bundles over a Riemann surface, let us now construct that of parabolic vector bundles. This section closely follows section 4 of Hurtubise and Jeffrey (cite). From now on, we restrict our attention to $G=SU(2)$. As always, let $\Sigma$ be a compact connected Riemann surface. Fix a line bundle $L$ of sufficiently high degree so that for all $E\in N(k,d), E\otimes L$ is a globally generated sheaf (and redefine $E = E\otimes L$). Then for some large $N$ we can write $E$ as a quotient of the trivial sheaf on $\Sigma$:
	\begin{equation}
		\phi: \OO^N \to E.
	\end{equation}
	Just as in section \ref{s:m-embedding} we have a mapping $\hat{\beta}$ taking $E$ to $V_1$. Now we add the parabolic data. At a point $p_i$, the map $\alpha_i:E\to \C_{p_i}$ pulls back to $\hat{\alpha}_i = \alpha_i\circ \phi$ in 
	\begin{equation}
		V_2 = H^0(\OO^N)^\ast.
	\end{equation}
	Since we are only interested in $\alpha_i$ up to (independent) scaling, the parabolic bundle $(E,\alpha)$ represents an equivalence class in
	\begin{equation}
		Z := \PP(V_1) \times \PP(V_2) \times ... \times \PP(V_2),
	\end{equation}
	where there are $n$ copies of $\PP(V_2)$, one for each puncture. Then letting $\tilde{\MM}$ denote the space of parabolic vector bundles, and $\tilde{\iota}:\tilde{\MM}\to Z$ denote the map taking $(E,\alpha)$ to $(\hat{\beta},\hat{\alpha})$, we get a closed subvariety $X = \tilde{\iota}\left(\tilde{\MM}\right)$ in $Z$. 

	Now we also have weights $\gamma = (\gamma_1,...,\gamma_n)$ for the parabolic structure. These vary the choice of polarization of $X$, namely the choice of line bundles on which the action of $SL(N,\C)$ \textit{linearises}. Let us recall what this means:
	\begin{definition}
		Given a linear algebraic group $G$ and a $G$-variety $X$, a line bundle $p:L\to X$ \emph{linearises} if there is an action of $G$ on $L$ such that for all $l\in L$, $g\in G$,
		\begin{equation}
			p(g\cdot l) = g\cdot p(l),
		\end{equation}
		and which restricts to a linear isomorphism $L_x \cong L_{g\cdot x}$ on fibres.
	\end{definition}
	In this case $G\cong SL(N,\C)$ which acts on $V_1$ and each copy of $V_2$.
	
	
	Let $\pi_1:Z \to \PP(V_1)$ and $\pi_{2,i}:Z\to \PP(V_2)$ denote the projections to the first factor and to the i-th factor of $Z$ respectively. Let
	\begin{equation}
		L_0 = \pi_1^\ast(\OO(N)),~ \text{ and } ~ L_{1,i} = \pi_1^\ast(\OO(N-1))\otimes \pi_{2,i}^\ast (\OO(2)).
	\end{equation}
	Then the linearisation corresponding to weights $\gamma = (\gamma_1,...,\gamma_n)$ is 
	\begin{equation}
		L_\gamma = (L_0)^{s_0} \otimes \left(
		\otimes_{i=1}^n (L_{1,i})^{s_{1,i}}
		\right),
	\end{equation}
	where $s_0(\gamma_i) = s_{1,i}(1-\gamma_i)$.
	\begin{lemma}
		$L_i$ is a linearisation of the action of $G$ on $X$.
	\end{lemma}
	\begin{proof}
		????????????/
	\end{proof}
	In summary, what we have now are a collection of moduli spaces of parabolic bundles semi-stable with respect to weights $\gamma$, with $\gamma_i \neq 0,1$. We want to fit these spaces together, and in such a way that we can include $\gamma_i = 0,1$. To do this, we put a $(\PP^1)^n$-bundle over $X$,
	\begin{equation}
		Y = \PP(L_0\oplus L_{1,1})\oplus \PP(L_0\oplus L_{1,2})\oplus ... \oplus \PP(L_0\oplus L_{1,n}).
	\end{equation}
	We endow $Y$ with the natural polarisation $\OO(1,1,...,1)$. Now $Y$ contains all the stable points for the various choice of weights, which correspond to the various possible holonomies of the unitary connections. We still need to account for gauge equivalence, which apriori suggests we take the $SL(N,\C)$ quotient. It will turn out that this is not quite correct, as we need to make an adjustment with $\gamma_i = 0,1$.
	
	Consider the space of weighted parabolic bundles as quadruples $(E, \alpha_i, A_i, \gamma)$, where $\alpha_i:E\to \C_{p_i}$ is the parabolic structure, $A_i$ is a subspace of $E_{p_i}$, with $A_i = \ker\alpha_i$ whenever $\alpha_i \neq 0$ (equiv. $\gamma \neq 0$), and $\gamma$ are the weights as usual. When $\gamma_i \neq 0$, we have not added any new information. When $\gamma_i = 0$ we are adding a projective class $A_i$ for the parabolic structure even as the map vanishes.
	
	On the other hand when $\gamma_1 = 1$, the sheaves can acquire torsion meaning they are no longer locally free. To handle this, we first need
	\begin{lemma}[H\& J Lemma 4.11]
		Let $E_t$, $t\in\C$ be a family of coherent rank 2 sheaves over $\Sigma$, with $E_t$ locally free at $p$ for $t\neq 0$, and with $E_0$ having torsion subsheaf $\C_p$ near $p$. Let $\phi_t \in H^0(\Sigma, \wedge^2(E)^\ast)$ be a family of $SL(2,\C)$ structures on $E_t$. Then $\phi_0$ vanishes at $p$.
		\label{t:sl2-lemma}
	\end{lemma}
	\begin{proof}
		If $z$ is a local co-ordinate on $\Sigma$ on an open set containing $p$, such that $z(p)=0$, one can obtain $E_t$ locally (up to reparameterization) from the exact sequence 
		\begin{equation}
			\OO \xrightarrow{(0,t^k,z)} \OO \oplus \OO \oplus \OO \xrightarrow{~~~~~}E_t
		\end{equation}
		for some integer $k$. Then the $SL(2,\C)$ structures $\phi_t$ are multiples of $e_1^\ast \wedge (-ze^\ast_2 + t^k e^\ast_3)$, which vanishes when $z=t=0$.
	\end{proof}
	Since $Y$ is a projective bundle, we can freely tensor with line bundles and write $Y$ as the bundle
	\begin{equation}
		Y = \bigoplus_{i=1}^n\PP\left(\pi_1^\ast\left(
		\OO(-1)
		\right)\oplus \pi_{2,i}^\ast\left(\OO(-2)\right)\right).
	\end{equation}
	In this form, when $\gamma \neq 1$ we have a natural lift of a parabolic bundle $(E,\alpha)\in X$ to $Y$, given by 
	\begin{equation}
		\hat{E} = \left(
		(\hat{\beta}, \hat{\alpha}_1^2),(\hat{\beta},\hat{\alpha}_2^2),...,(\hat{\beta},\hat{\alpha}^2)
		\right),
	\end{equation}
	which we want to extend to the torsion case where $\gamma = 1$. When we have torsion at say $p_i$ ($\gamma_i = 1)$, we could rescale the torsion subsheaf of $E$, modifying $\hat{\alpha}_i^2$ to $c\hat{\alpha}_i^2$ for some $c$. This rescaling should remain in the same equivalence class, and if $\hat{\beta} \neq 0$ then since $\hat{\beta}$ is not rescaled this would not be the case. Therefore we want that the $i$-th component of the lift of $E$ in $Y$ should be $(0,\hat{\alpha}_i^2)$. This is achieved as follows; recall that $\hat{\beta}$ is defined by composing
	\begin{equation}
		\wedge^2(\OO^N) \xrightarrow{\phi} H^0(\Sigma, \wedge^2 E) \xrightarrow{\xi} H^0(\Sigma, L^2)
	\end{equation}
	where $\phi:\OO^N \to E$ is a quotient defining $E$, and $\xi$ was its $SL(2,\C)$ structure. We have a commutative diagram:
	\begin{equation}
		TBD
	\end{equation}
	Then for $\gamma \neq 1$ one has $\xi = \text{ev}_\OO^{-1}\circ \xi^{p_i} \circ \text{ev}_{\wedge^2(E)}$, and so for $\gamma_i=1$ we use this definition to define $\xi^{p_i}$ at the torsion points of $E$. Then, from lemma \ref{t:sl2-lemma}, one has that $\hat{\beta}_i$ vanishes when there is torsion at $p_i$. Using this definition we can define a lift for all $(E,\alpha)\in X$ to $Y$ by $
	\hat{E} = \left(
	(\hat{\beta}, \hat{\alpha}_1^2),(\hat{\beta},\hat{\alpha}_2^2),...,(\hat{\beta},\hat{\alpha}^2)
	\right).$
	
	Next Hurtubise and Jeffrey analyse which elements of $Y$ are stable or semistable as weighted parabolic bundles. In lemma (ref) we saw that torsion in the kernal of $\alpha_i$ destabilised $(E,\alpha) \in X$. The same is true in $Y$:
	\begin{lemma}[H\&J Lemma 4.13]
		A semi-stable element $y\in Y$ corresponds to a bundle parabolic bundle $(E,\alpha)$ with no torsion in the kernal of $\alpha$.
	\end{lemma}
	To allow the the $\alpha_i$ to go to zero but still preserve the information of which projective class we have at $p_i$, we define a new map. Let $(E,\alpha)\in X$ with $SL(2,\C)$ structure $\phi$. Then we can map $(E,\alpha,\phi)\to Y$ by
	\begin{equation}
		(E,\alpha,\phi) \to \bigoplus_{i=1}^n\left(
		(\phi(p_i)^N, \phi(p_i)^{N-1}\alpha_i^2
		\right),
	\end{equation}
	and we define $\hat{Y}$ to be the closure in $Y$ of the image of this map. In this closure, we can take $\alpha_i =0$, but since $\phi(p_i)\neq 0$ we preserve the information of a subspace of $E$ at $p_i$ (unsure?). Then since $\phi(p_i)\neq 0$, lemma \ref{t:sl2-lemma} guarantees $E$ is torsion free at $p_i$. Finally the proposition regarding stability is
	\begin{theorem}[H\&J Prop 4.14]
		Let 
		\begin{equation}
			y = \left(
			(b_1, a_1), (b_2, a_2),...,(b_n, a_n)
			\right)
		\end{equation}
		be a point in $\hat{Y}$. Let $\Gamma(y)$ be the set of $\gamma_i \in [0,1]$ such that $\gamma_i =0$ if $a_i =0$ and $\gamma_i = 0$ if $b_i = 0$. Then $y$ is semi-stable if and only if for one element $\gamma \in \Gamma(y)$, $\pi(y)\in X$ is $\gamma$-semi-stable.
	\end{theorem}
	Therefore semi-stable elements in $\hat{Y}$ all project down to semi-stable elements of $X$ for some choice of weights, and the GIT quotient $\hat{Y}\sslash SL(N,\C)$ corresponds to equivalence classes of quadruples $(E,\alpha_i, \hat{\alpha}_i, \phi)$ where $(E,\alpha)$ is a parabolic bundle, $\hat{\alpha}_i$ is a subspace of $E|_{p_i}$ (which is the kernal of $\alpha_i$ when $\alpha_i \neq 0$) and $\phi$ is an $SL(2,\C)$ structure. However this is not quite the final moduli space we want to construct, as we have added the extra information of $\hat{\alpha}_i$ when $\alpha_i = 0$. At these points, $\hat{\alpha}_i \in \PP_1=\PP(E|_{p_i})$. We want to collapse these extra $\PP_1$s. To do this, embed $V_1$ into $W_1 = V_1^{\otimes N}$ so that a non-zero element $l$ in $L_{0} = \pi_1^\ast(\OO(N))$ can be thought of as an element in $W_1^\ast$ by taking $v_1\otimes...\otimes v_n$ to $l(v_1)l(v_2)...l(v_n)$. Similarly, embedding $V_2$ into $W_2 = V_1^{\otimes N-1}\otimes V_2\otimes V_2$ allows us to think of non-zero elements in $L_{1,i}$ as elements of $W_2^\ast$. Then this maps $\hat{Y}$ to a subvariety $\tilde{Y}$ in $\bigotimes_{i=1}^n \PP(W_1\otimes W_2)$, and the map collapses the unwanted $\PP_1$s while being an embedding otherwise. 
	
	Finally, we let $\cP = \tilde{Y}\sslash SL(N,\C)$ be the geometric quotient, and we call it the \emph{moduli space of framed parabolic bundles}.
	
	\section{Framed Parabolic Bundles on a Trinion}
	To understand the moduli space of framed parabolic bundles on our entire Riemann surface, it is insightful to mirror the construction of section 3 and decompose the Riemann surface into trinions. Towards this end, let us compute $\cP$ for one trinion $D$, a copy of $\PP^1$ with three marked points. Without loss of generality, let the three marked points be $z=0,1,\infty$. 
	
	\begin{lemma}
		A degree-0 framed parabolic sheaf $(E,\alpha)$ on $D$ which is semi-stable must fall into one of four cases:
		\begin{enumerate}
			\item $E$ is trivial; $E\cong \OO\oplus \OO$.
			\item $E$ is torsion-free but not trivial, and $E\cong \OO(1)\oplus\OO(-1)$.
			\item $E$ has torsion at one point $p$, and $E\cong \OO\oplus\OO(-1)\oplus \C_p$.
			\item $E$ has torsion at two points $p_1,p_2$, and $E\cong \OO(-1)\oplus\OO(-1)\oplus\C_{p_1}\oplus \C_{p_2}$.
		\end{enumerate}
	\end{lemma}
	\begin{proof}
		Suppose that $E$ has no torsion. Then $E \cong \OO(j)\oplus \OO(-j)$, $j\in\mathbb{Z}$. If $E$ is semi-stable with respect to some weights $\gamma$, the semi-stability condition is that for all subbundles $F$ of $E$,
		\begin{equation}
			2\deg(F) \leq rk(F)\left(
			0-\sum_{i=1}^n \gamma_i
			\right) + 2\sum_{i=1}^n(1-\mu_i(F) + \sigma(F))\gamma_i,
		\end{equation}
		where we recall that $\sigma(F) = \frac{1}{rk(E)}$ if $F=E$ and $0$ otherwise, and $\mu_i(F) = 1$ is $F_{p_i} \subset \ker \alpha_i$ and $0$ otherwise. Consider $F = \OO(j)$. The equation becomes
		\begin{equation}
			2j \leq \sum_{i=1}^n \gamma_i - 2\mu_i(F).
		\end{equation}
		Since $\gamma_i \leq 1$, This can never be satisfied for $j \geq 2$, and therefore $E = \OO\oplus\OO$ or $E=\OO(1)\oplus\OO(-1)$ are the only choices which can yield a semi-stable parabolic sheaf. 
		
		When $E$ has torsion at one of the marked points, say $p$, then let $E' = E/\text{Tor}(E)$.
		which is a bundle with $E\cong \OO(j-1)\oplus\OO(-j)$. Let $F=\OO(j-1)\oplus \C_p$. Then the stability condition for $E$ and $F$ is
		\begin{equation}
			2j \leq \sum_{i=1}^3 \gamma_i -2\mu_i(F)
		\end{equation}
		The right-hand side is at most 3, and therefore if $j\geq 2$ then the sheaf $\OO(j-1)\oplus \C_p$ is destabilising. Therefore we must have $E' \cong \OO(-1)\oplus \OO$, and $E \cong \OO(-1)\oplus \OO \oplus \C_p$. 
		
		If $E$ has two torsion points, then $E'\cong \OO(-j)\oplus\OO(j-2)$ If $j\geq 2$, then $\OO(j-2)\oplus \C_{p_1}\oplus \C_{p_2}$ will be destabilizing. If $j=0$, then $F=\OO\oplus\C_{p_1}\oplus \C_{p_2}$ has stability condition
		\begin{equation}
			4 \leq \sum_{i=1}^3 \gamma_i - 2\mu_i(F).
		\end{equation}
		The left hand side is at most 3, so $F$ is destabilizing. So $E' \cong \OO(-1)\oplus \OO(-1)$ is the only semistable choice, and $E \cong \OO(-1)\oplus \OO(-1)\oplus \C_{p_1} \oplus \C_{p_2}$. 
		
		Finally, if $E$ has torsion at all three marked points, $E' = \OO(-j)\oplus \OO(j-3)$. Then from lemma \ref{l:ss-lemma}, $\gamma_i = 1$ for all $i$, $\mu_i = 0$ for all $i$.  Therefore the stability condition for the subsheaf $F=\OO(j-3)\oplus\C_{0}\oplus \C_{1}\oplus \C_{\infty}$ becomes
		\begin{equation}
			2j \leq -3 + 2(3) = 3
		\end{equation}
		This rules out $j\geq 2$. For $j=0,1$ the subsheaf $\OO(-j)\oplus\text{Tor}(E)$ will be destabilizing; the stability condition is
		\begin{equation}
			2(3-j) \leq 3.
		\end{equation}
		Hence there are no semi-stable sheaves $E$ with torsion at three points.
	\end{proof}
	The lemma gives us the following corollary.
	\begin{theorem}
		For all degree-0 semi-stable parabolic sheaves $(E,\alpha)$ over $D$, $E\otimes \OO(1)$ is generated by four global sections. In particular, there is an exact sequence:
		\begin{equation}
		\OO(-2)\oplus \OO(-2) \xrightarrow{Az+Bz^{-1}} \OO(-1)^{\oplus 4} \xrightarrow{} E \to 0.
		\end{equation}
		With $A,B$ represented by 4x2 matrices. The map $\OO(-2)\oplus \OO(-2) \xrightarrow{Az+Bz^{-1}} \OO(-1)^{\oplus 4}$ is injective as a map of bundles away from the torsion points of $E$.
	\end{theorem}
	\begin{proof}
		From the lemma, $E\otimes \OO(1)$ falls into one of four cases, each of which is a subsheaf of $\OO^{\oplus 4}$. Each case gives us an exact sequence:
		\begin{enumerate}
			\item $\OO\oplus \OO\to \OO^{\oplus 4}\xrightarrow{} E\otimes\OO(1) \to 0$
			\item $\OO\oplus\OO(-1) \to \OO^{\oplus 4}\xrightarrow{} E\otimes\OO(1)  \to 0$
			\item $\OO\oplus\OO(-1) \to \OO^{\oplus 4}\xrightarrow{} E\otimes\OO(1) \to 0$
			\item $\OO(-1)\oplus (-1)\to \OO^{\oplus 4}\xrightarrow{} E\otimes\OO(1) \to 0$
		\end{enumerate}
		However since $\OO \hookrightarrow \OO(-1)$, we always have the fourth sequence. Tensoring it with $\OO(-1)$ we obtain the sequence in the theorem.
		
		Since $E$ is rank 2 away from torsion, and $E = \OO(-1)^{\oplus 4}/\text{im}(Az+Bz^{-1})$ by exactness, at every point $p$ without torsion $\text{im}(Az+Bz^{-1})|_p = \text{im}(Ap+Bp^{-1})$ has dimension 2. The domain $\OO(-2)\oplus\OO(-2)|_p$ has dimension 2, and so by rank-nullity theorem the kernal of $Ap+Bp^{-1}$ has dimension $2-2=0$ and is hence injective. 
	\end{proof}
	The 4x2 matrices $A,B$ depend on $E$ and in fact determine $E$. The parabolic structure $\alpha = (\alpha_0,\alpha_1,\alpha_\infty)$ is realized as maps $\OO(-1)^{\oplus 4}\to \C_{p_i}$ for which $\text{im}\left(\OO(-2)\oplus\OO(-2)\right) \subset \ker\alpha_{p_i}$. These maps are represented by row vectors $V_0,V_1,V_\infty$ in $\C^4$ satisfying the conditions:
	\begin{align}
		V_0 A &=0, & V_1(A+B) &= 0, & V_\infty B &=0.
	\end{align}
	
	Finally, the framing ($SL(2,\C)$ structure) is determined by the isomorphism
	\begin{equation}
		\wedge^2 (E) \cong \left(\wedge^2 \OO(-2)\oplus \OO(-2)\right)^\ast\otimes \wedge^4\OO(-1)^{\oplus 4},
	\end{equation}
	so the data of $A,B, V_0,V_1$ and $V_\infty$ determines a framed parabolic bundle.
	
	There is an action of $GL(2,\C)\times GL(4,\C)$ on this data, by
	\begin{equation}
		\label{e:p3-conds}
		(g,G)\cdot (A,B,V_0,V_1,V_\infty) = (GAg^{-1}, GBg^{-1}, V_0G^{-1}, V_1G^{-1}, V_\infty G^{-1}),
	\end{equation}
	and the orbits of this action are the isomorphism classes of framed parabolic bundles we are interested in. We define four functions which are projectively invariant under this action:
	\begin{align}
		\label{e:p3-coords}
		x&=\det(A,B),& y&=\det\begin{pmatrix}
		V_0 A\\
		V_1 B\\
		\end{pmatrix}, & z&= \det\begin{pmatrix}
		V_0 B\\
		V_\infty A\\
		\end{pmatrix}, & w&= \det\begin{pmatrix}
		V_1 A\\
		V_\infty B\\
		\end{pmatrix}.
	\end{align}
	The action of $(g,G)$ on these functions pulls out the determinants of $g$ and $G$, so we can instead consider just the orbits of equivalence classes of $(x,y,z,w) \mod \C^\ast$ under the action of $SL(2,\C)\times SL(4,\C)$. 
	\begin{theorem}
		\label{t:p3-iso}
		The map $\Phi:\cP|_D$ taking a framed parabolic sheaf $(E,\alpha)$ to the class $[x:y:z:w]$ as defined in equations \ref{e:p3-coords} is an isomorphism of $\cP|_D$ with $\PP^3$. 
	\end{theorem}
	\begin{lemma}
		A point in $\cP|_D$ defined by $(A,B,V_0,V_1,V_\infty)$ is semi-stable if and only if one of the co-ordinates $x,y,z$ or $w$ is non-zero. This tells us $\Phi$ is well defined.
	\end{lemma}
	\begin{proof}
		In this proof we will use another characterization of stability, which is that $E$ is semi-stable if the closure of $\text{Orb}_G(E)$ does not contain zero, and $E$ is stable if the stabilizer is finite. 
		
		If $(A,B,V_0,V_1,V_\infty)$ defines an unstable bundle, we want to show that all the co-ordinates vanish. Suppose the closure of $(A,B,V_0,V_1,V_\infty)'s$ orbit under $SL(2,\C)\times SL(4,\C)$ contains $(0,0,0,0,0)$. First suppose the orbit itself contains $0$. Then there is $G\in SL(4,\C)$ such that $V_i G^{-1} =0$, and since $G$ is invertible this means $V_i = 0$ for all $V_0,V_1,V\infty$; this makes $y,z,w$ all 0. Furthermore, if $g\in SL(2,\C)$ such that $GAg^{-1} = GBg^{-1} =0$ then $A=B=0$ by invertibility of $G$ and $g$. Hence $x=0$; so if the orbit contains $0$ then $x=y=z=w=0$. 
		
		Now suppose that $0$ is in the closure of the orbit. This means there is a sequence of matrices $(G_i,g_i)_{i=1}^\infty$ which approach matrices which send $(A,B,V_0,V_1,V_\infty)$ to $0$. By the previous argument, either $G$ or $g$ is not invertible, or all the co-ordinates $x,y,z,w$ equal 0. Since the determinant is a continuous function on $SL(n,\C)$, the limit of the determinant is the determinant of the limit, and since $G_i$ and $g_i$ all have determinant $1$, so does their limit. They must be invertible, and hence $x=y=z=w=0$.\vspace{1em}
		
		
		Conversely, consider a point whose co-ordinates all vanish. This tells us that there is a basis of $\C^2$ for which $V_i A = (a_i,0)$ and $V_i B = (b_i ,0)$ for $i=0,1,\infty$. Pick a basis for $\C^4$ in which each $V_i$ has the form $(\ast,\ast,\ast,0)$. Then let $n>0$ and $G(z) = \text{diag}(z^{-n}, z^{-n}, z^{-n}, z^{3n})$. $G(z)$ is a one parameter family in $SL(4,\C)$ for which $G^{-1}(z)V_i \to 0$ as $z\to 0$. Let $g(z) = \text{diag}(z^{-m}, z^{m})$ for some $n < m < 3m$. Then
		\begin{equation}
			GAg^{-1}(z) =
			\begin{pmatrix}
			z^{m-n} A_{11} & z^{-m-n}A_{12}\\
			z^{m-n} A_{21} & z^{-m-n}A_{22}\\
			z^{m-n} A_{31} & z^{-m-n}A_{32}\\
			z^{m+3n} A_{41} & z^{3n-m}A_{42}\\
			\end{pmatrix} 
		\end{equation}
		If the span of the $V_i$ is 3-dimensional, then in our basis we must have $A_{12}=A_{22}=A_{32}=0$ from equation \ref{e:p3-conds}. In this case, as $z\to 0$ we have $GAg^{-1}(z) \to 0$, so this family $(G,g)(z)$ is destabilizing. If the span is 2-dimensional, $A_{32}$ may be non-zero. In this case, we modify $G(z) = \text{diag}(z^{-n}, z^{-n}, z^n, z^n)$ and get a destablizing family. Similarly, if the span of the $V_i$ is 1-dimensional, then $A_{12}=0$ and we let $G(z) = \text{diag}(z^{-3n}, z^n, z^n, z^n)$ to get a destabilizing family. All this holds for $B$ as well.
		
		In the last case with all $V_i=0$, we need only consider $A$ and $B$. Since $\det(A,B) = 0$, there is a basis in which
		\begin{equation}
			(A,B) = 
			\begin{pmatrix}
		A_{11} & A_{12}	& B_{11} & B_{12}\\
		A_{21} & A_{22} & B_{21} & B_{22}\\
		A_{31} & A_{32} & B_{31} & B_{32}\\
		A_{41} & A_{42} & B_{41} & B_{42}\\
			\end{pmatrix}
		\end{equation}
		has a row of zeros. Suppose without loss of generality it is the last row. Then let $0 < m < n$ and let $G(z) = \text{diag}(z^{n}, z^{n}, z^{n}, z^{-3n})$ and $g(z) = \text{diag}(z^{-m}, z^{m})$. The family $(G,g)(z)$ will destabilize $(A,B,0,0,0)$. 
	
	\end{proof}
	Now we prove theorem \ref{t:p3-iso}.
	\begin{proof}
		If we fix one of $x,y,z$ or $w$ to equal 1, then we are left with an open affine in $\PP^3$ isomorphic to $\C^3$. From the lemma we know $\Phi$ patches together correctly on the intersections of these affines, so if we can prove it is bijective on each affine then it will be bijective everywhere. 
	\end{proof}
	Our moduli space is therefore $\cP|_D \cong \PP^3$ over the trinion $D$. We also have an action of $(\C^\ast)^3$ on $\cP|_D$ where each factor scales the parabolic structure at one point; for $\lambda_i\in \C^\ast$,
	\begin{equation}
		(\lambda_1,\lambda_2,\lambda_3)\cdot (A,B,V_0,V_1,V_\infty) = (A,B,\lambda_1 V_0, \lambda_2 V_1, \lambda_3 V_\infty).
	\end{equation}
	In terms of our co-ordinates for $\PP^3$, the action is given by
	\begin{equation}
		(\lambda_1, \lambda_2,\lambda_3)\cdot [x:y:z:w] = [x:\lambda_1\lambda_2 y: \lambda_1\lambda_3 z: \lambda_2\lambda_3 w].
	\end{equation}
	Consider the line bundle $\OO(1)$ of homogeneous linear functions over $\PP^3$. The action on $\PP^3$ linearizes on $\OO(1)$; if $ax+by+cz+dw \in \OO(1)(\PP^3)$ then define
	\begin{equation}
	(\lambda_1, \lambda_2,\lambda_3) \cdot (ax+by+cz+dw) = ax +b\lambda_1\lambda_2 y + c\lambda_1\lambda_3 z + d\lambda_2\lambda_3 w. 
	\end{equation}
	
	\section{Gluing $\cP$ over Trinions }
	Given two Riemann surfaces $\Sigma_1$, $\Sigma_2$ with marked points $p_1$ and $p_2$, we can glue them together by identifying $p_1$ and $p_2$ to form $\Sigma = \Sigma_1 \coprod \Sigma_2 / (p_1 \sim p_2)$. This will be a nodal complex curve. If $\cP_1$ and $\cP_2$ denote the moduli space of parabolic sheaves corresponding to $\Sigma_1$ and $\Sigma_2$, we can also consider gluing them together. For framed parabolic sheaves $(E_i,\alpha_i)$, with $E_i$ over $\Sigma_i$ and $\alpha_i:(E_i)|_{p_i}\to \C$, we can combine the two into a diagram:
	\begin{equation}
		(E_1)|_{p_1} \to \C \rightarrow (E_2)|_{p_2}.
	\end{equation}
	We consider two such diagrams to be equivalent if the framed parabolic bundles on each surface are isomorphic, and if the induced maps $(E_1)|_{p_1}/\ker(\alpha_1) \to (E_1)_{p_1}/\ker(\alpha_1)$ are the same. Then taking equivalence classes under this equivalence amounts to quotienting out the anti-diagonal action of $\C^\ast$ on the framing at $p_0$ and $p_1$, and we have $\cP_{\Sigma} = \cP_0 \times \cP_1 \sslash \C^\ast$. 
	
	In particular, given a compact Riemann surface $\Sigma$ with trinion decomposition $\{D_\gamma\}_{\gamma=1}^{2g-2}$, we can pinch the boundary circles ${C_i}_{i=1}^{3g-3}$ down to points to obtain a singular surface $\tilde{\Sigma}$, consisting of $2g-2$ copies of $\PP^1$ with three marked points glued along those $3g-3$ total marked points. Then we can obtain the moduli space $\cP$ for $\tilde{\Sigma}$ by gluing the moduli spaces $\cP_\gamma$ for each trinion $D_\gamma$. From the previous section we know each $\cP_\gamma$ is isomorphic to $\PP^3$. Heuristically, this lets us estimate the dimension of $\cP$ as follows. We have a total of $3(2g-2)$ dimensions for the $\PP^3$ over each trinion, and we have to quotient an action of $\C^\ast$ along $3g-3$ curves. Therefore, we expect that $\cP$ has dimension $3(2g-2)-3g-3 = 3g-3$, which agrees with our dimension calculation for the moduli space $\MM$ from section 2 and 3. \vspace{1em}
	
	Given two trinions $D_1$, $D_2$ with moduli spaces $\cP_1 \cong \cP_2 \cong \PP^3$, each of which has their own twisting sheaf $\OO(1)$, we can ask if we get a twisting sheaf exhibiting $\cP_1 \times \cP_2 \sslash \C^\ast$ as a quasi-projective scheme. The Segre embedding lets us embed $\PP^3 \times \PP^3$ into $\PP^{15}$, which has its own twisting sheaf $\OO(1)_{\PP^{15}}$. To recall, if $[x:y:z:w]$ and $[x':y':z':w']$ are co-ordinates for each copy of $\PP^3$, then the Segre embedding is the map
	\begin{equation}
		([x:y:z:w], [x':y':z':w']) \to [xx':xy':xz':xw':yx':...:ww'],
	\end{equation}
	where the right-hand side is ordered lexicographically. Let $X$ denote the image of $\PP^3\times \PP^3$ in $\PP^{15}$ and let $p:X\to \PP^3$ denote the projection onto one factor. The Segre embedding has the property that (cite)
	\begin{equation}
		\OO(1)_{\PP^{15}}|_X = p^\ast \OO(1) \otimes p^\ast \OO(1).
	\end{equation}
	To do just one case, suppose the gluing of $D_1$ and $D_2$ is at the point $0$ in both trinions, so that the antidiagonal $\C^\ast$ action is
	\begin{align*}
		\lambda \cdot [x:y:z:w] &\to [x:\lambda y:\lambda z: w]\\
		\lambda \cdot [x':y':z':w']&\to [x:\frac{1}{\lambda}y': \frac{1}{\lambda}z':w'],
	\end{align*}
	then this gives a corresponding action on $\OO(1)_{\PP^{15}}$, whose sections are linear combinations of the co-ordinate functions on $X$. The GIT quotient is then defined as
	\begin{equation}
		X\sslash \C^\ast := \text{Proj}\left(\bigg(\bigoplus_{n\geq 0} \Gamma(X, \left(\OO(1)_{\PP^{15}}|_X\right)^n)\bigg)^{\C^\ast}\right).
	\end{equation}
	The twisting sheaf of $X\sslash \C^\ast$ is exactly the degree-0 component of this ring, with sections $\Gamma(X, \left(\OO(1)_{\PP^{15}}|_X\right)^{\C^\ast}.$ Looking at the antidiagonal action when both points are $0$, we see the invariant sections of $\OO(1)_{\PP^{15}}|_X$ are those with no $y,z,y'$ or $z'$ components, i.e:
	\begin{equation}
		\Gamma(X, \left(\OO(1)_{\PP^{15}}|_X\right))^{\C^\ast} = \text{span}(xx',xw',wx',ww').
	\end{equation}
	
	In the full-trinion decomposition setting, repeating this process at each point gives us the moduli space $\cP$ with a very ample line bundle $\LL$ embedding $\cP$ into a large projective space. Any Fubini-Study metric on that projective space gives $\LL$ a Chern connection $\theta \in \Omega^{1,1}(T^\ast \cP)$ such that $i\theta$ is a metric for $\cP$. The complex structure and metric also yield a symplectic form $\omega$ for $\cP$. Later, we will want to check that when a point in $\cP$ is a stable point in $\MM$, this symplectic form agrees with the Atiyah-Bott form on $\MM$.
