\documentclass[]{article}

%opening
\title{no title}
\date{\today}
\usepackage{amssymb}
\usepackage{amsmath}
\usepackage{tikz-cd}
\usepackage{quiver}
\usepackage{amsthm}

\newtheorem{theorem}{Theorem}
\newtheorem{definition}{Definition}

\newcommand{\C}{\mathbb{C}}
\newcommand{\Hom}{\text{Hom}}
\newcommand{\Ann}{\text{Ann}}
\newcommand{\OO}{\mathcal{O}}
\newcommand{\LL}{\mathcal{L}}
\newcommand{\MM}{\mathcal{M}}
\newcommand{\End}{\text{End }}
\newcommand{\coker}{\text{coker}~}
\newcommand{\dbar}{\overline{\partial}}
\newcommand{\cA}{\mathcal{A}}
\newcommand{\cG}{\mathcal{G}}
\newcommand{\Tr}{\text{Tr }}
\newcommand{\HH}{\mathbb{H}}
\newcommand{\sslash}{\mathbin{/\mkern-4mu/}}

\begin{document}
	Following section 4 of (Hurtubise  and Jeffrey), we describe the moduli space of framed parabolic vector bundles.
\begin{definition}
	A \emph{parabolic bundle} over a complex manifold $\Sigma$ is a holomorphic vector bundle $E$ over $\Sigma$ with a \emph{parabolic structure}, which is a point of marked points $\{p_1,...,p_n\}$ and for each point, a flag of subspaces in the fibre $E_{p_k}$. 
\end{definition}
In particular if $E$ has rank 2, then a parabolic structure on $E$ is a choice of points $\{p_k\}$ and a sheaf homomorphism $\alpha:E\to S$ where $S := \bigoplus_{k} \C_{p_k}$.

There is an adapted notion of stability for parabolic vector bundles.
\begin{definition}
	Let $\gamma_1,...,\gamma_n \in [0,1]$ be a set of weights. For a subbundle (not neccesarily proper) $F$ of $E$ we set $\mu_i(F) = 1$ if $F_{p_i} \subset \ker\alpha_i$, and $\mu_i = 0$ otherwise. Define $\sigma(F) = \frac{1}{rk(E)}$ if $F=E$ and $0$ otherwise. Then we say a pair $(E,\alpha)$ is \emph{stable} with respect to $\gamma$ if 
	\begin{equation}
	rk(E)\deg(F) < rk(F)\left(\deg(E) - \sum_{i=1}^n \gamma_i\right) 
	+ rk(E)\sum_{i=1}^n(1 - \mu_i(F) + \sigma_i(F))\gamma_i.
	\end{equation}
	If the inequality is not strict, $(E,\alpha)$ is \emph{semi-stable}.
\end{definition}

\end{document}
