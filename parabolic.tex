\documentclass[]{article}

%opening
\title{no title}
\date{\today}
\usepackage{amssymb}
\usepackage{amsmath}
\usepackage{tikz-cd}
\usepackage{quiver}
\usepackage{amsthm}

\newtheorem{theorem}{Theorem}
\newtheorem{definition}{Definition}
\newtheorem{lemma}{Lemma}

\newcommand{\C}{\mathbb{C}}
\newcommand{\Hom}{\text{Hom}}
\newcommand{\Ann}{\text{Ann}}
\newcommand{\OO}{\mathcal{O}}
\newcommand{\LL}{\mathcal{L}}
\newcommand{\MM}{\mathcal{M}}
\newcommand{\End}{\text{End }}
\newcommand{\coker}{\text{coker}~}
\newcommand{\dbar}{\overline{\partial}}
\newcommand{\cA}{\mathcal{A}}
\newcommand{\cG}{\mathcal{G}}
\newcommand{\Tr}{\text{Tr }}
\newcommand{\HH}{\mathbb{H}}
\newcommand{\sslash}{\mathbin{/\mkern-4mu/}}

\begin{document}
	In order to fit in the non-smooth fibres, with holonomy angles $0$ or $\pi$, we will consider a different perspective on the problem. Let us fix the holonomy around the boundary loops of a trinion, which we think of as a thrice-punctured sphere, and consider the space of connections with this fixed holonomy. Due to Mehta and Seshadri (cite) there is a correspondence between the moduli of $\pi_1(\Sigma)$ representations into $SU(2)$ and that of rank-2 holomorphic bundles with an $SL(2,\C)$ structure and a parabolic structure at the punctures of $\Sigma$, which we call a parabolic bundle. Under this correspondence, the eigenvalues of the holonomy get translated into a set of weights for the parabolic structure. 
	
	We want to consider the moduli space of connections with all possible holonomies, and therefore we will want to fit all these moduli of parabolic bundles together, and in such a way that we can even include the $\theta_i = 0,\pi$ cases, which will correspond to weights $0$ and $1$. 
\begin{definition}
	A \emph{parabolic bundle} over a complex manifold $\Sigma$ is a holomorphic vector bundle $E$ over $\Sigma$ with a \emph{parabolic structure}, which is a point of marked points $\{p_1,...,p_n\}$ and for each point, a flag of subspaces in the fibre $E_{p_k}$. 
\end{definition}
In particular if $E$ has rank 2, then a parabolic structure on $E$ is a choice of points $\{p_k\}$ and a sheaf homomorphism $\alpha:E\to S$ where $S := \bigoplus_{k} \C_{p_k}$.

There is an adapted notion of stability for parabolic vector bundles.
\begin{definition}
	Let $\gamma_1,...,\gamma_n \in [0,1]$ be a set of weights. For a subbundle (not neccesarily proper) $F$ of $E$ we set $\mu_i(F) = 1$ if $F_{p_i} \subset \ker\alpha_i$, and $\mu_i = 0$ otherwise. Define $\sigma(F) = \frac{1}{rk(E)}$ if $F=E$ and $0$ otherwise. Then we say a pair $(E,\alpha)$ is \emph{stable} with respect to $\gamma$ if 
	\begin{equation}
	rk(E)\deg(F) < rk(F)\left(\deg(E) - \sum_{i=1}^n \gamma_i\right) 
	+ rk(E)\sum_{i=1}^n(1 - \mu_i(F) + \sigma_i(F))\gamma_i.
	\end{equation}
	If the inequality is not strict, $(E,\alpha)$ is \emph{semi-stable}.
\end{definition}
This lemma of Hurtubise and Jeffrey (cite) summarizes some important results about tweighted parabolic bundles.
\begin{lemma}
	If $(E,\alpha)$ is a parabolic bundle semi-stable with respect to weights $\gamma$, then:
	\begin{enumerate}
		\item The kernal of $\alpha$ is torsion free, and the torsion subsheaf of $E$ is non-zero only at the $p_i$, equalling $0$ or $\C$ at each $p_i$.
		\item If $\gamma_i >0$, then $\alpha_i \neq 0$.
		\item If $\gamma_i < 1$, then $E$ is torsion free at $p_i$.
		\item If $\gamma_i \in (0,1)$, one has a parabolic structure at $p_i$, and if all the weights are in $(0,1)$, the stability condition is identical to that of parabolic bundles with weights $(1-\gamma_i)/2$ and $(1+\gamma_i)/2$.
		\item If $(E,\alpha)$ is locally free at $p_i$ and $\alpha \neq 0$, then for $\gamma_i = 0$, there is a family $(E_t, \alpha_t)$, $t\in \C$, of semi-stable pairs such that $(E_t,\alpha_t)\cong (E,\alpha)$ for $t\neq 0$, and $\alpha_0 = 0$. 
		\item If $(E,\alpha)$ is locally free at $p_i$ and $\alpha \neq 0$, then for $\gamma_i = 1$, there is a family $(E_t,\alpha_t)$, $t\in \C$, such that $(E_t, \alpha_t)\cong (E,\alpha)$, $t\neq 0$ and $E_0$ has torsion at $p_i$.
	\end{enumerate}
\end{lemma}
\begin{proof}
	Hurtubise and Jeffrey (cite) lemma 4.3
\end{proof}
In particular, this lemma tells us that when $\gamma_i\in (0,1)$, which will correspond to connections with holonomy in $(0,\pi)$, we get well-behaved parabolic structures without torsion. It also tells us there are two edge cases to consider. When $\gamma_i = 0$, the parabolic structure vanishes, and when $\gamma_i = 1$ we acquire torsion. Each of these cases will need to be dealt with to obtain a suitable moduli space.
\end{document}
