
	The degeneration of Section \ref{s:degeneration} allows us to relate the moduli space $\MM$ to the toric variety $\cP$, with a surjection $\phi:\MM\to \cP$ that is compatible with the symplectic structure on both spaces. Furthermore, the toric variety is projective and has a very ample line bundle $\LL_0$ whose sections are computed by counting lattice points in $\cP$'s polytope. In this section, we want to investigate the space of holomorphic sections $H^0(\cP,\LL_0)$ and we want to define a polarization of $\cP$ which will allow us to discuss the cohomology of fibre-wise flat sections, $\oplus_{k\geq 0}H^k(\cP, \JJ_{\pi,0})$. In particular, we ask if the degeneration from $\MM$ to $\cP$ preserves these spaces or not.
	 
\section{Polarization and Prequantum Line Bundle on $\cP$}
	From Section \ref{s:mastermoduli}, $\cP$ has a symplectic form $\omega$, and the degeneration $\phi:\MM\to \cP$ is symplectomorphic on an open dense subset $U$ of $\MM$. Thus, it remains to equip $\cP$ with a real polarisation $\pi_{\cP}$ and a line bundle $\LL_0$ with curvature $2\pi i \omega$. In both cases, we mirror the construction of the corresponding object for $\MM$.
	
	For $\MM$ we defined functions $f_i:\MM\to \mathbb{R}$ by taking the cosines of the holonomies of a connection around each loop in a given trinion decomposition of the surface $\Sigma$. Now for $\cP$, we have $\Sigma_0$ with $n=3g-3$-punctures and the extended moduli space $\MM^G$; given an element $x = (C_1,(D_i,C_i)_{i=2}^n)\in \MM^G$, it is the matrices $\{C_j\}$ which correspond to these holonomies. Taking the implosion, we obtained elements $W_j \in D(G)_{impl}$ with $W_j = (D_k, C_k)$, and so we want to define functions $f_k:\MM^G \to \mathbb{R}$ by 
	\begin{equation}
		\tilde{f}_k(C_1,(D_i,C_i)_{i=2}^n) = \frac{1}{\pi}\cos^{-1} \left(\Tr C_k\right)
	\end{equation}
	To get functions on $P$, one must check that these $\tilde{f}_k$ pass to the symplectic implosion. For the action of $G_k$ on $\MM^G$, with moment map $\Phi_k(C_1,(D_i,C_i)_{i=2}^n) = (C_k)^{-1}$, and a given face $\sigma$ of $\Delta$ we have
	\begin{equation}
		\Phi_k^{-1}(\sigma_0) = \{
		(C_1,(D_i,C_i)_{i=2}^n) \in \MM^G ~|~ (C_k)^{-1} \in \sigma
		\}
	\end{equation}
	Hence $\tilde{f}_k$ is constant on on $\Phi_k^{-1}(\sigma)$. To perform the quotient of $[G_\sigma, G_\sigma]$, we have two cases: if $\sigma = \Delta^0$ then $[G_\sigma, G_\sigma] = \{e\}$ and so $\Phi_k$ passes to the quotient. If $\sigma \in \{0,1\}$, then $[G_\sigma, G_\sigma] = SU(2)$ and we have 
	\begin{equation}
		\tilde{f}_k(g\cdot \left(C_1,(D_i,C_i)_{i=2}^n)\right) = \frac{1}{\pi}\cos^{-1}\Tr(\Ad_{g} (C_k)) = \tilde{f}_k(C_1,(D_i,C_i)_{i=2}^n).
	\end{equation}
	Hence $\tilde{f}_k$ is invariant and passes to the quotient by $[G_\sigma,G_\sigma]$. For $l \neq k$, the value of $\tilde{f}_l$ depends only on $C_l$, which is not acted on by $G_k$ and therefore $\tilde{f}_l$ also passes to the quotient. It remains to check that these functions pass under the quotient by the first copy of $G$. Recall that after quotienting, we obtain the elements $(W_1,W_2,...,W_n)$, where $W_i=(D_i,C_i)\in D(G)_{impl}$ ($D_1 =\mathds{1}$). The action of the first $G$ on $W_k$ is given by (\ref{e:first-action}):
	\begin{equation}
		(D_k,C_k) \to (gD_k, C_k)).
	\end{equation}
	The functions $\tilde{f}_k$ do not depend on $D_k$ and are therefore constant on the equivalence classes, and pass to the quotient to obtain functions $f_k:\cP\to \mathbb{R}$.
	
	From Equation \ref{e:degen-action} we know that $\phi:\MM\to\cP$ sends a connection with holonomy $C_k$ around a trinion decomposition curve $c_k$ to the orbit with $W_k = (D_k\mu^{-1}, \mu C_k\mu^{-1})$, $\mu^{-1}\in\Stab(C_k)$. Therefore if we define $\theta_{k,P} = \cos^{-1}(f_k)$, $\theta_{k,P}:\cP \to \mathbb{R}$, we have
	\begin{equation}
		\theta_{k,P} \circ \phi = \theta_k
	\end{equation}
	where $\theta_k:\MM\to\mathbb{R}$ were the functions defined in Section \ref{s:jeffreyweitsman}. Letting $\pi_P = (\theta_{1,P},...,\theta_{3g-3,P})$ we thus obtain a polarisation of $\cP$ with $\pi_P\circ \phi = \pi$.
	
	Next we build a prequantum line bundle on $P$. On the locus where $\phi:\MM\to P$ is a symplectomorphism, we can simply define $\LL_P = \LL$, but we must discuss the extension to the rest of $P$. Recall that we defined a function $\Theta:\cA\times\cG\to U(1)$ for connections over $\Sigma$, which we computed to be (Equation (\ref{e:cs-gauge})) 
	\begin{equation}
	\Theta(A,g) = \exp\left[-ik\int\limits_\Sigma \Tr(dg g^{-1}\wedge A)\right].
	\end{equation}
	For the singular curve $\Sigma_0$, we can define $\Theta$ exactly the same way, which allows us to define an equivalence relation on $\cA_0\times \mathbb{C}$ by $(A,z)\sim (g\cdot A, \Theta(A,g)z)$. Then we can define a line bundle $\LL_0 = \cA_0 \times \mathbb{C}/\sim$ over $\cA_0/\cG_0 = \cP$.	
	
	We want this line bundle to be compatible with the degenerate $\phi:\MM\to\cP$, in the sense that $\phi^\ast \LL_0 = \LL$.
	\begin{lemma}
		Let $V_i$ be a tubular neighbourhood of $C_i$ in $\Sigma$. Let $V_{0}$ denote the (disconnected) image of $V_i$ under the surface degeneration. Then for every pair $(A,g)\in\cA\times \cG$ on $\Sigma$ which degenerates to a pair $(A_0, g_0)$ on $\Sigma_0$, there exists liftings $(\bA,\bg)$ and $(\bA_0,\bg_0)$ such that
		\begin{equation}
			\int\limits_{V_i} \Tr(\bg^{-1}d\bg\wedge \bA) = \int\limits_{V_0} \Tr(\bg_0^{-1}d\bg_0\wedge \bA_0).
		\end{equation}
		\label{l:main-lemma}
	\end{lemma}
	\begin{proof}
		Recall that $V_i=Q_1 = \{(x,y)\in \mathbb{C}^2~|~ xy=1\}$, and it degenerates to $V_0 = Q_0 = \{(x,y)\in\mathbb{C}^2~|~xy=0\} = Q_{0,x} \cup Q_{0,y}$, where $Q_{0,x},Q_{0,y}$ are the locus with $x\neq0$ and $y\neq 0$ respectively. The curve $Q_1$ has a loop $\gamma_1$, and cutting $Q_1$ along $\gamma_1$ disconnects it into the components with $x>y$ and $y>x$. Let $Q_x$ denote where $x>y$, and $Q_y$ where $y>x$.
		
		Let $N=\{Q_t\}_{t\neq 0}\cong V_i\times(0,1]$. Then one possible lifting of $A$ from $V_i$ to $N$ is:
		\begin{equation}
			\bA = \frac{\alpha}{2}\left(\frac{dx}{x}-\frac{dy}{y} + \frac{dt}{t}\right)
		\end{equation}
		Now, since $t=xy$ we have $dt =ydx+xdy$ and therefore
		\begin{align*}
			\bA_x &= \frac{\alpha}{2}\left(\frac{dx}{x}-\frac{dy}{y} + \frac{ydx+xdy}{xy}\right)\\
			&= \alpha~ \frac{dx}{x}.
		\end{align*}
		Another possible lifting is 
		\begin{equation}
			\bA_y = \frac{\alpha}{2}\left(\frac{dx}{x}-\frac{dy}{y} -  \frac{dt}{t}\right) = -\alpha~\frac{dy}{y}.
		\end{equation}
		These liftings are equal on the curve $\gamma_1$. We define the lifting we will use, $\bA$ to be $\bA_x$ on $Q_x$ and $\bA_y$ on $Q_y$.
		
		For the gauge transformation $g(x,y)$ we can do a similar lifting process;
		\begin{equation}
			\bg(x,y,t) =\begin{cases}
			g\left(x,t/x\right), & x>y\\
			g\left(t/y,y\right), & y>x\\
			\end{cases}.
		\end{equation}
		Then the integral over $V_i$ becomes:
		\begin{align*}
			\int\limits_{V_i} \Tr(\bg^{-1}d\bg\wedge d\bA) = \int\limits_{Q_x} \Tr(g^{-1}(x,1/x)dg(x,1/x)\wedge \alpha~ \frac{dx}{x}) - \int\limits_{Q_y} \Tr(g^{-1}(1/y,y)dg(1/y,y)\wedge \alpha~ \frac{dy}{y})
		\end{align*}
		On the other hand, recall that $A_0$ is given by:
		\begin{equation}
			A_0 = \begin{cases}
			\alpha \frac{dx}{x}, & x\neq 0\\
			-\alpha \frac{dy}{y}, & y\neq 0\\
			\end{cases}
		\end{equation}
		which can be lifted to any three manifold $N'$ in a horizontal manner, i.e. by not adding any tangential part. Similarly, our gauge transformation $g_0$ coming from $g$ was defined by 
		\begin{equation}
		g_0(x,y) =\begin{cases}
		g\left(x,0\right), & x\neq 0\\
		g\left(0,y\right), & y \neq 0 \\
		\end{cases}.
		\end{equation}
		Therefore the integral over $V_0$ becomes
		\begin{align*}
		\int\limits_{V_0} \Tr(\bg_0^{-1}d\bg_0\wedge d\bA_0) = \int\limits_{Q_{x,0}} \Tr(g^{-1}(x,0)dg(x,0)\wedge \alpha~ \frac{dx}{x}) - \int\limits_{Q_{y,0}} \Tr(g^{-1}(0,y)dg(0,y)\wedge \alpha~ \frac{dy}{y}).
		\end{align*}
		Finally, all four of $Q_x,Q_y, Q_{x,0}$ and $Q_{y,0}$ are diffeomorphic to $S^2$ with two points removed, and a diffeomorphism between $Q_x$ and $Q_{x,0}$ is given by our local model for the curves. This diffeomorphism pulls back $g(x,0)$ to $g(x,x^{-1})$. The analogous equality holds for $Q_y$ and $Q_{y,0}$. Thus these two integrals are equal.
	\end{proof}
	\begin{theorem}
		Given a pair $(A,g) \in \mathcal{A}\times\mathcal{G}$ over $\Sigma$, which degenerates to the pair $(A_0, g_0)$ over $\Sigma_0$, we have
		\begin{equation}
			\Theta(A,g) = \Theta(A_0, g_0).
		\end{equation}
		In particular, this implies that if $(A,g)$ and $(B,h)$ both degenerate to $(A_0,g_0)$, then $\Theta(A,g)=\Theta(B,g)$. 
		\label{t:l=l0}
	\end{theorem}
	\begin{proof}
		From Equation (\ref{e:cs-gauge}) we know
		\begin{equation}
			\Theta(A,g) = \exp\left[-ik\int\limits_\Sigma \Tr(dg g^{-1}\wedge A)\right].
		\end{equation}
		
		Let $\{C_i\}_{i=1}^{3g-3}$ be any trinion decomposition of $\Sigma$. Let $V_i$ be a tubular neighbourhood in $\Sigma$ of each $C_i$. Then on $U:= \Sigma - \bigcup_{i=1}^{3g-3}V_i$, the degeneration does nothing, and we have $(A,g)|_U = (A_0,g_0)|_U$. Therefore, we have
		\begin{align*}
				\Theta(A,g) &= \exp\left[-ik\int\limits_\Sigma \Tr(dg~ g^{-1}\wedge A)\right]\\
				&= \exp\left[-ik\int\limits_U \Tr(dg~g^{-1}\wedge A)
				-ik \sum_{i=1}^{3g-3}\int\limits_{V_i} \Tr(dg~g^{-1}\wedge A)\right]\\
				&= \exp\left[-ik\int\limits_U \Tr(dg_0~g_0^{-1}\wedge A_0)
				-ik \sum_{i=1}^{3g-3}\int\limits_{V_{0,i}} \Tr(dg_0~g_0^{-1}\wedge A_0)\right],~ (\text{Lemma } \ref{l:main-lemma})\\
				&= \exp\left[-ik\int\limits_{\Sigma_0} \Tr(dg_0~ g_0^{-1}\wedge A_0)\right]\\
				&= \Theta(A_0,g_0).
		\end{align*}
	\end{proof}
	Using Theorem \ref{t:l=l0} we can show $\LL = \phi^\ast \LL_0$. By definition
	\begin{equation}
	\phi^\ast \LL_0 = \left\{
	(x,(y,z)) \in \MM\times\LL_0  ~|~ \phi(x) = y 		
	\right\}.
	\end{equation}
	Given a point $(x,z) \in \LL$ we can map it to $(x, (\phi(x),z))$ in $\phi^\ast \LL_0$. Since $\phi$ is surjective, this is surjective. Suppose then that for some $(x,z)$ and $(x',z') \in \LL$ we have
	\begin{equation}
	(x, (\phi(x), z)) = (x', (\phi(x'), z')).
	\end{equation}
	Then $x=x'$ and $z = z' \mod \Theta$, namely $z' = \Theta(\phi(x), g_0)z$ for some $g_0 \in \cG_0$. Then to show $z=z'$ in $\LL$, we need that there is some $g\in \cG$ such that $\Theta(A,g)z' = z$, where $A\in \cA$ represents $x\in \MM$. The surjectivity of the degeneration gives the existence of some $g$ that degenerates to $g_0$, and Theorem \ref{t:l=l0} tells us that $\Theta(A_0, g_0)z = \Theta(A,g)z = z'$ and therefore $z=z'$ in $\LL$. 
	
\section{Holomorphic Sections of $\LL$}
\begin{lemma}
	The line bundle $\LL_0$ on the toric moduli space $P(D)$ over a trinion $D$, has curvature $\omega$ equal to the Fubini-Study metric on $P(D) \cong \PP^3$.
\end{lemma}
{\tiny }\begin{proof}
	Let $\MM(D)$ denote the moduli space of flat $SU(2)$ connections over a single trinion. Then $\pi(\MM(D)) = \Delta$ and over the interior $\Delta^{0}$ we have the symplectomorphism $\phi$. Thus, on on the dense torus $(\C^\ast)^3$ inside $P(D)$, we have $\phi^\ast \omega_P = \omega_{\MM}$. Recall that the form on $\MM(D)$ is the Atiyah-Bott form (Equation (\ref{e:ab-form})).
	
	The prequantum bundle $\LL$ on $\MM(D)$ is isomorphic to the determinant line bundle (Theorem \ref{t:cs=det}) and the determinant line bundle's sections give an embedding $\iota$ of $\MM(D)$ into projective space (Proposition \ref{t:det-embed}). Therefore $\LL$ is very ample over $\MM(D)$ and $\LL = \iota^\ast \OO(1)$. The curvature $\omega_{fs}$ of $\OO(1)$ is the Fubini-Study metric, so $\iota^\ast \omega_{fs} = \omega_\MM = \phi^\ast \omega_P$ over $(\C^\ast)^3 \subset P(D)$.
	
	Therefore, on $P(D) = \PP^3$ we have two forms, $\omega_{fs}$ and $\omega_P$, which agree on an open dense subset. Therefore they are equal everywhere on $P(D)$.
\end{proof}
\begin{corollary}
	\label{t:P3-veryample}
	The line bundle $\LL_0$ over $P(D)$ is ample.
\end{corollary}
\begin{proof}
	From the lemma, the curvature of $\LL_0$ is a positive (1,1) form. Then since $P(D)$ is a compact Kahler manifold, the Kodaira embedding theorem says $\LL_0$ is ample.
\end{proof}
\begin{theorem}
	\label{t:lp-ample}
	The line bundle $\LL_0$ over $P$ is ample. 
\end{theorem}
\begin{proof}
	Let $D$ be a trinion and $P(D)$ the toric moduli space for $D$. By Corollary \ref{t:P3-veryample}, there exists large enough $n$ for which $\LL_{P(D)}^{\otimes^n}$ is very ample over $P(D)$. That is, since $P(D) = \PP^3$,
	\begin{equation}
	\OO(1) = \LL_{P(D)}^{\otimes^n}.
	\end{equation}
	From the discussion in section \ref{s:p-gluing} we know that $P$ can be constructed by gluing $\PP^3$s along a trinion decomposition of the Riemann surface $\Sigma$, and that the bundles $\OO(1)$ over each copy of $\PP^3$ glue together to give a very ample line bundle on $P$. Letting $\iota$ denote the Segre embedding (equation \ref{e:segre}) $\PP^3 \coprod \PP^3 \xrightarrow{\iota} X\subset \PP^{15}$ and $p:X\to \PP^3$ the projection, we have
	\begin{equation}
	\OO(1)_{\PP^{15}}|_X = p^\ast \OO(1)\otimes p^\ast \OO(1) = p^\ast \LL_{P(D)}^{\otimes n}\otimes p^\ast \LL_{P(D)}^{\otimes n} = \left(p^\ast \LL_{P(D)}\otimes p^\ast \LL_{P(D)}\right)^{\otimes n}.
	\end{equation}
	Let $p_1$ and $p_2$ denote the points in the copies of $\PP^3$ which are being identified. Now we consider the quotient of $\PP^3 \coprod \PP^3$ under the anti-diagonal action of $\C^\ast$ on the framing at $p_1$ and $p_2$. Away from $p_1$ and $p_2$, the quotient
	\begin{equation}
	q: \PP^3 \cup \PP^3 \to \PP^3 \cup \PP^3 / \C^\ast
	\end{equation}
	does nothing. On a (disconnected) neighbourhood $V_0$ of $p_1$ and $p_2$, which is glued to obtain the tubular neighbourhood $V$ of the identified point $p=p_1=p_2$, lemma \ref{l:main-lemma} tells us that $q^\ast \LL_{P} = p^\ast \LL_{P(D)}\otimes p^\ast \LL_{P(D)}$. Therefore we have that on the glued space $\PP^3\coprod \PP^3 / \C^\ast$, $q^\ast \LL_0^{\otimes n} = \OO(1)_{\PP^{15}}$, and hence $\LL_0$ is very ample on this glued space. Repeating this at each puncture on the degenerated Riemann surface $\Sigma$ we obtain the desired result.
\end{proof}
\begin{corollary}
	The line bundle $\LL_0$ over $P$ is very ample.
\end{corollary}
\begin{proof}
	From Theorem \ref{t:lp-ample}, there exists an embedding $\iota:P\to\PP^N$ and an integer $n>0$ such that
	\begin{equation}
	\LL^{\otimes n}_P = \iota^\ast \OO(1)_{\PP^N}.
	\end{equation}
	However, we also know that there is an open dense subset $U \subset P$ for which $\phi:\MM\to P$ is a symplectomorphism. Hence, on this subset, the curvature $\omega_P$ of $\LL_0$ agrees with that of $\LL$ on $M$ which is the Fubini-Study metric $\omega_{fs}$. Then the curvature of $\LL^{\otimes n}_P$ is $n\omega_{fs}$. However $\OO(1)_{\PP^N}$ has curvature given by the Fubini-Study metric, so we must have $n\omega_{fs} = \omega_{fs}$ and therefore $n=1$. Therefore, $\LL_0 = \iota^\ast \OO(1)_{\PP^N}$.
\end{proof}
\begin{theorem}
	The holomorphic sections $H^0(\MM,\LL) = H^0(\cP, \LL_0)$, and for all, $k>0$ $H^k(\MM,\LL) = H^k(\cP,\LL_0) = 0$.
\end{theorem}
\begin{proof}
	The bundle $\LL$ is isomorphic to the determinant bundle on $\MM$, which is very ample (Theorems \ref{t:cs=det}, \ref{t:det-embed}). As a very ample bundle on a compact projective variety, $H^k(\MM,\LL) = 0$ for $k>0$. Then since $\phi:\MM\to P$ is surjective, $\phi^\ast:H^k(\cP,\LL_0)\to H^k(\MM,\LL)$ is injective and hence $H^k(\cP,\LL_0) =0$ as well.
	
	For the global sections, we already know $\phi^\ast$ is injective, and so it remains to show that it is also surjective. Suppose then that $s\in H^0(\LL,\MM)$. Let $U \subset M$ be the locus on which $\phi$ is a symplectomorphism. The $\dbar$ operator defining holomorphic sections is the $(0,1)$ part of the connection on $\LL$, and thus over $U$ holomorphic sections on $\LL$ agree with those on $\LL_0$.  
\end{proof}

Consider one trinion $D$, over which $\MM(D) = \PP^3 = \cP(D)$. Let $[x:y:z:w]$ denote homogeneous co-ordinates. Let $x\in \MM(D)$ be an element such that $\pi(x)$ is not an interior triple. If $\pi(x)$ is a vertex of $\Delta$ then $x = [1:0:0:0], [0:1:0:0], [0:0:1:0],$ or  $[0:0:0:1]$. In this case, the $(\C^\ast)$ actions on $\PP^3$ are all trivial so $\sigma(x)$ is invariant under the degeneration.

If $\pi(x)$ is on an edge of $\Delta$ then $x = [0:1:1:0]$ e.t.c. in this case, there is one non-trivial $\C^\ast$ which $\sigma(x)$ may depend on. Suppose 


\begin{theorem}
	For the prequantum systems $(M,\pi,\LL)$ and $(P,\pi_{P}, \LL_0)$, let $\JJ_\pi ~(\JJ_{\pi,P})$ be the sheaf of sections of $\LL ~(\LL_0)$ which are covariant constant on the fibres of $\pi ~(\pi_P)$. Then
	\begin{equation}
	H^0(\JJ_\pi, \MM) = H^0(\JJ_{\pi,P}, P).
	\end{equation}
\end{theorem}
\begin{proof}
	The degeneration $\phi:\MM\to\cP$ is surjective, and thus $\phi^\ast:H^0(\JJ_{\pi,0},P) \to H^0(\JJ_\pi,\MM)$ is injective. Hence it remains to show that $\phi^\ast$ is surjective.
	\smallskip
	
	From theorem \ref{t:l=l0}, we have that $\LL=\phi^\ast \LL_P$ and $\phi^\ast \omega_P = \omega$. Suppose $s\in H^0(\JJ_{\pi},\MM)$ and suppose $x \in P$. Let $y_1,y_2 \in \phi^{-1}(x)$; it is permitted that $y_1=y_2$. Then $\pi(y_1) = \pi_P\circ \phi(x) = \pi(y_2)$ so $y_1$ and $y_2$ are in the same fibre of $\pi$. Stalkwise we have
	\begin{equation}
	\LL_{y_1} \cong \LL_{P,x} \cong \LL_{y_2},
	\end{equation} 
	so it makes sense to ask if $s(y_1) = s(y_2)$. Since $s$ is covariant constant on the fibres of $\pi$, and the fibres of $\pi$ are (path?) connected \cite[thm 2.5]{jeffrey_bohr-sommerfeld_1992}, we must have $s(y_1)=s(y_2)$, and so it is well defined to let $s'\in H^0(\JJ_{\pi,P},P)$ be given by $s'(x) = s\left(\phi^{-1}(x)\right)$. Then
	\begin{equation}
	\phi^\ast s'(y) = s'(\phi(x)) = s(y),
	\end{equation}
	so $\phi^\ast$ is surjective.
\end{proof}


\section{Counting Sections of $\JJ_\pi$}

	want to show:
	\begin{itemize}
		\item P is the toric variety corresponding to the moment polytope 
		\item the cartier divisor from the polytope gives the line bundle $J_\pi$
	\end{itemize}
	then we have
	\begin{itemize}
		\item (Fulton p.66) $\Gamma(P,\JJ_\pi)$ given by point count in the polytope
		\item (Fulton p.74) Convexity of the polytope implies higher cohomologies on $P$ are all zero
	\end{itemize}

	Here we will count the sections of $\JJ_\pi$ over $P$ by using the fact that $P$ is a toric variety \cite[Thm 3.13]{hurtubise_representations_2000}. Let $\Delta$ denote the moment polytope of $\pi:P\to\mathbb{R}^n$ and let $M$ denote the period lattice (cite JW) in $\mathbb{R}^n$. 
	
	Then $\Delta$ equips $P$ with a Cartier divisor $D$ determined by the convex function $\psi:\Delta \to \mathbb{R}^n$,
	\begin{equation}
		\psi(v) = \min_{u\in P}\langle u,v\rangle
	\end{equation}
	whose corresponding line bundle $\OO(D)$ is generated by its sections (Fulton p.73). For each point $u = (\theta_1,...,\theta_{3g-3})\in \Delta \cap M$, 
	\begin{lemma}
		The sections of $F$ are constant with respect to the moment map $\pi$.
	\end{lemma} 
	
	
	
		
	