
	 
\section{Polarization and Prequantum Line Bundle on $\cP$}
	\label{s:l0-def}
	The toric variety $\cP$ is projective and has a very ample line bundle $\LL_0$ whose sections are computed by counting lattice points in $\cP$'s polytope. In this section, we want to investigate the space of holomorphic sections $H^0(\cP,\LL_0)$ and we want to define a polarization of $\cP$ which will allow us to discuss the cohomology of fibre-wise flat sections, $\oplus_{k\geq 0}H^k(\cP, \JJ_{\pi,0})$. In particular, we ask if the degeneration from $\MM$ to $\cP$ preserves these spaces or not.
	
	From Section \ref{s:mastermoduli}, $\cP$ has a symplectic form $\omega$, and the degeneration $\phi:\MM\to \cP$ is symplectomorphic on an open dense subset $U$ of $\MM$. Thus, it remains to equip $\cP$ with a real polarisation $\pi_{\cP}$ and a line bundle $\LL_0$ with curvature $2\pi i \omega$. In both cases, we mirror the construction of the corresponding object for $\MM$.
	
	For $\MM$ we defined functions $f_i:\MM\to \mathbb{R}$ by taking the cosines of the holonomies of a connection around each loop in a given trinion decomposition of the surface $\Sigma$. Now for $\cP$, we have $\Sigma_0$ with $n=3g-3$-punctures and the extended moduli space $\MM^G$; given an element $x = (C_1,(D_i,C_i)_{i=2}^n)\in \MM^G$, it is the matrices $\{C_j\}$ which correspond to these holonomies. Taking the implosion, we obtained elements $W_j \in D(G)_{impl}$ with $W_j = (D_k, C_k)$, and so we want to define functions $f_k:\MM^G \to \mathbb{R}$ by 
	\begin{equation}
		\tilde{f}_k(C_1,(D_i,C_i)_{i=2}^n) = \frac{1}{\pi}\cos^{-1} \left(\Tr C_k\right)
	\end{equation}
	To get functions on $P$, one must check that these $\tilde{f}_k$ pass to the symplectic implosion. For the action of $G_k$ on $\MM^G$, with moment map $\Phi_k(C_1,(D_i,C_i)_{i=2}^n) = (C_k)^{-1}$, and a given face $\sigma$ of $\Delta$ we have
	\begin{equation}
		\Phi_k^{-1}(\sigma_0) = \{
		(C_1,(D_i,C_i)_{i=2}^n) \in \MM^G ~|~ (C_k)^{-1} \in \sigma
		\}
	\end{equation}
	Hence $\tilde{f}_k$ is constant on on $\Phi_k^{-1}(\sigma)$. To perform the quotient of $[G_\sigma, G_\sigma]$, we have two cases: if $\sigma = \Delta^0$ then $[G_\sigma, G_\sigma] = \{e\}$ and so $\Phi_k$ passes to the quotient. If $\sigma \in \{0,1\}$, then $[G_\sigma, G_\sigma] = SU(2)$ and we have 
	\begin{equation}
		\tilde{f}_k(g\cdot \left(C_1,(D_i,C_i)_{i=2}^n)\right) = \frac{1}{\pi}\cos^{-1}\Tr(\Ad_{g} (C_k)) = \tilde{f}_k(C_1,(D_i,C_i)_{i=2}^n).
	\end{equation}
	Hence $\tilde{f}_k$ is invariant and passes to the quotient by $[G_\sigma,G_\sigma]$. For $l \neq k$, the value of $\tilde{f}_l$ depends only on $C_l$, which is not acted on by $G_k$ and therefore $\tilde{f}_l$ also passes to the quotient. It remains to check that these functions pass under the quotient by the first copy of $G$. Recall that after quotienting, we obtain the elements $(W_1,W_2,...,W_n)$, where $W_i=(D_i,C_i)\in D(G)_{impl}$ ($D_1 =\mathds{1}$). The action of the first $G$ on $W_k$ is given by (\ref{e:first-action}):
	\begin{equation}
		(D_k,C_k) \to (gD_k, C_k)).
	\end{equation}
	The functions $\tilde{f}_k$ do not depend on $D_k$ and are therefore constant on the equivalence classes, and pass to the quotient to obtain functions $f_k:\cP\to \mathbb{R}$.
	
	From Equation \ref{e:degen-action} we know that $\phi:\MM\to\cP$ sends a connection with holonomy $C_k$ around a trinion decomposition curve $c_k$ to the orbit with $W_k = (D_k\mu^{-1}, \mu C_k\mu^{-1})$, $\mu^{-1}\in\Stab(C_k)$. Therefore if we define $\theta_{k,P} = \cos^{-1}(f_k)$, $\theta_{k,P}:\cP \to \mathbb{R}$, we have
	\begin{equation}
		\theta_{k,P} \circ \phi = \theta_k
	\end{equation}
	where $\theta_k:\MM\to\mathbb{R}$ were the functions defined in Section \ref{s:jeffreyweitsman}. Letting $\pi_P = (\theta_{1,P},...,\theta_{3g-3,P})$ we thus obtain a polarisation of $\cP$ with $\pi_P\circ \phi = \pi$.
	
	Next we build a prequantum line bundle on $P$. On the locus where $\phi:\MM\to P$ is a symplectomorphism, we can simply define $\LL_P = \LL$, but we must discuss the extension to the rest of $P$. Recall that we defined a function $\Theta:\cA\times\cG\to U(1)$ for connections over $\Sigma$, which we computed to be (Equation (\ref{e:cs-gauge})) 
	\begin{equation}
	\Theta(A,g) = \exp\left[-ik\int\limits_\Sigma \Tr(dg g^{-1}\wedge A)\right].
	\end{equation}
	For the singular curve $\Sigma_0$, we can define $\Theta$ exactly the same way, which allows us to define an equivalence relation on $\cA_0\times \mathbb{C}$ by $(A,z)\sim (g\cdot A, \Theta(A,g)z)$. Then we can define a line bundle $\LL_0 = \cA_0 \times \mathbb{C}/\sim$ over $\cA_0/\cG_0 = \cP$.	
	
	We want this line bundle to be compatible with the degeneration $\phi:\MM\to\cP$, in the sense that $\phi^\ast \LL_0 = \LL$.
	\begin{lemma}
		Let $V_i$ be a tubular neighbourhood of $C_i$ in $\Sigma$. Let $V_{0}$ denote the (disconnected) image of $V_i$ under the surface degeneration. Then for every pair $(A,g)\in\cA\times \cG$ on $\Sigma$ which degenerates to a pair $(A_0, g_0)$ on $\Sigma_0$, there exists liftings $(\bA,\bg)$ and $(\bA_0,\bg_0)$ such that
		\begin{equation}
			\int\limits_{V_i} \Tr(\bg^{-1}d\bg\wedge \bA) = \int\limits_{V_0} \Tr(\bg_0^{-1}d\bg_0\wedge \bA_0).
		\end{equation}
		\label{l:main-lemma}
	\end{lemma}
	\begin{proof}
		Recall that $V_i=Q_1 = \{(x,y)\in \mathbb{C}^2~|~ xy=1\}$, and it degenerates to $V_0 = Q_0 = \{(x,y)\in\mathbb{C}^2~|~xy=0\} = Q_{0,x} \cup Q_{0,y}$, where $Q_{0,x},Q_{0,y}$ are the locus with $x\neq0$ and $y\neq 0$ respectively. The curve $Q_1$ has a loop $\gamma_1$, and cutting $Q_1$ along $\gamma_1$ disconnects it into the components with $x>y$ and $y>x$. Let $Q_x$ denote where $x>y$, and $Q_y$ where $y>x$.
		
		Let $N=\{Q_t\}_{t\neq 0}\cong V_i\times(0,1]$. Then one possible lifting of $A$ from $V_i$ to $N$ is:
		\begin{equation}
			\bA = \frac{\alpha}{2}\left(\frac{dx}{x}-\frac{dy}{y} + \frac{dt}{t}\right)
		\end{equation}
		Now, since $t=xy$ we have $dt =ydx+xdy$ and therefore
		\begin{align*}
			\bA_x &= \frac{\alpha}{2}\left(\frac{dx}{x}-\frac{dy}{y} + \frac{ydx+xdy}{xy}\right)\\
			&= \alpha~ \frac{dx}{x}.
		\end{align*}
		Another possible lifting is 
		\begin{equation}
			\bA_y = \frac{\alpha}{2}\left(\frac{dx}{x}-\frac{dy}{y} -  \frac{dt}{t}\right) = -\alpha~\frac{dy}{y}.
		\end{equation}
		These liftings are equal on the curve $\gamma_1$. We define the lifting we will use, $\bA$ to be $\bA_x$ on $Q_x$ and $\bA_y$ on $Q_y$.
		
		For the gauge transformation $g(x,y)$ we can do a similar lifting process;
		\begin{equation}
			\bg(x,y,t) =\begin{cases}
			g\left(x,t/x\right), & x>y\\
			g\left(t/y,y\right), & y>x\\
			\end{cases}.
		\end{equation}
		Then the integral over $V_i$ becomes:
		\begin{align*}
			\int\limits_{V_i} \Tr(\bg^{-1}d\bg\wedge d\bA) = \int\limits_{Q_x} \Tr(g^{-1}(x,1/x)dg(x,1/x)\wedge \alpha~ \frac{dx}{x}) - \int\limits_{Q_y} \Tr(g^{-1}(1/y,y)dg(1/y,y)\wedge \alpha~ \frac{dy}{y})
		\end{align*}
		On the other hand, recall that $A_0$ is given by:
		\begin{equation}
			A_0 = \begin{cases}
			\alpha \frac{dx}{x}, & x\neq 0\\
			-\alpha \frac{dy}{y}, & y\neq 0\\
			\end{cases}
		\end{equation}
		which can be lifted to any three manifold $N'$ in a horizontal manner, i.e. by not adding any tangential part. Similarly, our gauge transformation $g_0$ coming from $g$ was defined by 
		\begin{equation}
		g_0(x,y) =\begin{cases}
		g\left(x,0\right), & x\neq 0\\
		g\left(0,y\right), & y \neq 0 \\
		\end{cases}.
		\end{equation}
		Therefore the integral over $V_0$ becomes
		\begin{align*}
		\int\limits_{V_0} \Tr(\bg_0^{-1}d\bg_0\wedge d\bA_0) = \int\limits_{Q_{x,0}} \Tr(g^{-1}(x,0)dg(x,0)\wedge \alpha~ \frac{dx}{x}) - \int\limits_{Q_{y,0}} \Tr(g^{-1}(0,y)dg(0,y)\wedge \alpha~ \frac{dy}{y}).
		\end{align*}
		Finally, all four of $Q_x,Q_y, Q_{x,0}$ and $Q_{y,0}$ are diffeomorphic to $S^2$ with two points removed, and a diffeomorphism between $Q_x$ and $Q_{x,0}$ is given by our local model for the curves. This diffeomorphism pulls back $g(x,0)$ to $g(x,x^{-1})$. The analogous equality holds for $Q_y$ and $Q_{y,0}$. Thus these two integrals are equal.
	\end{proof}
	\begin{theorem}
		Given a pair $(A,g) \in \mathcal{A}\times\mathcal{G}$ over $\Sigma$, which degenerates to the pair $(A_0, g_0)$ over $\Sigma_0$, we have
		\begin{equation}
			\Theta(A,g) = \Theta(A_0, g_0).
		\end{equation}
		In particular, this implies that if $(A,g)$ and $(B,h)$ both degenerate to $(A_0,g_0)$, then $\Theta(A,g)=\Theta(B,g)$. 
		\label{t:l=l0}
	\end{theorem}
	\begin{proof}
		From Equation (\ref{e:cs-gauge}) we know
		\begin{equation}
			\Theta(A,g) = \exp\left[-ik\int\limits_\Sigma \Tr(dg g^{-1}\wedge A)\right].
		\end{equation}
		
		Let $\{C_i\}_{i=1}^{3g-3}$ be any trinion decomposition of $\Sigma$. Let $V_i$ be a tubular neighbourhood in $\Sigma$ of each $C_i$. Then on $U:= \Sigma - \bigcup_{i=1}^{3g-3}V_i$, the degeneration does nothing, and we have $(A,g)|_U = (A_0,g_0)|_U$. Therefore, we have
		\begin{align*}
				\Theta(A,g) &= \exp\left[-ik\int\limits_\Sigma \Tr(dg~ g^{-1}\wedge A)\right]\\
				&= \exp\left[-ik\int\limits_U \Tr(dg~g^{-1}\wedge A)
				-ik \sum_{i=1}^{3g-3}\int\limits_{V_i} \Tr(dg~g^{-1}\wedge A)\right]\\
				&= \exp\left[-ik\int\limits_U \Tr(dg_0~g_0^{-1}\wedge A_0)
				-ik \sum_{i=1}^{3g-3}\int\limits_{V_{0,i}} \Tr(dg_0~g_0^{-1}\wedge A_0)\right],~ (\text{Lemma } \ref{l:main-lemma})\\
				&= \exp\left[-ik\int\limits_{\Sigma_0} \Tr(dg_0~ g_0^{-1}\wedge A_0)\right]\\
				&= \Theta(A_0,g_0).
		\end{align*}
	\end{proof}
	Using Theorem \ref{t:l=l0} we can show $\LL = \phi^\ast \LL_0$. By definition
	\begin{equation}
	\phi^\ast \LL_0 = \left\{
	(x,(y,z)) \in \MM\times\LL_0  ~|~ \phi(x) = y 		
	\right\}.
	\end{equation}
	Given a point $(x,z) \in \LL$ we can map it to $(x, (\phi(x),z))$ in $\phi^\ast \LL_0$. Since $\phi$ is surjective, this is surjective. Suppose then that for some $(x,z)$ and $(x',z') \in \LL$ we have
	\begin{equation}
	(x, (\phi(x), z)) = (x', (\phi(x'), z')).
	\end{equation}
	Then $x=x'$ and $z = z' \mod \Theta$, namely $z' = \Theta(\phi(x), g_0)z$ for some $g_0 \in \cG_0$. Then to show $z=z'$ in $\LL$, we need that there is some $g\in \cG$ such that $\Theta(A,g)z' = z$, where $A\in \cA$ represents $x\in \MM$. The surjectivity of the degeneration gives the existence of some $g$ that degenerates to $g_0$, and Theorem \ref{t:l=l0} tells us that $\Theta(A_0, g_0)z = \Theta(A,g)z = z'$ and therefore $z=z'$ in $\LL$. 
	
\section{Holomorphic Sections of $\LL$}
\begin{lemma}
	The line bundle $\LL_0$ on the toric moduli space $P(D)$ over a trinion $D$, has curvature $\omega$ equal to the Fubini-Study metric on $P(D) \cong \PP^3$.
\end{lemma}
{\tiny }\begin{proof}
	\label{l:metric}
	Let $\MM(D)$ denote the moduli space of flat $SU(2)$ connections over a single trinion. Then $\pi(\MM(D)) = \Delta$ and over the interior $\Delta^{0}$ we have the symplectomorphism $\phi$. Thus, on on the dense torus $(\C^\ast)^3$ inside $P(D)$, we have $\phi^\ast \omega_P = \omega_{\MM}$. Recall that the form on $\MM(D)$ is the Atiyah-Bott form (Equation (\ref{e:ab-form})).
	
	The prequantum bundle $\LL$ on $\MM(D)$ is isomorphic to the determinant line bundle (Theorem \ref{t:cs=det}) and the determinant line bundle's sections give an embedding $\iota$ of $\MM(D)$ into projective space (Proposition \ref{t:det-embed}). Therefore $\LL$ is very ample over $\MM(D)$ and $\LL = \iota^\ast \OO(1)$. The curvature $\omega_{fs}$ of $\OO(1)$ is the Fubini-Study metric, so $\iota^\ast \omega_{fs} = \omega_\MM = \phi^\ast \omega_P$ over $(\C^\ast)^3 \subset P(D)$.
	
	Therefore, on $P(D) = \PP^3$ we have two forms, $\omega_{fs}$ and $\omega_P$, which agree on an open dense subset. Therefore they are equal everywhere on $P(D)$.
\end{proof}
\begin{corollary}
	\label{t:P3-veryample}
	The line bundle $\LL_0$ over $P(D)$ is ample.
\end{corollary}
\begin{proof}
	From the lemma, the curvature of $\LL_0$ is a positive (1,1) form. Then since $P(D)$ is a compact Kahler manifold, the Kodaira embedding theorem says $\LL_0$ is ample.
\end{proof}
\begin{theorem}
	\label{t:lp-ample}
	The line bundle $\LL_0$ over $\cP$ is ample. 
\end{theorem}
\begin{proof}
	Let $D$ be a trinion and $P(D)$ the toric moduli space for $D$. By Corollary \ref{t:P3-veryample}, there exists large enough $n$ for which $\LL_{P(D)}^{\otimes^n}$ is very ample over $P(D)$. That is, since $P(D) = \PP^3$,
	\begin{equation}
	\OO(1) = \LL_{P(D)}^{\otimes^n}.
	\end{equation}
	From the discussion in section \ref{s:p-gluing} we know that $\cP$ can be constructed by gluing $\PP^3$s along a trinion decomposition of the Riemann surface $\Sigma$. The decomposition consists of $2g-2$ copies of $\PP^3$, and we must quotient by $3g-3$ actions of $\C^\ast$. Letting $\iota$ denote the Segre embedding (Equation \ref{e:segre}) $\PP^3 \coprod \PP^3 \xrightarrow{\iota} X\subset \PP^{15}$ and $p_i:X\to \PP^3$ the projection to each factor, we have
	\begin{equation}
	\OO(1)_{\PP^{15}}|_X = p_1^\ast \OO(1)\otimes p_2^\ast \OO(1) = p_1^\ast \LL_{P(D)}^{\otimes n}\otimes p_2^\ast \LL_{P(D)}^{\otimes n} = \left(p_1^\ast \LL_{P(D)}\otimes p_2^\ast \LL_{P(D)}\right)^{\otimes n}.
	\end{equation}
	Repeating this for all the copies, we can embed the product of the $\PP^3$s into $\PP^N$ for some large $N$, and we obtain an ample bundle 
	\begin{equation}
		\tilde{L} = \bigotimes_{i=1}^{3g-3} p_i^\ast \LL_{P(D)}
	\end{equation}
	Taking the GIT quotient under $\mathbb{T}:=(\C^\ast)^{3g-3}$, the result is given by the homogeneous spectrum of the invariant sections of $\tilde{\LL}$, that is
	\begin{equation}
		\cP = \text{Proj}\left(
		\bigoplus_{k\geq 0} H^0(X,\tilde{L}^k)^\mathbb{T}
		\right).
	\end{equation}
	Therefore, if we denote $q:X\to \cP$ as the quotient, then $q^\ast \OO(1)_{\cP} = \tilde{\LL}^{\otimes n}$. 
	
	On the other hand, a section of $\tilde{\LL}$ is a tensor product  $\otimes_{i=1}^{3g-3}\sigma_i$ of sections in $\LL_{P(D)}$ for each trinion $D$, and we can define a section of $\LL_0$ over $\cP$ by defining $\sigma(x) = \otimes_{i=1}^{3g-3}\sigma_i(\tilde{x_i})$, where $q(\tilde{x_i}) = x$. For most sections, this will not be well defined as the value of $\sigma_i$ may depend on the choice of $\tilde{x_i}$,  but a section will be invariant under the $\mathbb{T}$ action exactly if it does not depend on this choice: if $\tilde{x_i}$ and $\tilde{y_i}$ both map to $x$ then there is some $t\in \mathbb{T}$ with $t\cdot\tilde{x_i} = \tilde{y_i}$ and therefore $\sigma_i(x_i) = \sigma_i(y_i)$. Thus $\LL_0 \cong q^\ast\tilde{\LL}$ and $\LL^\{\otimes n\} \cong q^\ast \OO(1)_{\cP}$ meaning $\LL_0$ is ample.
	
	Finally, $\LL_0$ is very ample as by the same argument as Lemma \ref{l:metric}, its curvature is given on $\cP$ by the Fubini-Study metric, and thus the curvature of $\LL_0^n = n\omega_{fs}$. The curvature of $\OO(1)$ is the Fubini-Study metric, so for these to be equal we must have $n=1$.
\end{proof}
\begin{corollary}
	For $k\geq 0$, $H^k(\cP,\LL_0) =0$.
\end{corollary}
\begin{proof}
	$\LL_0$ is a very ample bundle on a complex projective variety, so $H^k(\MM,\LL)=0$ for $k>0$ \cite[III, Prop 2.6.1]{grothendieck_elements_1960}.
\end{proof}
The prequantum line bundle bundle $\LL$ is also very ample over a complex projective variety, so $H^k(\MM,\LL)=0$ for $k>0$ as well. Therefore, if the sections $H^0(\MM,\LL)$ and $H^0(\cP,\LL)$ are the same, then the entire cohomologies will be the same. In fact, we know from Jeffrey and Weitsman that the dimension of $H^0(\cP,\LL)$ is computed by the Verlinde formula, and it has also been shown that the dimension of $H^0(\MM,\LL)$ is computed by the Verlinde formula \cite{faltings_proof_1994}\cite{schottenloher_mathematical_2008}. This argument proves $H^0(\MM,\LL)\cong H^0(\cP,\LL_0)$, but in a non-canonical way. One may hope to find a direct isomorphism via the degeneration $\phi:\MM\to\cP$, and such an isomorphism would then provide another way to prove the Verlinde formula for $H^0(\MM,\LL)$. If the degeneration were holomorphic, then this would be doable, however it is not. The degeneration of surfaces is not holomorphic, and since the holomorphic structure on the moduli space is determined by that of the surface, $\phi$ is not holomorphic.

	