\section{Polarization and Prequantum Line Bundle on $P$}
	From section \ref{s:mastermoduli}, $P$ has a symplectic form $\omega$, and the degeneration $\phi:\MM\to P$ is symplectomorphic on an open dense subset $U$ of $\MM$. Thus, it remains to equip $P$ with a real polarisation $\pi_P$ and a line bundle $\LL_0$ with curvature $2\pi i \omega$. In both cases, we mirror the construction of the corresponding object for $\MM$.
	
	For $\MM$ we defined functions $f_i:\MM\to \mathbb{R}$ by taking the cosines of the holonomies of a connection around each loop in a given trinion decomposition. For an element $x=(A_i,B_i,D_j,C_j)$ in $\MM^G$, it is the elements $\{C_j\}$ which correspond to these holonomies, and so we want to define functions $f_k:\MM^G \to \mathbb{R}$ by 
	\begin{equation}
		\tilde{f}_k(A_i,B_i,D_j,C_j) = \frac{1}{\pi}\cos^{-1} \left(\Tr C_k\right)
	\end{equation}
	To get functions on $P$, one must check that these $\tilde{f}_k$ pass to the symplectic implosion. For the action of $G_k$ on $\MM^G$, with moment map $\Phi_k(A_i,B_i,D_j,C_j) = C_j^{-1}$, and a given face $\sigma$ of $\Delta$ we have
	\begin{equation}
		\Phi_k^{-1}(\sigma_0) = \{
		(A_i,B_i,D_j,C_j) \in \MM^G ~|~ C_k^{-1} \in \sigma
		\}
	\end{equation}
	Hence $\tilde{f}_k$ is constant on on $\Phi_k^{-1}(\sigma)$ and passes to the quotient by $[G_\sigma,G_\sigma]$. For $l != k$, the value of $\tilde{f}_l$ depends only on $C_l$, which is not acted on by $G_k$ and therefore $\tilde{f}_l$ also passes to the quotient. It remains to check that these functions pass under the quotient by the first copy of $G$ (TODO).
	
	Next we build a prequantum line bundle on $P$. On the locus where $\phi:\MM\to P$ is a symplectomorphism, we can simply define $\LL_P = \LL$, but we must discuss the extension to the rest of $P$. Recall that we defined a function $\Theta:\cA\times\cG\to U(1)$ for connections over $\Sigma$. For the singular curve $\Sigma_0$, we can define $\Theta$ exactly the same way, which would allow us to define an equivalence relation on $\cA_0\times \mathbb{C}$ by $(A,z)\sim (g\cdot A, \Theta(A,g)z)$. Then we can define a line bundle $\tilde{\LL} = \cA_0 \times \mathbb{C}/\sim$ over $\MM^T$. To get a bundle on $P$, we want to check that $\Theta$ passes well under the quotient map to each strata of the imploded cross-section $P$. On the generic locus where $\phi$ is a symplectomorphism, there is nothing to check, so we're left only with the subset where $\phi$ is merely surjective. 
	\begin{lemma}
		Let $V_i$ be a tubular neighbourhood of $C_i$ in $\Sigma$. Let $V_{0}$ denote the (disconnected) image of $V_i$ under the surface degeneration. Then for every pair $(A,g)\in\cA\times \cG$ on $\Sigma$ which degenerates to a pair $(A_0, g_0)$ on $\Sigma_0$, there exists liftings $(\bA,\bg)$ and $(\bA_0,\bg_0)$ such that
		\begin{equation}
			\int\limits_{V_i} \Tr(\bg^{-1}d\bg\wedge \bA) = \int\limits_{V_0} \Tr(\bg_0^{-1}d\bg_0\wedge \bA_0).
		\end{equation}
		\label{l:main-lemma}
	\end{lemma}
	\begin{proof}
		Recall that $V_i=Q_1 = \{(x,y)\in \mathbb{C}^2~|~ xy=1\}$, and it degenerates to $V_0 = Q_0 = \{(x,y)\in\mathbb{C}^2~|~xy=0\} = Q_{0,x} \cup Q_{0,y}$, where $Q_{0,x},Q_{0,y}$ are the locus with $x\neq0$ and $y\neq 0$ respectively. The curve $Q_1$ has a loop $\gamma_1$, and cutting $Q_1$ along $\gamma_1$ disconnects it into the components with $x>y$ and $y>x$. Let $Q_x$ denote where $x>y$, and $Q_y$ where $y>x$.
		
		Let $N=\{Q_t\}_{t\neq 0}\cong V_i\times(0,1]$. Then one possible lifting of $A$ from $V_i$ to $N$ is:
		\begin{equation}
			\bA = \frac{\alpha}{2}\left(\frac{dx}{x}-\frac{dy}{y} + \frac{dt}{t}\right)
		\end{equation}
		Now, since $t=xy$ we have $dt =ydx+xdy$ and therefore
		\begin{align*}
			\bA_x &= \frac{\alpha}{2}\left(\frac{dx}{x}-\frac{dy}{y} + \frac{ydx+xdy}{xy}\right)\\
			&= \alpha~ \frac{dx}{x}.
		\end{align*}
		Another possible lifting is 
		\begin{equation}
			\bA_y = \frac{\alpha}{2}\left(\frac{dx}{x}-\frac{dy}{y} -  \frac{dt}{t}\right) = -\alpha~\frac{dy}{y}.
		\end{equation}
		These liftings are equal on the curve $\gamma_1$. We define the lifting we will use, $\bA$ to be $\bA_x$ on $Q_x$ and $\bA_y$ on $Q_y$.
		
		For the gauge transformation $g(x,y)$ we can do a similar lifting process;
		\begin{equation}
			\bg(x,y,t) =\begin{cases}
			g\left(x,t/x\right), & x>y\\
			g\left(t/y,y\right), & y>x\\
			\end{cases}.
		\end{equation}
		Then the integral over $V_i$ becomes:
		\begin{align*}
			\int\limits_{V_i} \Tr(\bg^{-1}d\bg\wedge d\bA) = \int\limits_{Q_x} \Tr(g^{-1}(x,1/x)dg(x,1/x)\wedge \alpha~ \frac{dx}{x}) - \int\limits_{Q_y} \Tr(g^{-1}(1/y,y)dg(1/y,y)\wedge \alpha~ \frac{dy}{y})
		\end{align*}
		On the other hand, recall that $A_0$ is given by:
		\begin{equation}
			A_0 = \begin{cases}
			\alpha \frac{dx}{x}, & x\neq 0\\
			-\alpha \frac{dy}{y}, & y\neq 0\\
			\end{cases}
		\end{equation}
		which can be lifted to any three manifold $N'$ in a horizontal manner, i.e. by not adding any tangential part. Similarly, our gauge transformation $g_0$ coming from $g$ was defined by 
		\begin{equation}
		g_0(x,y) =\begin{cases}
		g\left(x,0\right), & x\neq 0\\
		g\left(0,y\right), & y \neq 0 \\
		\end{cases}.
		\end{equation}
		Therefore the integral over $V_0$ becomes
		\begin{align*}
		\int\limits_{V_0} \Tr(\bg_0^{-1}d\bg_0\wedge d\bA_0) = \int\limits_{Q_{x,0}} \Tr(g^{-1}(x,0)dg(x,0)\wedge \alpha~ \frac{dx}{x}) - \int\limits_{Q_{y,0}} \Tr(g^{-1}(0,y)dg(0,y)\wedge \alpha~ \frac{dy}{y}).
		\end{align*}
		Finally, all four of $Q_x,Q_y, Q_{x,0}$ and $Q_{y,0}$ are diffeomorphic to $S^2$ with two points removed, and a diffeomorphism between $Q_x$ and $Q_{x,0}$ is given by our local model for the curves. This diffeomorphism pulls back $g(x,0)$ to $g(x,x^{-1})$. The analogous equality holds for $Q_y$ and $Q_{y,0}$. Thus these two integrals are equal.
	\end{proof}
	\begin{theorem}
		Given a pair $(A,g) \in \mathcal{A}\times\mathcal{G}$ over $\Sigma$, which degenerates to the pair $(A_0, g_0)$ over $\Sigma_0$, we have
		\begin{equation}
			\Theta(A,g) = \Theta(A_0, g_0).
		\end{equation}
		In particular, this implies that if $(A,g)$ and $(B,h)$ both degenerate to $(A_0,g_0)$, then $\Theta(A,g)=\Theta(B,g)$
		 $\LL = \phi^\ast \LL_0 =$.
		\label{t:l=l0}
	\end{theorem}
	\begin{proof}
		From Equation (\ref{e:cs-gauge}) we know
		\begin{equation}
			\Theta(A,g) = \exp\left[-ik\int\limits_\Sigma \Tr(dg g^{-1}\wedge A)\right].
		\end{equation}
		
		Let $\{C_i\}_{i=1}^{3g-3}$ be any trinion decomposition of $\Sigma$. Let $V_i$ be a tubular neighbourhood in $\Sigma$ of each $C_i$. Then on $U:= \Sigma - \bigcup_{i=1}^{3g-3}V_i$, the degeneration does nothing, and we have $(A,g)|_U = (A_0,g_0)|_U$. Therefore, we have
		\begin{align*}
				\Theta(A,g) &= \exp\left[-ik\int\limits_\Sigma \Tr(dg~ g^{-1}\wedge A)\right]\\
				&= \exp\left[-ik\int\limits_U \Tr(dg~g^{-1}\wedge A)
				-ik \sum_{i=1}^{3g-3}\int\limits_{V_i} \Tr(dg~g^{-1}\wedge A)\right]\\
				&= \exp\left[-ik\int\limits_U \Tr(dg_0~g_0^{-1}\wedge A_0)
				-ik \sum_{i=1}^{3g-3}\int\limits_{V_{0,i}} \Tr(dg_0~g_0^{-1}\wedge A_0)\right],~ (\text{Lemma } \ref{l:main-lemma})\\
				&= \exp\left[-ik\int\limits_{\Sigma_0} \Tr(dg_0~ g_0^{-1}\wedge A_0)\right]\\
				&= \Theta(A_0,g_0).
		\end{align*}
	\end{proof}
	This theorem allows us to define the line bundle $\LL_P$ over $P$ by taking the quotient of $\LL$ under the relation $(x,z)\sim (y,z') \iff (\phi(x)=\phi(y), z=z')$. In particular, we have (???)
	\begin{equation}
		\phi^\ast \LL_P = \{(x,(y,z))\in \MM\times \LL_P ~|~ \phi(x) = y\} = \LL
	\end{equation}

\section{Invariance of $\LL$ under Degeneration.}
	\begin{theorem}
		For the prequantum systems $(M,\pi,\LL)$ and $(P,\pi_{P}, \LL_P)$, let $\JJ_\pi ~(\JJ_{\pi,P})$ be the sheaf of sections of $\LL ~(\LL_P)$ which are covariant constant on the fibres of $\pi ~(\pi_P)$. Then
		\begin{equation}
			H^0(\JJ_\pi, \MM) = H^0(\JJ_{\pi,P}, P).
		\end{equation}
	\end{theorem}
	\begin{proof}
		The degeneration $\phi:\MM\to\cP$ is surjective, and thus $\phi^\ast:H^0(\JJ_{\pi,0},P) \to H^0(\JJ_\pi,\MM)$ is injective. Hence it remains to show that $\phi^\ast$ is surjective.
		\smallskip
		
		From theorem \ref{t:l=l0}, we have that $\LL=\phi^\ast \LL_P$ and $\phi^\ast \omega_P = \omega$. Suppose $s\in H^0(\JJ_{\pi},\MM)$ and suppose $x \in P$. Let $y_1,y_2 \in \phi^{-1}(x)$; it is permitted that $y_1=y_2$. Then $\pi(y_1) = \pi_P\circ \phi(x) = \pi(y_2)$ so $y_1$ and $y_2$ are in the same fibre of $\pi$. Stalkwise we have
		\begin{equation}
			\LL_{y_1} \cong \LL_{P,x} \cong \LL_{y_2},
		\end{equation} 
		so it makes sense to ask if $s(y_1) = s(y_2)$. Since $s$ is covariant constant on the fibres of $\pi$, and the fibres of $\pi$ are (path?) connected \cite[thm 2.5]{jeffrey_bohr-sommerfeld_1992}, we must have $s(y_1)=s(y_2)$, and so it is well defined to let $s'\in H^0(\JJ_{\pi,P},P)$ be given by $s'(x) = s\left(\phi^{-1}(x)\right)$. Then
		\begin{equation}
			\phi^\ast s'(y) = s'(\phi(x)) = s(y),
		\end{equation}
		so $\phi^\ast$ is surjective.
		\end{proof}
	
	
	
		
	